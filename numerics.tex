\chapter{Numerical treatment}
\label{chap:num}
% * numerical treatments necessary
% * main goal: calculate reduced density operator
% * other things: absorption spectra, single trajectories


%%%%%%%%%%%%%%%%%%%%%%%%%%%%%%%%%%%%%%%%%%%%%%%%%%%%%%%%%%%%%%%%%%%%%%%%%%%%%%%
\section{Hierarchical Equations of Motion}
\label{sec:num.heom}
% * Tanimura HEOMs
%%%%%%%%%%%%%%%%%%%%%%%%%%%%%%%%%%%%%%%%%%%%%%%%%%%%%%%%%%%%%%%%%%%%%%%%%%%%%%%


%%%%%%%%%%%%%%%%%%%%%%%%%%%%%%%%%%%%%%%%%%%%%%%%%%%%%%%%%%%%%%%%%%%%%%%%%%%%%%%
\section{Stochastic Hierarchical Equations of Motion}
\label{sec:num.sheom}
% * main idea
%
% TODO more general cases?
%%%%%%%%%%%%%%%%%%%%%%%%%%%%%%%%%%%%%%%%%%%%%%%%%%%%%%%%%%%%%%%%%%%%%%%%%%%%%%%

In this section we present one of the main results of this work, namely a numerical method to solve the non-Markovian stochastic Schrödinger equation
\begin{equation}
  \partial \psi_t = -\ii h \psi_t + L\ZZ_t \psi_t - \adj{L}\int_0^t \alpha(t-s) \frac{\delta\psi_t}{\delta\ZZ_s} \dd s
  \label{eq:num.nmsse}
\end{equation}
without making use of the $O$-operator substitution.
Therefore we need to conceive a different way to deal with the nonlocality of the functional derivative with respect to time and the noise-process, as it prevents us from employing the common techniques for dealing with stochastic Schrödinger equations in the Markovian regime \cite{???}.

It turns out that for certain correlation functions the linear NMSSE~\ref{eq:num.nmsse} is formally equivalent to an infinite hierarchy of completely local stochastic differential equations.
% TODO Really?
Although we borrow the main idea from the hierarchical equations of motion presented in the last section, we need to deal with some peculiarities of our NMSSE first.

%%%%%%%%%%%%%%%%%%%%%%%%%%%%%%%%%%%%%%%%%%%%%%%%%%%%%%%%%%%%%%%%%%%%%%%%%%%%%%%
\subsection{Time derivation}
%TODO change title
\label{sub:num.sheom.time_deriv}

% TODO Make clear, that we dont imply normalization
The basic idea is to absorb the action of the functional derivative on $\psitZ$ into an auxiliary stochastic pure state
\begin{equation}
  \psitZ[1] := \int_0^t \alpha(t - s) \frac{\delta \psi_t(\ZZ)}{\delta \ZZ_s} \dd s.
  \label{eq:num.first_order}
\end{equation}
We recall that the integral boundaries arise only after we apply the derivative on states $\psi_t(\ZZ)$ that satisfy the additional condition $\delta \psitZ / \delta \ZZ_s = 0$ for $ s < 0$ and $s > t$.
Therefore we may write \autoref{eq:num.first_order} more concise as
\begin{equation*}
  \psitZ[1] = \left( \int \alpha(t - s) \frac{\delta}{\delta \ZZ_s} \dd s \right) \psitZ =: \adjZZ_t \psitZ
\end{equation*}
with the integrated functional derivation operator $\adjZZ_t$.

% TODO Hierarchies for arbitrary α?
The first step toward deriving an equation of motion for $\psitZ[1]$ is answering the question under which conditions a simple expression for $\dot\adjZZ_t$ exists.
Since the latter is given by $\dot\adjZZ_t = \int\dot\alpha(t-s) \, \delta / \delta \ZZ_s \dd s$ we assume an exponential bath correlation function
\begin{equation*}
  \alpha(t) = g \, \exp[-\gamma \abs{t} - \ii \Omega t] = g \exp[-\ii \Omega t] \left( \Theta(t) \exp[-\gamma t] + \Theta(-t) \exp[\gamma t] \right).
\end{equation*}
with the Heaviside function $\Theta$.
As the singular terms in the time derivative cancel we obtain
\begin{equation*}
  \dot\alpha(t) = g \exp[-\ii \Omega t] \left( (-\gamma - \ii\Omega) \Theta(t) \exp[-\gamma t] + (\gamma - \ii\Omega)\Theta(-t)\exp[\gamma t] \right).
\end{equation*}
Therefore we cannot exploit $\dot\adjZZ_t \propto \adjZZ_t$ on the level of operators in general.
Even the vanishing of the functional derivative $\delta \phi_t(\ZZ) / \delta \ZZ_s$ for $s > t$ with some arbitrary stochastic state $\phi_t(\ZZ)$ does not imply $\dot\adjZZ_t \phi_t(\ZZ) \propto \adjZZ_t\phi_t(\ZZ)$ as the following example shows:
Take $\phi_t(\ZZ) = \varphi \cdot (\ZZ_t + \ZZ_{t'})$ for some noise independent system state $\varphi$ and $0 < t' < t$.
It clearly satisfies the required boundary conditions, but
\begin{equation*}
  \dot\adjZZ_t \phi_t(\ZZ) = (\dot\alpha(0) + \dot\alpha(t-t')) \varphi = \left( -2\ii\Omega - (\gamma + \ii\Omega) \exp[-(\gamma + \ii\Omega) (t-t')] \right) \varphi
\end{equation*}
which is not proportional to
\begin{equation*}
  \adjZZ_t \phi_t(\ZZ) = (\alpha(0) + \alpha(t - t')) \varphi = g (1 + \exp(-\gamma + \ii \Omega)(t-t')) \varphi.
\end{equation*}
We see that the problematic first summand arises due to singular behavior of $\delta \phi_t(\ZZ) / \delta \ZZ_s$ for $s = t$.

% TODO Where.
Such problems do not occur once we restrict $\dot\adjZZ_t$ to solutions of our NMSSE~\ref{eq:num.nmsse} with vacuum initial conditions.
Indeed we see from \autoref{eq:???} that the functional derivative of $\psitZ$ at the upper boundary is regular and therefore has vanishing weight under the integral.
Hence we obtain for the time derivative of our integrated derivation operator
\begin{equation}
  % TODO Watch for change in \psitZ
  \dot\adjZZ_t \psitZ = - (\gamma + \ii \Omega) \adjZZ_t\psitZ
  \label{eq:num.dot_adjZZt}
\end{equation}
In the remaining work we use the shorthand notation $w = \gamma + \ii\Omega$.
% TODO Finalize!

%%%%%%%%%%%%%%%%%%%%%%%%%%%%%%%%%%%%%%%%%%%%%%%%%%%%%%%%%%%%%%%%%%%%%%%%%%%%%%%
\subsection{Linear Hierarchy}
\label{sub:num.sheom.lin}
% * derivation
% * terminator
%
% TODO Surpession by γ / Ω; differences, in the following combined into w

We now return to our non-Markovian stochastic Schrödinger equation; with the auxiliary stochastic state~\ref{eq:num.first_order} it can written as
\begin{equation*}
  \partial_t \psitZ = -\ii\Hsys\psitZ + L\ZZ_t\psitZ - \adj{L}\psitZ[1].
\end{equation*}
Hence if we can derive a tractable equation of motion for $\psitZ[1]$ we reduce the problems with a functional derivative to propagating a coupled system of simpler equations.
With the result from the last section \autoref{eq:num.dot_adjZZt} and the original NMSSE~\ref{eq:num.nmsse} we find
\begin{align}
  \partial_t (\adjZZ_t \psi_t) &= - w \adjZZ_t\psi_t + \adjZZ_t ( -\ii\Hsys + L\ZZ_t - \adj{L} \adjZZ_t ) \psi_t \nonumber \\
  &= (-\ii\Hsys - w + L\ZZ_t) \psit[1] + [\adjZZ_t, \ZZ_t] L\psit - \adj{L} \adjZZ_t \psit[1],
  \label{eq:num.dot_psi1}
\end{align}
where we use that $\adjZZ_t$ commutes with all system operators.
It is not surprising that the functional derivative reappears in the equation for $\psit[1]$; therefore we need to introduce another auxiliary state.
This scheme leads to an infinite hierarchy of auxiliary states defined by
\begin{equation}
  \psit[k] := \adjZZ_t \psit[k-1] = \adjZZ_t^k \psit.
  \label{eq:num.auxiliary_states}
\end{equation}
Expressed in the new auxiliary states and with $[\adjZZ_t, \ZZ_s] = \alpha(t-s)$ \autoref{eq:num.dot_psi1} reads
\begin{equation*}
  \partial_t \psit[1] = (-\ii\Hsys - w + L\ZZ_t) \psit[1] + \alpha(0) L\psit[0] - \adj{L}\psit[2].
\end{equation*}
Along these lines it is straightforward to derive the full hierarchy of equations of motions for all $\psit[k]$.
Since the commutator $[\adjZZ_t, \ZZ_s]$ is a $\Complex$-number each auxiliary state only couples to the order directly above and below
% TODO More steps?
\begin{equation}
  \partial_t\psit[k] = (-\ii\Hsys - kw + L\ZZ_t)\psit[k] + k \alpha(0) \psit[k-1] - \adj{L} \psit[k+1].
  \label{eq:num.hierarchy_lin}
\end{equation}
The vacuum initial condition for the true quantum trajectory $\delta \psi_0 / \delta \ZZ_s = 0$ requires that all auxiliary states vanish at $t=0$.

% TODO Check grammar
Of course the infinite hierarchy is even more intricate to solve than the original non-Markovian stochastic Schrödinger equation; therefore we need to truncate at some finite order.
It is quite remarkable that this can be done in a self-consistent manner, which even approximately incorporates all truncated orders into the remaining equations.
We start by transforming \autoref{eq:num.linear_hierarchy} into an equivalent integral equation.
Notice that
\begin{align}
  \psit[k + 1] = &\int_0^t \exp[-(k+1)w(t-s)] \Texp[\int_s^t -\ii\Hsys + L\ZZ_u \dd u] \nonumber \\
  &\left( (k+1) \alpha(0)L\psi_s^{(k)} - \adj{L} \psi_s^{(k+2)} \right) \dd s
  \label{eq:num.terminator_integral}
\end{align}
% TODO "Formally" correct here?
formally satisfies the corresponding equation of motion together with the required initial condition.

% TODO Continue!

%%%%%%%%%%%%%%%%%%%%%%%%%%%%%%%%%%%%%%%%%%%%%%%%%%%%%%%%%%%%%%%%%%%%%%%%%%%%%%%
\subsection{Nonlinear Hierarchy}
\label{sub:num.sheom.nonlin}
% * derivation

We mention in \autoref{sec:nmqsd.nonlin_nmsse} how the scaling of our Monte-Carlo sampling with the number of realizations improves drastically we use a nonlinear version.
This can be achieved within our microscopical using the comoving coherent states defined in \autoref{eq:nmqsd.comoving_flow}.
Up to a certain extend this method can also be employed for our hierarchical equations of motion.

As a function of the coherent state labels $\cc\zz$ instead of the process $\ZZ$ we define comoving auxiliary states by
\begin{equation*}
  \tilde\psi_t^{(k)}(\cc\zz) := (\psit[k] \circ \vec\phi_t)(\cc\zz) = \psit[k](\vec\phi_t(\cc\zz)).
\end{equation*}
Then the same steps that give \autoref{eq:nmqsd.nmsse_nonlin} lead to a similar result
% TODO More steps?
\begin{align}
  \partial_t \tilde\psi_t^{(k)}(\cc\zz) &= \left( -\ii\Hsys - kw + L\ZZ_t + L \int_0^t \cc{\alpha(t-s)} \qmean{\adj L}_s \dd s \right)\tilde\psi_t^{(k)}(\cc\zz) \nonumber\\
  &+ k \alpha(0) L \tilde\psi_t^{(k-1)} - (\adj{L} - \qmean{\adj L}_t) \tilde\psi_t^{(k+1)}(\cc\zz),
  \label{eq:num.hierarchy_nonlin}
\end{align}
with the normalized expectation value taken with respect to the true quantum trajectory---or put differently with respect to the zeroth order auxiliary state
\begin{equation*}
  \qmean{\adj L}_s = \frac{\bra{\tilde\psi_t^{(0)}} \adj{L} \ket{\tilde\psi_t^{(0)}}}{\braket{\tilde\psi_t^{(0)}}{\tilde\psi_t^{(0)}}}.
\end{equation*}
Notice that the exponential correlation function necessary for the hierarchy also simplifies the treatment of the memory-term in \autoref{eq:num.hierarchy_nonlin}.
% TODO Improve!
Indeed we can derive a simple and closed evolution equation for it.

But one caveat remains: for the convolutionless formulation we can go one step further and even derive an equation for normalized pure state trajectories~\ref{eq:nmqsd.nmsse_nonlin_full}.
The same does not hold true for the hierarchical equations of motion.
% TODO Elaborate!

% TODO Terminator

%%%%%%%%%%%%%%%%%%%%%%%%%%%%%%%%%%%%%%%%%%%%%%%%%%%%%%%%%%%%%%%%%%%%%%%%%%%%%%%
\subsection{Multimodes}
\label{sub:num.sheom.nonlin}
% * rectangular vs triangular truncation
% TODO Change title
% TODO Seperate section in chapter 2?
% TODO Reference to Schulten paper?

Of course most physically interesting systems cannot be modeled with only a single exponential bath mode.
We now present the changes necessary to accommodate a more general environmental structure given by a finite number of exponential modes coupling to the system with arbitrary operators.
As the crucial points do not depend on the choice of linear or nonlinear version we are only concerned with the former in this section.

% TODO Grammar!
The linear non-Markovian stochastic Schrödinger equation for a finite number $N$ of independent environments may be derived along the same lines presented in \autoref{sec:nmqsd.lin_nmsse}.
It reads
\begin{equation*}
  \partial_t \psit = -\ii\Hsys\psit + \sum_{j=1}^N L_j \ZZ_{j, t} \psit - \sum_{j=1}^N \adj{L}_j \int_0^t \alpha_j(t - s) \frac{\delta \psit}{\delta \ZZ_{j, t}} \dd s
\end{equation*}
with independent noise processes satisfying
\begin{equation*}
  \E\,Z_{i, t} = 0, \quad \E\,Z_{i, t} Z_{j, s} =0, \quad\mbox{and}\quad \E\,Z_{i, t} \ZZ_{j, s} = \delta_{ij}\alpha_i(t-s).
\end{equation*}
Remember that our hierarchical equations of motion rely on an exponential bath correlation function; therefore we need to introduce several derivation operators $\adjZZ_{j, t}$, one for each process.
Just as in the single-mode case, the auxiliary states are obtained by successively applying $\adjZZ_{i, t}$ on the quantum trajectory $\psitZ$
\begin{equation}
  \psit[k_1, \dots, k_N] := \adjZZ_{1, t}^{k_1} \dots \adjZZ_{N, t}^{k_N} \psit.
  \label{eq:num.auxiliary_states_multi}
\end{equation}
Note that all $\adjZZ_{j, t}$ mutually commute.
Consequently \autoref{eq:num.auxiliary_states_multi} is the most general form how a functional derivative may occur in the derivation of the corresponding hierarchy.
Hereinafter we use the shorthand notation $\psit[\kk]$ for the auxiliary state defined above.
Similar to~\ref{eq:num.hierarchy_lin} its equation of motion reads
\begin{equation}
  \partial_t\psit[\kk] = (-\ii\Hsys - \kk\cdot\ww + \sum_j L_j \ZZ_{j, t})\psit[\kk] + \sum_j k_j \alpha_j(0) \psit[\kk - \ee_j] - \sum_j \adj{L}_j \psit[\kk + \ee_j],
  \label{eq:num.hierarchy_lin_multi}
\end{equation}
where $\ee_j$ denotes the $j$-th unit vector in $\Reals^N$ and $\kk\cdot\ww = \sum_j k_jw_j$ is the euclidean scalar product.\footnote{Although $\ww$ is complex in general no complex conjugation occurs at any time.}

% TODO Do this with pure tikz
\begin{figure}
  \centering
  \begin{subfigure}[b]{.4\columnwidth}
    \centering
    \includegraphics[scale=.85]{img/cubic_hierarchy}
    \caption{Quadratic}
    \label{fig:num.trunc_cubic}
  \end{subfigure}
  \begin{subfigure}[b]{.4\columnwidth}
    \centering
    \includegraphics[scale=.85]{img/triang_hierarchy}
    \caption{Triangular}
    \label{fig:num.trunc_tria}
  \end{subfigure}
  \caption{Comparison of the two truncation schemes in the special case of $N=2$ processes with order $D$.}
% TODO Insert more blabla
  \label{fig:num.trunc}
\end{figure}
% TODO Too much "states"
When it comes to truncating the hierarchy~\ref{eq:num.hierarchy_lin_multi} the most obvious strategy is simply to cut off each mode separately at given order $D$.
In other words the truncation condition reads $0 \le k_j \le D$ for all $j=1,\dots,N$; any auxiliary state not satisfying it is set to zero.
We refer to this scheme as \quotes{cubic truncation scheme} since the shape of all states under consideration resembles an $N$-cube (see also \autoref{fig:num.trunc_cubic}).
Clearly the number of auxiliary states scales exponentially like $(D+1)^N$, which makes the treatment of physically interesting systems with $N$ in the order of $100$ absolutely impossible.

Examining \autoref{eq:num.hierarchy_lin_multi} closer we notice that the term responsible for suppression of the $\kk$-th order auxiliary state is $\exp(-\kk \cdot \ww)$.
Instead of treating each mode individually we use a condition better suited for the product $\kk\cdot\ww$, namely $0 \le \abs{\kk} \le D$ with $\abs{\kk} = \sum_j k_j$.
For $N=2$ the corresponding states form a triangular shape as shown in \autoref{fig:num.trunc_tria}, hence the name \quotes{triangular truncation}.
The appropriate generalization to arbitrary $N$ is a simplex where the number of elements is given by
\begin{equation*}
  \sum_{d=0}^D {d + N - 1 \choose N - 1} \approx % TODO Stirling, Polynomial
\end{equation*}
showing a much softer scaling compared to the cubic scheme.
% TODO Polynomial for large N?
In \autoref{fig:num.scaling} we display the number of auxiliary states required for a given number of modes and truncation order $D$.
The difference is even more pronounced for larger $N$ required in the study of realistic systems.
% TODO Done that? Really?
As an example take the Fenna-Matthews-Olson complex further investigated in \autoref{sec:num.fmo}:
In a simplified model we attach one independent environment with 13 exponential terms to each of its seven sites for a total of 91 modes.
This yields an insurmountable number of about $10^{30}$ auxiliary states for the cubic truncation, compared to only 5151 using the triangular scheme.

\begin{figure}
  \centering
  \includegraphics{img/scaling.pdf}
  \caption{Comparison\dots}
  \label{fig:num.scaling}
\end{figure}

% TODO Mention manual cutoff????

%%%%%%%%%%%%%%%%%%%%%%%%%%%%%%%%%%%%%%%%%%%%%%%%%%%%%%%%%%%%%%%%%%%%%%%%%%%%%%%
\section{Correlation Function Expansion}
\label{sec:num.expansion}
% * pade spectrum decomposition
% * also mention other (Matsubara e.g.)
% * uniqueness? Theoretically yes, in practices doesnt matter!
% * Plot for Ohmic spectral density (where do these appear?)
% * Show highly structured density, approximation!
%%%%%%%%%%%%%%%%%%%%%%%%%%%%%%%%%%%%%%%%%%%%%%%%%%%%%%%%%%%%%%%%%%%%%%%%%%%%%%%



%%%%%%%%%%%%%%%%%%%%%%%%%%%%%%%%%%%%%%%%%%%%%%%%%%%%%%%%%%%%%%%%%%%%%%%%%%%%%%%
\section{Spin-Boson Model}
\label{sec:num.spin_boson}
% * short intro
% * Depth-dependence
% * dependence on number of expansion terms
% * single trajectories
%%%%%%%%%%%%%%%%%%%%%%%%%%%%%%%%%%%%%%%%%%%%%%%%%%%%%%%%%%%%%%%%%%%%%%%%%%%%%%%


%%%%%%%%%%%%%%%%%%%%%%%%%%%%%%%%%%%%%%%%%%%%%%%%%%%%%%%%%%%%%%%%%%%%%%%%%%%%%%%
\section{FMO-Complex}
\label{sec:num.fmo}
% * model
%%%%%%%%%%%%%%%%%%%%%%%%%%%%%%%%%%%%%%%%%%%%%%%%%%%%%%%%%%%%%%%%%%%%%%%%%%%%%%%


%%%%%%%%%%%%%%%%%%%%%%%%%%%%%%%%%%%%%%%%%%%%%%%%%%%%%%%%%%%%%%%%%%%%%%%%%%%%%%%
\subsection{Absorption Spectra}
\label{sub:num.fmo.absorption}
% * derivation of formula
% * why NMSSE so cool for it

%%%%%%%%%%%%%%%%%%%%%%%%%%%%%%%%%%%%%%%%%%%%%%%%%%%%%%%%%%%%%%%%%%%%%%%%%%%%%%%
\subsection{Transfer Dynamics}
\label{sub:num.fmo.dynamics}

\cite{PlMGr86_casimir}
