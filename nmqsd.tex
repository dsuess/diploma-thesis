\chapter{Non-Markovian Quantum State Diffusion}
\label{chap:nmqsd}
% * lin/nonlin Markovian SSE
% * little History
% * alternatives standard (projection?), pseudomodes
% * relevant/irre
%
% FIXME Citations

The description of open quantum systems in terms of diffusive stochastic differential equations has a long tradition \cite{}.
At first, seen merely as a tool to unravel a given master equation of Lindblad type, it was realized later that they posses a strong microscopical foundation in terms of continuous measurements or memoryless quantum environments \cite{}.\\



% ✔ same applies to nmqsd unravelling: \cite{St96_lin_nmqsd}; microscopical theory: ...
%     => later is described in first section
% ✔ on its basis we derive a linear NMSSE, completely equivalent to micro. model, but numerical inferior
% * key point in hierachy later

Its non-Markovian generalization, the non-Markovian quantum state diffusion (NMQSD), took a quite similar path, which we roughly follow in this chapter:
Although first discovered as an unravelling for the Feynman-Vernon influence functional in terms of stochastic propagators \cite{St96_lin_nmqsd}, the corresponding non-Markovian stochastic Schrödinger equation (NMSSE) was derived based upon the standard open-system-model, which we \cm{recall} in \autoref{sec:nmqsd.model}.
Following the lines of Diósi, Strunz and Gisin \cite{DiSt97_nmsse,DiGiSt98_nmqsd,StDiGi99_nmq_traj} we derive both a linear and numerically superior non-linear version of the NMSSE in \autoref{sec:nmqsd.lin_nmsse} and~\ref{sec:nmqsd.nonlin_nmsse} respectively.

% * interpreation no so clear as markovian case
% * finite temperature, since derivation depends on vacuum initial bath state
%
% * study exemplary system, propose direct solution for T=0

\Autoref{sec:nmqsd.interpretation} is concerned with the question if our NMSSE has a physical interpretation or is just merely a computational tool.
%FIXME
\cm{Anschließend} we drop the requirement of zero initial temperature used in the \cm{vorherig} sections.
This chapter is \cm{abgeschlossen} by the treatment of an analytically soluble two-level system at zero temperature.\\

%FIXME Reference ok?
Most of the material covered can be found in reference \cite{St01_habil}, which we follow loosely.


%%%%%%%%%%%%%%%%%%%%%%%%%%%%%%%%%%%%%%%%%%%%%%%%%%%%%%%%%%%%%%%%%%%%%%%%%%%%%%%
\section{The Microscopical Model}
\label{sec:nmqsd.model}
% ✔ standard model (why oscillators, why linear coupling?)
% ✔ reservoir/environment
% ✔ initial states
%
% TODO Physical examples for such a model
% TODO Why harmonic oscillators for bosonic bath
% TODO Picture
%
%%%%%%%%%%%%%%%%%%%%%%%%%%%%%%%%%%%%%%%%%%%%%%%%%%%%%%%%%%%%%%%%%%%%%%%%%%%%%%%

It is the foremost goal of this work to obtain a dynamical equation for an open quantum system.
Nevertheless we introduce a full model of system and its environment first, the bosonic and non-relativistic standard model of an open quantum system extensively studied for example in the book of Weiss \cite{We99_dissipative_systems}.
There are three reasons for such a microscopical approach:
On one hand this serves the purpose to better understand the physical origin of macroscopical properties used to characterize the bath later on.
But more important, starting with a closed quantum system is the only strategy allowing us to derive the NMSSE from first principles, namely the Schrödinger equation.
%FIXME
A last argument in favor of the microscopic approach, PRODUCT STATES, ENTANGLEMENT, etc. ---we will not dwell on this any further.\\


As a starting point we consider an environment consisting of a finite number $N$ of uncoupled harmonic oscillators\footnote{%
  We use \quotes{environment}, \quotes{reservoir} and \quotes{bath} interchangeably, altough the later two suggest a large size compared to the system.
}.
A Generalization to an infinite number can be carried out formally along the same lines, replacing sums by infinite series or even integrals; a different approach within our framework is presented later.
The dynamics of both system and environment are then described by a unitary time evolution with the Hamiltonian
\begin{equation}
  \Htot = \Hsys \otimes \unit  +  \unit \otimes \Henv  +  \Hint,
  \label{eq:nmqsd.Htot}
\end{equation}
where $\Hsys$ and $\Henv$ are the free Hamiltonians of the system and the bath respectively.\footnote{%
  For some models like the damped harmonic oscillator \cite{CaLe83_diss_system} an additional renormalization term arises from the interaction.
  Nevertheless such a contribution is best attributed to $\Hsys$ since it only acts on the system's Hilbert space.
}
The latter is a sum over independent harmonic oscillators $\Henv = \sum_\lambda \omega_\lambda \adj{a}_\lambda a_\lambda$ expressed in bosonic ladder operators $a_\lambda$ and $\adj{a}_\lambda$ of the $\lambda$\th mode with frequency $\omega_\lambda$.
Treating a finite number of independent reservoirs poses no further difficulties and therefore is not elaborated in this section.

For the interaction between environment and system we confine ourselves to the case of linear coupling
\begin{equation}
  \Hint = \sum_\lambda \cc{g}_\lambda \, L \otimes \adj{a}_\lambda + g_\lambda \, \adj{L} \otimes a_\lambda.
  \label{eq:nmqsd.Hint}
\end{equation}
Here $L$ denotes the coupling operator in the system's Hilbert space and $g_\lambda \in \Complex$ the coupling strength of the $\lambda$\th mode.
% FIXME More details, taylor expansion?
Since in typical examples the coupling of an individual bath mode scales inversely with the environment size \cite{We99_dissipative_systems}, the linear coupling in~\ref{eq:nmqsd.Hint} seems reasonable for macroscopic large environments.
%FIXME stop repetition in "environment"
But our framework also incorporates small environments---even to the extreme of a single harmonic oscillator---with strong coupling as well.
For such cases the linearity needs to be imposed as another assumption of the model.

Beside the Hamiltonian another important influence on the system's subdynamics is the initial state, specifically the initial entanglement between system and bath.
Throughout this work we only consider product initial conditions, where the bath is in the vacuum state with respect to all $a_\lambda$
\begin{equation}
  \ket{\Psi_0} = \ket{\psi_0} \bigotimes\limits_\lambda \ket{0_\lambda}.
  \label{eq:nmqsd.initial_conditions}
\end{equation}
Such a choice is not as restrictive as it seems on first glance: In \autoref{sec:nmqsd.temperature} we show how a thermal bath state can be mapped to~\ref{eq:nmqsd.initial_conditions}.
% FIXME cite Richard
However whether the NMSSE is applicable to initially entangled states is a question of current research.\\



To absorb the free dynamics of the environment in time dependent creation and annihilation operators, we switch to the interaction picture with respect to $\Henv$.
Since the bath operators only obtain an additional phase $\exp[\pm \ii \omega_\lambda t]$, the transformed Hamiltonian from \autoref{eq:nmqsd.Htot} reads\footnote{%
  We refrain from introducing another label to distinguish between time-evolution pictures---in what follows we always work in the interaction picture.
}
\begin{equation}
  \Htot(t) = \Hsys \otimes \unit  +  \sum_\lambda \left( \cc{g}_\lambda \exp[\ii \omega_\lambda t] \, L \otimes \adj{a}_\lambda + g_\lambda \exp[-\ii \omega t] \, \adj{L} \otimes a_\lambda \right).
  \label{eq:nmqsd.Htot}
\end{equation}
Our choice of unentangled initial conditions with a vacuum bath state ensures that the reduced density operator remains unaffected under the change of time-evolution picture.

It is instructive to rewrite the last equation using the operator valued force
\begin{equation}
  B(t)=\sum_\lambda g_\lambda a_\lambda \exp[-\ii\omega_\lambda t].
  \label{eq:nmqsd.force_operator}
\end{equation}
The total Hamiltonian then reads $\Htot(t) = \Hsys \otimes \unit  +  L \otimes \adj{B(t)}  +  \adj{L} \otimes B(t)$.
Already from this equation it can be seen, that the complete action of the environment on the system is encoded in the operators $B(t)$.
An important---and within our model the only---characteristic of them is the correlation function $\alpha(t-s) = \big\langle  (B(t) + \adj{B(t)})(B(s) + \adj{B(s)}) \big\rangle_\rho$ for an arbitrary bath state $\rho$.
For a thermal state at temperature $T$, the correlation function can be calculated analytically \cite{FeHi10_path_integrals}
\begin{equation}
  \alpha_T(t - s) = \sum_\lambda  \abs{g_\lambda}^2  \left( \operatorname{coth} \frac{\omega_\lambda}{2T} \, \cos \omega_\lambda (t-s)  -  \ii \sin \omega_\lambda(t-s) \right).
  \label{eq:nmqsd.thermal_correlation_function}
\end{equation}
Introducing the spectral density $J(\omega) = \sum_\lambda \abs{g_\lambda}^2 \delta(\omega - \omega_\lambda)$ and taking the limit $T \to 0$, the above equation can be rephrased as
\begin{equation}
  \alpha(t - s) = \qmean{B(t)\adj{B(s)}}_0 = \int_0^\infty J(\omega) \exp[-\ii\omega (t-s)] \dd \omega.
  \label{eq:nmqsd.correlation_function}
\end{equation}
In other words, the correlation function is simply given as one-sided Fourier transform of the spectral density.
% TODO Check this!
Of course this connection between response function and power spectrum---and its general form for $T\neq0$---is well known as fluctuation-dissipation relation.
Since a genuine physical spectral density is real, we require admissible correlation function to be hermitian $\alpha(-t) = \cc{\alpha(t)}$.\\

% TODO Types of correlation function, incommensurate frequencies, generalization --> contiuum-approximation, exp. decay, Markov, inclussion of negative frequencies,


%%%%%%%%%%%%%%%%%%%%%%%%%%%%%%%%%%%%%%%%%%%%%%%%%%%%%%%%%%%%%%%%%%%%%%%%%%%%%%%
\section{Linear NMSSE}
\label{sec:nmqsd.lin_nmsse}
% ✔ Bargman States, hilbert space valued functions
% ✔ derivation
% * problems, non-locality in noise
% ✔ reduced density operator
% * relative state --> interpretation; but also connection with H_s valued functions
% ✔ zero temperature, importance for calculations
% * quantum trajectory Carmichael
%
% TODO relation to unravelling
% TODO Make clear that interaction picture does not matter for purely system observables
%
%%%%%%%%%%%%%%%%%%%%%%%%%%%%%%%%%%%%%%%%%%%%%%%%%%%%%%%%%%%%%%%%%%%%%%%%%%%%%%%

The linear non-Markovian stochastic Schrödinger equation derived in this section is an equivalent reformulation of the interaction-picture Schrödinger equation
\begin{equation}
  \partial_t \ket{\Psi_t} = -\ii \Htot(t) \ket{\Psi_t}, \qquad \ket{\Psi_0} = \ket{\psi_0} \otimes \ket{0},
  \label{eq:nmqsd.schroedinger_ia}
\end{equation}
corresponding to the model of the last section:
Expressing the bath degrees of freedom in the Bargmann Hilbert space of anti-holomorphic functions\cite{Ba61_coherent_states} provides a representation that is well suited for a Monte-Carlo treatment.
To this end we introduce the unnormalized coherent state $\ket{z_\lambda} = \exp(z_\lambda \adj{a}_\lambda)\ket{0_\lambda}$ for each mode with resolution of the identity for the environment
\begin{equation}
  \unit = \int \frac{\exp[-\abs{\zz}^2]}{\pi^N} \, \ket{\zz}\bra{\zz} \dd^{2N}z,
  \label{eq:nmqsd.identity}
\end{equation}
Here we employ the shorthand notation $\ket{\zz} = \bigotimes_\lambda \ket{z_\lambda}$ and the \quotes{volume} integration measure for $N$ complex numbers $\dd^{2N}z = \prod_\lambda \dd\Re z_\lambda \dd\Im z_\lambda$.
Throughout this work the finite bath is often replaced by a continuum of oscillators; therefore we simply write $\mudz = \pi^{-N} \exp(-\abs{\vec z}^2) \dd^{2N}z$ to drop an explicit reference to $N$.
% TODO Rigorous existence?

\Autoref{eq:nmqsd.identity} allows us to express the full state in a time-independent environment basis
\begin{equation*}
  \ket{\Psi_t} = \int \ket{\psi_t(\cc\zz)} \otimes \ket{\zz} \mudz.
\end{equation*}
For the following derivation it is crucial to notice that the Bargmann transform $\zz \mapsto \psi_t(\cc\zz)$ is an anti-holomorphic function with values in the system's Hilbert space $\HHsys$.
Naturally it is equivalent to any other representation of the full state $\ket{\Psi_t}$.
As the coherent states are not orthogonal, but rather satisfy $\braket{\vec w}{\vec z} = \exp(\sum_\lambda \cc w_\lambda z_\lambda)$, the reduced density operator, obtained by tracing over the bath degrees of freedom, reads
%TODO Does this need a prove?
\begin{equation}
  \rho(t) = \Tr{env} \ket{\Psi_t}\bra{\Psi_t}
          = \int \ket{\psi_t(\cc\zz)}\bra{\psi_t(\cc\zz)} \mudz.
  \label{eq:nmqsd.reduced_matrix}
\end{equation}

% Only Schrödinger equation point of view!
% FIXME Citations of Strunz papers
After fixing the kinematic structure, the next step is to rewrite the dynamical equation:
The representation of the ladder operators follow from the usual rules $\bra\zz \adj{a}_\lambda = \cc z_\lambda \bra\zz$ and $\bra\zz a_\lambda = \partial_{\cc z_\lambda} \bra\zz$.
These expressions applied to \autoref{eq:nmqsd.schroedinger_ia} give us the system-bath Schrödinger equation in the transformed space
\begin{equation}
  \partial_t \psi_t(\cc\zz) = -\ii\Hsys\psi_t(\cc\zz)  -  \ii L \sum_\lambda \cc g_\lambda \exp[-\ii\omega_\lambda t] \cc z_\lambda \, \psi_t(\cc\zz)  -  \ii \adj{L} \sum_\lambda g_\lambda \exp[\ii\omega_\lambda t] \, \frac{\partial \psi_t}{\partial z_\lambda}(\cc\zz).
  \label{eq:nmqsd.hamiltonian_microsopic}
\end{equation}
Introducing an effective driving process like in \autoref{eq:nmqsd.force_operator}
\begin{equation}
  \ZZ_t(\cc\zz) = - \ii \sum_\lambda \cc g_\lambda \exp[\ii \omega_\lambda t] \cc z_\lambda
  \label{eq:nmqsd.stochastic_process}
\end{equation}
allows us to combine the effect of the first bath-interaction term into a single multiplication operator---or process for reasons explained in the next paragraph.
% FIXME Proof of functional chain rule?
A similar conversion works for the second term as well with the help of the functional chain rule $\frac{\partial}{\partial \cc z_\lambda} = \int \frac{\partial \ZZ_s}{\partial\cc z_\lambda} \frac{\delta}{\delta \ZZ_s} \dd s$.
Combined our new equation of motion ---the non-Markovian stochastic Schrödinger equation---reads
\begin{equation}
  \partial_t \psi_t = -\ii\Hsys\psi_t  +  L\ZZ_t\psi_t  -  \adj{L} \int_0^t \alpha(t-s) \frac{\delta \psi_t}{\delta \ZZ_s} \dd s.
  \label{eq:nmqsd.nmsse}
\end{equation}
As we shown in \autoref{sub:nmqsd.interpretation.unitary_view}, the integral boundaries arise due to the initial conditions~\ref{eq:nmqsd.initial_conditions}; a less formal argumentation goes as follows:
By construction of our processes~\ref{eq:nmqsd.stochastic_process} an initial state $\ket{\psi_0}\otimes\ket{0}$ translates to an initial $\psi_0(\ZZ)$ that is completely independent of any noise.
Then causality implies that $\psi_t$ can only depend on $\ZZ_s$ for $0 \le s \le t$.\\



Up to this point we have merely rewritten the original Schrödinger equation~\ref{eq:nmqsd.schroedinger_ia} to an equivalent form:
The original system-bath product Hilbert space $\HHsys \otimes \HHenv$ is replaced by a Hilbert space of $\HHsys$-valued functions.
A different attitude is quite fruitful, especially with a numerical solution of the NMSSE in mind:
\Autoref{eq:nmqsd.reduced_matrix} can be rewritten as $\rho_t = \E[\ket{\psi_t}\bra{\psi_t}]$, where $\E$ denotes the average over $\mudz = \pi^{-N} \exp(-\abs{\vec z}^2) \dd^{2N}z$.
Put differently the reduced density matrix $\rho_t$ arises by averaging over the stochastic pure state projectors $\ket{\psi_t(\cc\zz)}\bra{\psi_t(\cc\zz)}$ with Gaussian weight $\mudz$.
Hence we regard \autoref{eq:nmqsd.nmsse} as a stochastic differential equation for individual realisations $\psi_t(\cc\zz)$.
We refer to the later either as system state relative to $\ket{\zz}$ or, in the spirit of the stochastic Schrödinger equations emerging from continuous measurement theory \cite{Ca93_quantum_optics}, as quantum trajectory.

In this approach the driving force $\ZZ_t$ is implemented as classical stochastic process defined by the concrete version~\ref{eq:nmqsd.stochastic_process} and the underlying probability measure $\mu$.
It is a complex Gaussian process uniquely characterized by its expectation value and covariances
\begin{equation}
  %FIXME Z_t compared to \ZZ_s looks strange! The subscript is moved.
  \E\,Z_t = 0, \quad \E\,Z_t Z_s =0, \quad\mbox{and}\quad \E\,Z_t \ZZ_s = \alpha(t-s),
  \label{eq:nmqsd.process_properties}
\end{equation}
where $\alpha$ is the zero-temperature correlation function~\ref{eq:nmqsd.correlation_function} for $J(\omega) = \sum_\lambda \abs{g_\lambda}^2 \delta(\omega - \omega_\lambda)$.
By virtue of the initial conditions, $\psi_t$ depends on $\zz$ only through the driving process; thus we can drop the coherent state labels and simply write $\psi_t(\ZZ)$ denoting the trajectory corresponding to a realisation $\ZZ(\cc\zz)$.
It is this alternative point of view that makes the NMSSE-approach so powerful:
The entire influence of the environment is encoded in a complex function $\alpha$, which acts both as correlation function for the driving noise $\ZZ$ and as memory kernel for the damping term.
A generalization to an arbitrary number of bath-oscillators is now straightforward: simply replacing the correlation function allows an unified description of arbitrary harmonic environments.

Except in the limit $\alpha(t) \propto \delta(t)$, elaborated in the next paragraph, the driving process $\ZZ_t$ is correlated for different times.
This non-Markovian behavior, which makes a complete understanding of the dynamics highly desirable for application but also considerably harder, shows up in the equation of motion~\ref{eq:nmqsd.nmsse} as well.
The damping term contains the functional derivative over the whole timespan and therefore takes the complete history of $\psi_t(\ZZ)$ into account.
In its own right the derivative is just as problematic:
% FIXME Quite long sentence!
Since its computation requires not only the single realisation $\ZZ$, but in some sense all adjacent ones as well, it seems questionable to regard the NMSSE~\ref{eq:nmqsd.nmsse} as a genuine stochastic differential equation \cite{GaWi02_real_nmsse}.
% FIXME DOOMED???
Even from the purely pragmatic point of view both kinds of non-local behavior complicate a direct numerical simulation of the NMSSE, if not making it completely impracticable.
Nevertheless there are two quite distinct solutions as shown in \autoref{sub:nmqsd.lin_nmsse.convolutionless} and \autoref{chap:num}.


%%%%%%%%%%%%%%%%%%%%%%%%%%%%%%%%%%%%%%%%%%%%%%%%%%%%%%%%%%%%%%%%%%%%%%%%%%%%%%%
\subsection{Markov Limit}
\label{sub:nmqsd.markov}
% * markov limit
% * problem with negative energy oscillators
% * relation to lindblad/markovian sse
% * Ito vs Stratonovich
%
% FIXME Polish introduction
% FIXME Is dropping γ wise?

The best understood open systems are Markovian.
Based upon two physical assumptions, namely
% FIXME Good style?, What does memoryless actually mean?
\begin{description}
  \item[weak coupling] of the system to the reservoir and
  \item[memoryless environment,] that is the time evolution is completely time-local,
\end{description}
it is possible to derive a general form of a master equation governing the reduced dynamics \cite{Li76_generators_qdsg}.
Of course, the NMSSE is much more general.
It is only in the standard Markovian limit $\alpha(t) = \gamma\delta(t)$ we can expect to obtain an equation of motion that describes a reduced time evolution without memory.
A rescaling of the coupling operator $L$ allows us to set $\gamma = 1$ without loss of generality.

The vacuum initial conditions $\frac{\delta \psi_0}{\delta \ZZ_s} = 0$ with $s \in \Reals$ imply for an arbitrary bath correlation function
\begin{equation}
  \frac{\delta \psi_t}{\delta \ZZ_t} = \frac{1}{2} \, L \psi_t \qquad (t > 0)
  \label{eq:nmqsd.deriv_psit}
\end{equation}
as we show now.
% FIXME Really? Only due to interaction picture.
It is clear from its derivation that the NMSSE describes a unitary, time-dependent evolution.
Therefore it can be solved formally using the Dyson series
\begin{equation}
  \psi_t(\ZZ) = \sum_{n=0}^\infty (-\ii)^n \intl{0}{t}{t_1} \intl{0}{t_1}{t_2} \dots \intl{0}{t_{n-1}}{t_n}  \Htot(t_1) \dots \Htot(t_n) \, \psi_0,
  \label{eq:nmqsd.dyson}
\end{equation}
where $\Htot(t)$ is the reformulation of~\ref{eq:nmqsd.Htot} given by
\begin{equation*}
  -\ii \Htot(t) = -\ii \Hsys + L \ZZ_t - \adj{L} \intl{-\infty}{\infty}{s} \alpha(t-s) \frac{\delta}{\delta \ZZ_s}.
\end{equation*}
Throughout this work we often use the shorthand notation $\adjZZ_t = \int\mathrm{d}s \, \alpha(t-s) \frac{\delta}{\delta \ZZ_s}$ for the last term.

% FIXME More details?, Use Commutator notation instead?
Applying the functional derivative $\frac{\delta}{\delta \ZZ_s}$ to $\Htot(t)$ gives a single contribution $\ii \delta(t - s) L$, since both $\Hsys$ and $\adjZZ_t$ are independent of the noise.
% FIXME
% FIXME Talk about this with W
% FIXME
This allows us to calculated $\frac{\delta \psi_t}{\delta \ZZ_t}$ order by order in \autoref{eq:nmqsd.dyson}---the derivative of $\psi_0$ vanishes as imposed by the initial conditions.
We obtain for the term with a $n$-fold time-integral neglecting a constant phase
\begin{equation*}
  \intl{0}{t}{t_1} \dots \intl{0}{t_{n-1}}{t_n} \Big( \delta(t_1 - t) L \Htot(t_2) \dots \Htot(t_n) + \dots + \delta(t_n - t) \Htot(t_1) \dots L \Big).
\end{equation*}
We notice that the $i$\th summand contributes only if $t_i = t$:
For $i = 1$ this is exactly the integral boundary while for $i=2,\dots,n$ the integral boundary reaches $t$ only for $t_1 = t$.
As the latter condition has vanishing weight under the $t_1$ integral we eventually find \autoref{eq:nmqsd.deriv_psit}.

Let us return to the Markov limit of our NMSSE\@.
By virtue of the singular correlation function $\alpha = \delta$ the time-nonlocal damping operator reduces to a time-local form $\adjZZ_t = \frac{\delta}{\delta \ZZ_t}$ as it is expected from a memoryless environment.
Combined with \autoref{eq:nmqsd.deriv_psit} this leads to simple stochastic differential equation
\begin{equation*}
  \partial_t \psi_t(\ZZ) = -\ii \Hsys \psi_t(\ZZ) + L\ZZ_t\psi_t(\ZZ) - \frac{1}{2}\adj{L}L\psi_t(\ZZ),
\end{equation*}
driven by a complex White Noise $Z_t$ with $\E{Z_t \ZZ_s} = \delta(t-s)$.
In a formally exact fashion the equation above should be written as
\begin{equation}
  \dd\psi_t = (-\ii\Hsys\psi_t - \frac{1}{2} \adj{L}L \psi_t) \dd t + L\psi_t \dd \cc\xi_t
  \label{eq:nmqsd.ito}
\end{equation}
with a standard complex Brownian motion $\xi_t$.
It is well known that such stochastic differential equations are problematic as $\xi_t$ is not differentiable with respect to time.
To define the solution $\psi_t$ uniquely we need to specify an appropriate interpretation of the stochastic differential equation \cite[p.~36]{Ok03_sde}:
We imagine the Brownian motion as a limit of stochastic processes $\xi^{(n)}_t \to \xi_t$, such that $\xi^{(n)}_t$ are continuously differentiable with respect to time.
Replacing the Brownian motion in \autoref{eq:nmqsd.ito} by $\xi^{(n)}_t$ transforms it into a deterministic differential equation.
The limit of corresponding solutions $\psi^{(n)}_t$ coincides with $\psi_t$ only if we understand \autoref{eq:nmqsd.ito} it the Stratonovich sense.
However, in our case the It\=o- and Stratonovich form agree since $\E Z_t Z_s = 0$ \cite{GaCr85_handbook}.\\

% TODO Uncomment if cool
%The Belavkin or simply stochastic Schrödinger equation~\ref{eq:nmqsd.ito} is a well known result in continuous measurement theory and quantum optics, where it appears as an unravelling of the Linblad master equation \cite{BaGr09_trajectories,???}.
%Nevertheless our main ingredient in its derivation, namely the singular bath correlation function $\alpha = \delta$, shows how unphysical the Markov assumption is:
%As $\alpha$ is given by the Fourier transform of the spectral density $J$, this amounts to a system coupled evenly to Oscillators of arbitrary frequency.
%Besides
%% TODO Complete!
%% negative frequencies, no problem here; timescales

%%%%%%%%%%%%%%%%%%%%%%%%%%%%%%%%%%%%%%%%%%%%%%%%%%%%%%%%%%%%%%%%%%%%%%%%%%%%%%%
\subsection{Convolutionless Formulation}
\label{sub:nmqsd.lin_nmsse.convolutionless}
% * on the existence
% * why it solves problems, true stochastic equation
% * dynamics
% * application

% FIXME
As a cure for the non-locality issues, Diósi, Gisin, and Strunz \cite{DiGiSt98_nmqsd} proposed the powerful $O$-Operator substitution:
It is based on the additional assumption, that one may replace the functional derivative by a system operator $O$, which only depends on the realisation of $\ZZ$ itself
\begin{equation}
  \frac{\delta \psi_t(\ZZ)}{\delta \ZZ_s} = O(t, s, \ZZ) \psi_t(\ZZ).
  \label{eq:nmqsd.o_substition}
\end{equation}
Besides getting rid of the derivative, this substitution enables us to derive a convolutionless form of our NMSSE~\ref{eq:nmqsd.nmsse}
\begin{equation}
  \partial_t \psi_t = -\ii\Hsys\psi_t(\ZZ)  +  L\ZZ_t\psi_t(\ZZ)  -  \adj{L} \bar O(t, \ZZ) \psi_t(\ZZ)
  \label{eq:nmqsd.nmsse_o}
\end{equation}
with the time-local operator
\begin{equation}
  \bar O(t, \ZZ) := \int_0^t \alpha(t - s) O(t, s, \ZZ) \dd s.
  \label{eq:nmqsd.o_bar}
\end{equation}
Conclusively \autoref{eq:nmqsd.nmsse_o} turns into a genuine stochastic differential equation for the trajectory $\psi_t(\ZZ)$, but in the much smaller Hilbert space of the system.
This makes it exceptionally well suited for dealing with infinite sized environments numerically, provided the $\bar O$-operator is known.
Depending on the validity of the $O$-substitution the corresponding convolutionless NMSSE~\ref{eq:nmqsd.nmsse_o} might be as accurate as the original microscopic equation of motion~\ref{eq:nmqsd.schroedinger_ia}.

For a few simple models---for example the dissipative two level system presented in \autoref{sec:nmqsd.twolevel} or its higher dimensional generalizations \cite{JiZhYo12_exact_nmqsd}---an exact analytic expression for $O$ is known.
In these rare cases one proceeds as follows \cite{DiGiSt98_nmqsd}:
From the consistency condition
\begin{equation}
  \partial_t \frac{\delta \psi_t(\ZZ)}{\delta \ZZ_s} = \frac{\delta}{\delta \ZZ_s} \partial_t \psi_t(\ZZ)
  \label{eq:nmqsd.consistency_condition}
\end{equation}
and the initial condition familiar from \autoref{sub:nmqsd.markov}
% FIXME This does not aggree with Markov!!!
\begin{equation}
  O(s, s, \ZZ) = L
  \label{eq:nmqsd.o_initial}
\end{equation}
we derive an equation of motion for $O(t, s, \ZZ)$.
It still contains the functional derivative, but is converted to a system of coupled, deterministic equations using a power series ansatz
\begin{equation}
  O(t, s, \ZZ) = \sum_{n=0}^\infty \int_0^t \dots \int_0^t O_n(t, s, \nu_1, \dots, \nu_n) \dd \nu_1 \dots \nu_n.
  \autoref{eq:nmqsd.o_series}
\end{equation}
Nevertheless most treatments rely on approximation schemes, for example a perturbation expansion for small coupling parameter or almost-Markovian environments \cite{YuDiGiSt99_pertubation}.
% FIXME Citation
Also a closely related hierarchy of $O$-operators provides an efficient numerical algorithm similar in concept to the main result of this work \cite{}.

%%%%%%%%%%%%%%%%%%%%%%%%%%%%%%%%%%%%%%%%%%%%%%%%%%%%%%%%%%%%%%%%%%%%%%%%%%%%%%%
\subsection{Equivalent Master Equations}
\label{sub:nmqsd.lin_nmsse.master}
%TODO Change title
%TODO More references to prior work?
% * existence of master equation
% * relation to lindblad

In the previous section we have introduced a convolutionless formulation primarily to simplify the treatment of the NMSSE\@.
But the $O$-operator substitution is also essential clarify the connection to the master equations commonly used in the theory of open quantum systems.
The latter are formulated in terms of reduced density operators, which we recover from the trajectories by averaging over the pure states projectors $P_t = \ket{\psi_t(\ZZ)}\bra{\psi_t(\ZZ)}$.
For certain systems this can be done analytically in order to derive a equivalent master equation.

As a simple example we focus on models with a $\ZZ$ independent $\bar O$-operator such as the two-level system presented in \autoref{sec:nmqsd.twolevel}.
We follow the lines of Yu et al.~\cite{YuDiGiSt99_pertubation,YuDiGi00_master}, who also treat the general case using the functional expansion~\ref{eq:nmqsd.o_series}.
The pure states projectors' equations of motion,
\begin{equation}
  \partial_t P_t = -\ii [\Hsys, P_t] + \ZZ_t L P_t - \adj{L}\bar O(t)P_t + Z_t P_t \adj{L} - P_t \adj{\bar O(t)} L,
  \label{eq:nmqsd.pt_eom}
\end{equation}
yield a closed evolution equation for $\rho_t$ after averaging over the bath degrees of freedom only if we can restate the terms containing $\ZZ_t$ in a noise-independent manner.
This can be done with the help of Novikov's formula \cite{No65_functionals}
\begin{equation}
  \E[Z_t P_t] = \E[\intdd s \alpha(t - s) \frac{\delta}{\delta \ZZ_s} P_t].
  \label{eq:nmqsd.novikov}
\end{equation}
% \FIXME
A formal proof is provided in \autoref{sec:???}, but the main idea is simple:
Under a Gaussian integral $\intdd^2 z \, \exp(-\abs{z}^2) \dots$ the multiplication by $z$ can be rewritten as a derivation $\partial_{\cc z}$.
Partial integration yields a result similar to \autoref{eq:nmqsd.novikov}.

The right hand side of Novikov's formula is simplified further using the $O$-operator substitution.
Since $\ket{\psi_t}$ is analytical in $\ZZ$ and accordingly $\bra{\psi_t}$ analytical in $Z_t$, the derivative is further simplified to
\begin{equation*}
  \frac{\delta}{\delta \ZZ_s} \bigg( \ket{\psi_t(\ZZ)}\bra{\psi_t(\ZZ)} \bigg) = \left( \frac{\delta}{\delta \ZZ_s} \ket{\psi_t(\ZZ)} \right)\bra{\psi_t(\zz)} = O(t, s) \ket{\psi_t(\ZZ)}\bra{\psi_t(\ZZ)}.
\end{equation*}
Averaging over the equations of motion for the pure state projectors~\ref{eq:nmqsd.pt_eom} finally gives the master equation for the reduced density matrix $\rho_t$
\begin{equation}
  \partial_t \rho_t = -\ii [\Hsys, \rho_t]  +  [L, \rho_t \adj{\bar O(t)}]  +  [\bar O(t) \rho_t, \adj{L}].
  \label{eq:nmqsd.master}
\end{equation}
%FIXME Reference
This expression closely resembles the well known Lindblad master equation~\ref{eq:intro.} for Markovian open quantum systems, but involves time-dependent Linbladians.
%FIXME Correct form of O(s,s,Z)
As elaborated in \autoref{sub:nmqsd.markov} the $\bar O$-operator reduces to $\bar O(t) = \frac{\gamma}{2} L$ in the Markovian limit---thus our NMSSE reproduces the correct limit.


%%%%%%%%%%%%%%%%%%%%%%%%%%%%%%%%%%%%%%%%%%%%%%%%%%%%%%%%%%%%%%%%%%%%%%%%%%%%%%%
\section{Nonlinear NMSSE}
\label{sec:nmqsd.nonlin_nmsse}
% * non-uniquesness of unravelling ==> used here to change weights
% * Motivation (Monte Carlo!, normalized states)
% * derivation
% * discussion
%
% TODO We are going back to the microscopical model!
% TODO Ask Strunz if non-analyticity of Φ needs to be mentioned explicitely
% TODO Mention Girsanov
%%%%%%%%%%%%%%%%%%%%%%%%%%%%%%%%%%%%%%%%%%%%%%%%%%%%%%%%%%%%%%%%%%%%%%%%%%%%%%%

%TODO Remove either
In the last section we emphasize that the non-Markovian stochastic Schrödinger equation~\ref{eq:nmqsd.nmsse} can be interpreted either as a system-environment Schrödinger equation or as a stochastic differential equation for the trajectories $\psi_t(\ZZ)$.
% TODO Unravelling earlier?; This is the first time this equation is written down explicitely!!!
In the theory of Markovian systems the latter is often referred to as an unravelling of the corresponding master equation \cite{???}.
Either way in a numerical treatment the reduced density operator of the open system is determined by a Monte-Carlo evaluation of the partial trace (or stochastic average)
% TODO This has been defined earlier!!!
\begin{equation}
  \rho_t = \E[\ket{\psi_t}\bra{\psi_t}] = \int \ket{\psi_t(\zz)}\bra{\psi_t(\zz)} \mudz.
  \label{eq:nmqsd.reduced_matrix}
\end{equation}
The fineness of such a scheme is drastically reduced if there are few highly peaked contributions \cite{DuSh11_monte_carlo}.
% TODO Really?
As shown in a numerical investigation \cite{???} the NMSSE shows exactly this behavior: for most trajectories the norm $\braket{\psi_t(\ZZ)}{\psi_t(\ZZ)}$ goes to zero due to growing entanglement with the environment.
To recover the unitarity of the closed time evolution $\E[\braket{\psi_t}{\psi_t}] = \braket{\Psi_t}{\Psi_t} = 1$ the few trajectories with significant contribution have to be taken into consideration.
As we further elaborate in \autoref{sec:???} this requires an insurmountable sample size for certain system parameters.

% TODO This doesnt fit too well here
Seen purely as a stochastic tool to determine the reduced density operator the unravelling in \autoref{eq:nmqsd.reduced_matrix} is not unique:
we can perform any transformation under the integral which keeps its value fixed and obtain equally well relative states with a different measure.
As it improves the behavior of the Monte-Carlo evaluation noticeably we perform a change of measure such that the average can be taken over normalized states.
Such a procedure is well known from the theory of Markovian stochastic Schrödinger equation \cite{???} and results in a nonlinear equation of motion---the same is true for our non-Markovian approach.
% TODO Really?
Since the general case is treated later we focus on the convolutionless formulation of \autoref{eq:nmqsd.nmsse_o}.


Of course it is trivial to rewrite \autoref{eq:nmqsd.reduced_matrix} as an average over normalized states
\begin{equation*}
  \rho_t = \int \frac{\mathrm{d}^{2N} z}{\pi^N} \, \exp[-\abs{\zz}^2] \braket{\psitz}{\psitz} \, \frac{\ket{\psitz}\bra{\psitz}}{\braket{\psitz}{\psitz}},
\end{equation*}
now expressed with a time dependent density function.
To highlight the physical significance of the norm we notice that the latter is just the Q- or Husimi-function\footnote{We point out that the Husimi function is usually defined in terms of normalized coherent states. Hence the additional factor $\exp(-\abs{z}^2)$ for each oscillator in our notation.} of the bath given by \cite{Sc11_quantum_optics}
\begin{equation}
  Q_t(\zz, \cc\zz) = \frac{\exp[-\abs{\zz}^2]}{\pi^N}\, \bra{\zz} \Tr{sys} \big( \ket{\Psi_t}\bra{\Psi_t} \big)\ket{\zz}
                   = \frac{\exp[-\abs{\zz}^2]}{\pi^N}\, \braket{\psitz}{\psitz}.
  \label{eq:nmqsd.husimi}
\end{equation}
Expressed in terms of $Q_t$ the reduced density operator reads
\begin{equation}
  \rho_t = \int Q_t(\zz, \cc\zz) \, \frac{\ket{\psitz}\bra{\psitz}}{\braket{\psitz}{\psitz}} \mathrm{d}^{2N} z.
  \label{eq:nmqsd.rho_in_Q}
\end{equation}
% TODO IMPROVE!!!!
Due to being non-negative and normalized to unity $\int Q(\zz, \cc\zz) \dd z = 1$ the Husimi-function can be regarded as the (quasi)-probability distribution on phase space of the bath degrees of freedom:
Since a coherent state $\ket{z}$ resembles a wave packet localized around $z = (q + \ii \, p) / \sqrt{2}$, there is a well defined correspondence between coherent state labels $z$ and the canonical variables $(q, p)$.
% TODO This is just copied, rephrase!
Hence the norm of $\psitz$ simply determines the probability to find the bath oscillators in the coherent state $\ket{\zz}$.\\

We can now incorporate the dynamics of the environment in a comoving coherent state basis for our trajectories $\psitz$.
Making use of the microscopic Hamiltonian~\ref{eq:nmqsd.hamiltonian_microsopic} and the analyticity $\partial_{z_\lambda} \ket{\psitz}$ gives the time evolution of the Husimi-function
\begin{equation}
  \partial_t Q_z(\zz, \cc\zz) = - \sum_\lambda \partial_{\cc z_\lambda} \big( \ii g_\lambda \exp[-\ii \omega_\lambda t] \, \qmean{\adj L}_t \, Q_t(\zz, \cc\zz) \big) - \mathrm{c.c.}
  \label{eq:nmqsd.qdot}
\end{equation}
It is obvious that the equation of motion above contains the full back-reaction of the system due to the quantum average\footnote{We do not indicate its (non-holomorphic) dependence on $\cc\zz$ explicitly because our main goal is not the solution of \autoref{eq:nmqsd.eq:nmqsd.qdot}. Instead $Q_t$ is only used to derive normalized versions of our NMSSE-trajectories.}
\begin{equation*}
  \qmean{\adj L}_t = \frac{\bra{\psitz} \adj L \ket{\psitz}}{\braket{\psitz}{\psitz}}.
\end{equation*}
% TODO Check this
Remarkably \autoref{eq:nmqsd.qdot} can be solved with the method of characteristics since it has exactly the form of a (complex) Liouville equation.
The corresponding characteristic curves are are described by
\begin{equation}
  \cc{\dot z}_\lambda(t) = \ii g_\lambda \exp[-\ii \omega_\lambda t] \qmean{\adj L}_t.
  \label{eq:nmqsd.zdot}
\end{equation}
% TODO Phi holom/nonholom.?
We denote the corresponding flow by $\vec\phi_t$; hence by the usual abuse of notation $\cc z_\lambda(t) = \cc \phi_{\lambda,t}(\cc z_\lambda)$ with initial conditions $\cc z_\lambda(0) = \cc\phi_{\lambda, 0}(\cc z_\lambda) = \cc z_\lambda$.
\Autoref{eq:nmqsd.zdot} tells us that if the full state at time $t$ is $\ket{\psitz} \otimes \ket{\zz}$ and therefore the Husimi-function is localised around $\zz = (\vec q + \ii \, \vec p) / \sqrt{2}$ then the dominant contribution at $t + \Delta t$ comes from the coherent state $\ket{\zz + \dot\zz \Delta t}$.
For this reason we should use system states relative to $\ket{\zz(t)}$ instead in order to avoid vanishing contributions of single trajectories to $\rho_t$.

By construction the flow $\vec\phi_t$ also gives as a solution to \autoref{eq:nmqsd.qdot} for the Husimi-function
\begin{equation*}
  Q_t(\zz, \cc\zz) = \int Q_0(\zz_0, \cc\zz_0) \, \delta(\zz - \vec\phi_t(\zz_0)) \dd^{2N} z_0
\end{equation*}
where $\delta(\zz - \zz') = \prod_\lambda \delta(\Re(z_\lambda - z_\lambda')) \delta(\Im(z_\lambda - z_\lambda'))$.
% TODO Add reference
Since at the beginning our total state is given by the product $\ket{\Psi_t} = \ket{\psi_0} \otimes \ket{\zz}$ the initial condition for the Husimi-function reads $Q_0(\zz, \cc\zz) = \pi^{-N} \exp[-\abs{\zz}^2]$ as seen from \autoref{eq:nmqsd.husimi}.
With $\psi'_t = \psi_t \circ \vec\phi_t$ we can rewrite \autoref{eq:nmqsd.reduced_matrix} for the reduced density matrix as\footnote{Instead of directly introducing the normalized states $\tilde\psi_t$ as done by Strunz \cite{St01_habil} we explicitly define the interim state $\psi'_t$ to compare with the corresponding result for the hierarchy in \autoref{sub:num.sheom.nonlin}.}
\begin{equation}
  \rho_t = \int \frac{\mathrm{d}^{2N}z}{\pi^N} \, \exp[-\abs{\zz}^2] \, \frac{\ket\psitphi \bra\psitphi}{\braket{\psitphi}{\psitphi}}
         = \E[ \frac{\ket{\tilde\psi_t}\bra{\tilde\psi_t}}{\braket{\tilde\psi_t}{\tilde\psi_t}}].
  \label{eq:nmqsd.reduced_matrix_comoving}
\end{equation}
By its definition $\tilde\psi_t(\cc\zz)$ is just the relative state of $\ket{\Psi_t}$ belonging to the coherent state $\ket{\vec\phi_t(\cc\zz)}$.
Put differently these are exactly tho expansion coefficients of the full system-bath pure state in the comoving environmental basis.
It is quite remarkable that a closed equation of motion for the $\tilde\psi_t$ can be derived starting with
\begin{equation}
  \partial_t (\psi_t \circ \cc{\vec\phi}_t) = \partial_t\psi_t \circ \cc{\vec\phi}_t + \sum_\lambda (\partial_{\cc z_\lambda} \psi_t \circ \cc{\vec\phi}_t) \cdot (\partial_t \ccphitla).
  \label{eq:nmqsd.psiprime_dot}
\end{equation}
For the first term we can use the evolution equation~\ref{eq:nmqsd.nmsse_o} of the fixed-basis relative states $\psi_t$ with the $O$-operator substitution in place.
Replacing the coherent state labels by their comoving counterparts leads us to a shifted process:
The integral form of \autoref{eq:nmqsd.zdot}
\begin{equation}
  \ccphitla(\cc z_\lambda) = \cc z_\lambda + \ii g_\lambda \int_0^t \exp(-\ii \omega_\lambda s) \qmean{\adj L}_s \dd s
  \label{eq:nmqsd.comoving_flow}
\end{equation}
plugged into the microscopic version of the process~\ref{eq:nmqsd.stochastic_process} yields the shifted stochastic driving as
\begin{equation}
  \tildeZZ_t(\cc\zz) := \ZZ_t(\cc{\vec\phi}_t(\cc\zz)) = \ZZ_t(\cc\zz) + \int_0^t \cc{\alpha(t-s)} \qmean{\adj L}_s \dd s.
  \label{eq:stochastic_process_shiften}
\end{equation}
% TODO Mkay?
Since the $O$-operator substitution ensures that the equations of motion for $\psi_t$ are local with respect to $\ZZ$, the comoving dynamics just amount to replacing $\ZZ_t$ by $\tildeZZ_t$ in the first addend of~\ref{eq:nmqsd.psiprime_dot}.

% TODO Stress time independence of probabilty
The second addend, due to the intrinsic time dependence of the shifted coherent states, is treated on the same footing:
It is just the well-known functional derivative term from the NMSSE as \autoref{eq:nmqsd.zdot} reveals:
\begin{align*}
  \sum_\lambda \frac{\partial\ccphitla}{\partial t}(\cc z_\lambda) \cdot \frac{\partial\psi_t}{\partial \cc z_\lambda} (\cc{\vec\phi}_t(\cc\zz))
  &= \ii \sum_\lambda g_\lambda \exp[-\ii \omega_\lambda t] \qmean{\adj L}_t \, \frac{\partial\psi_t}{\partial \cc z_\lambda} (\cc{\vec\phi}_t(\cc\zz)) \\
  &= \qmean{\adj L}_t \, \int_0^t \alpha(t - s) \frac{\delta \psi_t}{\delta \ZZ_s} (\cc{\vec\phi}_t(\cc\zz)) \dd s \\
  &= \qmean{\adj L}_t \bar O(t, \tildeZZ) \tilde\psi_t(\tildeZZ),
\end{align*}
where the last line reflects the definition of the $\bar O$-operator in \autorefs{eq:nmqsd.o_substition} and~\ref{eq:nmqsd.o_bar}.
Both terms of~\ref{eq:nmqsd.psiprime_dot} combined yield the desired closed equation for $\tilde\psi_t$
\begin{equation}
  \partial_t \tilde\psi_t = -\ii\Hsys \tilde\psi_t + L\tildeZZ_t\tilde\psi_t - (\adj L - \qmean{\adj L}_t) \bar O(t, \tildeZZ) \tilde\psi_t.
  \label{eq:nmqsd.nmsse_nonlin}
\end{equation}
We want to recall that $\tilde\psi_t$ was introduced to allow averaging over normalized states in \autoref{eq:nmqsd.reduced_matrix_comoving}.
It does not imply $\tilde\psi_t(\tildeZZ)$ being normalized for all times, which would be favourable for interpreting the NMSSE as an stochastic equation for genuine pure system states.
It is quite remarkable that an extended version of \autoref{eq:nmqsd.nmsse_nonlin} exists that even preserves normalization of single realizations:
By considering the trajectories $\ket{\psi'_t} = \ket{\tilde\psi_t} / \sqrt{\braket{\tilde\psi_t}{\tilde\psi_t}}$ it is straightforward to derive the corresponding equation of motion \cite{???}
\begin{align}
  \partial_t\psi'_t &= -\ii\Hsys\psi'_t  +  \left(L - \qmean{L}_t\right) \tildeZZ_t\psi'_t  \nonumber \\
  &-  \left( (\adj{L} - \qmean{\adj L}_t) \bar O(t, \tildeZZ) - \qmean{(\adj{L} - \qmean{\adj L}_t) \bar O(t, \tildeZZ)} \right) \psi'_t
  \label{eq:nmqsd.nmsse_nonlin_full}
\end{align}
% TODO Copy to the paragraph above?
As mentioned in the motivation the nonlinear equations should be given precedence over the linear version when it comes to Monte-Carlo simulation.
They allow us to compute the density matrix as an average over realizations with same order of magnitude while restoring the reference measure (in the microscopic model) to the well known, time-independent Gaussian weight~\ref{eq:nmqsd.identity}.
Therefore generating realizations of the shifted processes $\tildeZZ_t$ is within the scope of general methods, as long as the expectation value $\qmean{\adj L}_s$ for all times $0 \le s \le t$ is known.
This is the only contribution to the nonlinear NMSSE which is explicitly time-nonlocal;
but since it involves only the average value there is no storage problem for numerical application.
% TODO OK?
Of course these comments neglect the question how to obtain the $\bar O$-operator, which contains all non-Markovian feedback of the environment.
For to application to any realistic physical system---except a few exactly solvable ones---this is actually the critical part in the implementation.


%%%%%%%%%%%%%%%%%%%%%%%%%%%%%%%%%%%%%%%%%%%%%%%%%%%%%%%%%%%%%%%%%%%%%%%%%%%%%%%
\section{Interpretation of NMSSE}
\label{sec:nmqsd.interpretation}
% * Interpreatation of Markovian SSE
%
% TODO Pure state for open systems problematic
%%%%%%%%%%%%%%%%%%%%%%%%%%%%%%%%%%%%%%%%%%%%%%%%%%%%%%%%%%%%%%%%%%%%%%%%%%%%%%%

In \autoref{sec:nmqsd.lin_nmsse} we propose two different schemes how to interpret our linear non-Markovian stochastic Schrödinger equation~\ref{eq:nmqsd.nmsse} from a formal point of view:
On one hand we treat it like a simple reformulation of the microscopic Schrödinger equation~\ref{eq:nmqsd.hamiltonian_microsopic}.
Alternatively the convolutionless formulation~\ref{eq:nmqsd.nmsse_o} resembles more closely the common stochastic differential equations.

% TODO Details on continous measurement
In the Markovian regime the latter perspective is often favoured\dots

It remained an open question for a long time whether the non-Markovian stochastic Schrödinger equation allows

%%%%%%%%%%%%%%%%%%%%%%%%%%%%%%%%%%%%%%%%%%%%%%%%%%%%%%%%%%%%%%%%%%%%%%%%%%%%%%%
\subsection{Linear NMSSE as Schrödinger Equation}
\label{sub:nmqsd.interpretation.unitary_view}
% * Hilbert space of Noise processes
% * functional Tailor expansion
% * calculation rules

%%%%%%%%%%%%%%%%%%%%%%%%%%%%%%%%%%%%%%%%%%%%%%%%%%%%%%%%%%%%%%%%%%%%%%%%%%%%%%%
\subsection{A Time-Nonlocal Picture}
\label{sub:nmqsd.interpretation.time_osci}
%TODO Change title
% * time oscillators
% * structure of interaction
% * make contact with Brownian motion picture
% * breakdown of interpretation for T ≠ 0


%%%%%%%%%%%%%%%%%%%%%%%%%%%%%%%%%%%%%%%%%%%%%%%%%%%%%%%%%%%%%%%%%%%%%%%%%%%%%%%
\section{Finite Temperature Theory}
\label{sec:nmqsd.temperature}
% * why necessary
% * why approach does not work anymore
%
% TODO quantum vs. classical noise
% TODO How it solves the negative frequency problem
%%%%%%%%%%%%%%%%%%%%%%%%%%%%%%%%%%%%%%%%%%%%%%%%%%%%%%%%%%%%%%%%%%%%%%%%%%%%%%%

Until now we were only concerned with the temperature zero theory, which was defined by an initial product state with the environment in the vacuum state $\ket{\Psi_0} = \ket{\psi_0} \otimes \ket{\vec 0}$.
It translates into our NMSSE-framework as the demand of vanishing functional derivatives at the time $t = 0$
\begin{equation*}
  \frac{\delta \psi_0}{\delta \ZZ_s} = 0 \quad (s \in \Reals).
\end{equation*}
% TODO What about U_s→ t L U^*_s→ t
This allows us to restrict the integral domain for the derivatives in \autoref{eq:nmqsd.nmsse} and is therefore crucial for the $O$-operator substitution.
Without the upper limit $t$ for the integral the hierarchical equations of motion presented in \autoref{sec:num.sheom} fail as well.
In order to treat non-zero temperature systems with the non-Markovian stochastic Schrödinger equation we devise two methods that map to the vacuum initial conditions of the zero-temperature case.

To start off we assume a product initial state, but this time with a Gibbs state $\rho(\beta) = \frac{\exp[-\beta \Henv]}{Z}$ on the environment's side
\begin{equation}
  \rho_0 = \ket{\psi_0}\bra{\psi_0} \otimes \rho(\beta)
  \label{eq:nmqsd.initial_rho}
\end{equation}
with the bath partition function $Z = \Tr \exp[-\beta \Henv]$ at inverse temperature $\beta = 1/k_B T$.
This choice amounts to the following experimental setting:
At $t_\mathrm{init} \to -\infty$ the environment is brought into contact with an even larger heat bath at given temperature, while the coupling to the system is switched off.
% TODO What about the bath--superbath entanglement?
% TODO BAD GRAMMAR!
The environment is allowed to thermalize until $t=0$ when the super-bath is removed and the system-coupling is instantly tuned as given by \autoref{eq:nmqsd.Hint}.
% TODO Really? Elaborate!
The choice of a pure state projector for the system in \autoref{eq:nmqsd.initial_rho} is merely for convenience;
in contrast we cannot drop its product form.
% TODO Too complicated?
Therefore a initial thermal state of the system and bath with respect to $\Htot$ cannot be treated with the NMSSE due to entanglement.


%%%%%%%%%%%%%%%%%%%%%%%%%%%%%%%%%%%%%%%%%%%%%%%%%%%%%%%%%%%%%%%%%%%%%%%%%%%%%%%
\subsection{Thermo Field Method}
\label{sub:nmqsd.temperature.thermofield}
% * method
%
% TODO Physical interpreatation of the thermal occupation number prefactors? Spontaneous and induced excitation/relaxation?
% TODO Purely real α --> classical thermal noise

Thermo field dynamics was introduced as a real-time approach to quantum fields at finite temperature \cite{???}.
It is favored over other methods in application to the NMSSE since it does not change the equation of motion~\cite{DiGiSt98_nmqsd} as shown below.
In the course of this section we follow the slightly more detailed accounts of Yu and Strunz \cite{Yu04_heat_bath,St01_habil}.

The main idea is to introduce a second fictitious bath of oscillators, which is independent from the physical environment and does not interact with the system.
Expressing its degrees of freedom in ladder operators $b_\lambda$ and $\adj{b}_\lambda$ gives us the new Hamiltonian in the Schrödinger picture
\begin{equation}
  \Htot = \Hsys \otimes \unit + \sum_\lambda (\cc{g}_\lambda L\otimes\adj{a}_\lambda + g_\lambda \adj{L}\otimes a_\lambda) + \unit \otimes \sum_\lambda \omega_\lambda (\adj{a}_\lambda a_\lambda - \adj{b}_\lambda b_\lambda).
  \label{eq:nmqsd.Htot_thermal}
\end{equation}
Although this Hamiltonian is not bounded from below due to negative frequencies of the fictitious oscillators, there are no stability problems since they do not interact with the physical degrees of freedom.
For the same reason the reduced dynamics obtained from \autoref{eq:nmqsd.Htot_thermal} are identical to the original microscopical model~\ref{eq:nmqsd.Htot}.
Therefore both yield equal reduced density matrices for our system provided we choose an initial state that reproduces \autoref{eq:nmqsd.initial_rho} upon tracing of the unphysical degrees of freedom.
% TODO Mhhh...
Since these are given by a product the choice of a total initial state is independent from the system and is equivalent to demand
\begin{equation}
  \Tr{b} \tilde\rho = \rho(\beta)
  \label{eq:nmqsd.rho_tilde}
\end{equation}
for the density matrix of both environments.
Here $\Tr{b}$ denotes the partial trace with respect to the fictitious degrees of freedom.

Remarkably a solution $\tilde\rho$ of \autoref{eq:nmqsd.rho_tilde} is given by the pure state projector on a vacuum state with respect to new annihilation operators $A$, $B$.
They are connected to the old ladder operators by a temperature dependent Bogoliubov transformation
\begin{align*}
  A_\lambda &= \sqrt{\bar n_\lambda + 1} \, a_\lambda + \sqrt{\bar n_\lambda} \, \adj{b}_\lambda \\
  B_\lambda &= \sqrt{\bar n_\lambda} \, \adj{a}_\lambda + \sqrt{\bar n_\lambda + 1} \, b_\lambda,
\end{align*}
with $\bar n_\lambda = \left( \exp(\beta \omega_\lambda) - 1 \right)^{-1}$ denoting the mean thermal occupation number of the (physical) oscillator mode $\lambda$.
% TODO Source! Sounds good?
An extensive but elementary calculation reveals that $\ket{0_{AB}}\bra{0_{AB}}$ with $\ket{0_{AB}} = \ket{0_A} \otimes \ket{0_B}$ defined by $A_\lambda\ket{0_{AB}} = B_\lambda\ket{0_{AB}} = 0$ satisfies \autoref{eq:nmqsd.rho_tilde}.

% TODO Sounds strange!
The doubling in degrees of freedom ensures that the reduced density matrix obtained from an initial pure state $\ket{\tilde\Psi_0} = \ket{\psi_0}\otimes\ket{0_{AB}}$ in the enlarged Hilbert space coincides with the original one lacking unphysical bath oscillators.
Expressed in these new coordinates the total Hamiltonian~\ref{eq:nmqsd.Htot_thermal} reads
\begin{align}
  \Htot = \Hsys\otimes\unit &+ \sum_\lambda \sqrt{\bar n_\lambda + 1} \, \left(\cc g_\lambda L\otimes\adj{A}_\lambda + g_\lambda \adj{L}\otimes A_\lambda \right) \nonumber \\
        \label{eq:nmqsd.Htot_thermal_shifted}
        &+ \sum_\lambda \sqrt{\bar n_\lambda} \, \left( g_\lambda \adj{L}\otimes\adj{B}_\lambda  + \cc g_\lambda L \otimes B_\lambda \right) \\
        &+ \unit \otimes \sum_\lambda \omega_\lambda \left( \adj{A}_\lambda A_\lambda - \adj{B}_\lambda B_\lambda \right). \nonumber
\end{align}
Our new Hamiltonian is identical to the zero-temperature model except for the system coupling to two separate oscillator baths instead of one;
therefore we need two independent processes $\ZZ_t$ and $\cc{W}_t$ for a stochastic version of \autoref{eq:nmqsd.Htot_thermal_shifted} in general:
\begin{align}
  \partial_t \psi_t = -\ii\Hsys\psi_t &+ L\ZZ_t\psi_t - \adj{L}\int_0^t \alpha_1(t-s) \frac{\delta \psi_t}{\delta \ZZ_s} \dd s \nonumber \\
  &+ \adj{L} \cc{W}_t \psi_t - L\int_0^t \alpha_2(t-s) \frac{\delta\psi_t}{\delta \cc{W}_s} \dd s.
  \label{eq:nmqsd.nmsse_thermal_2processes}
\end{align}
All effects of the original thermal initial state are now encoded in the correlation functions
\begin{equation*}
  \alpha_1(t) = \sum_\lambda (\bar n_\lambda + 1) \abs{g_\lambda}^2 \exp[-\ii\omega_\lambda t] \quad \mbox{and} \quad
  \alpha_2(t) = \sum_\lambda \bar n_\lambda \abs{g_\lambda}^2 \exp[\ii\omega_\lambda t]
\end{equation*}
for $\ZZ_t$ and $\cc{W}_t$ respectively.
Since both are Gaussian, their independence is equivalent to the vanishing of all their mutual correlations $\E[\ZZ_t W_s] = \E[Z_t W_s] = 0$.

% TODO FIX THE WHOLE PARAGRAPH FOR T=0 LIMITING CASE!
% * finite number of oscillators → no Bose Einstein condenstion, no problems!
As we doubled the bath degrees of freedom merely to cope with a thermal initial state it is quite natural that the zero-temperature result from \autoref{eq:nmqsd.nmsee} with a single driving process is recovered in the limit $T \to 0$\dots

The thermo field approach to our non-Markovian Schrödinger equation turns out to be especially simple in the case of self-adjoint coupling operators $\adj{L}$:
From \autoref{eq:nmqsd.nmsse_thermal_2processes} we see how both driving processes $\ZZ_t$ and $\cc{W}_t$ can be combined into a single one, which we will denote by $\ZZ_t$ again.
% TODO Too much since...
Since we took them to be mutually independent we find for our new sum process using $2\bar n_\lambda + 1 = \coth{\frac{\beta\omega_\lambda}{2}}$:
\begin{equation}
  \E[Z_t \ZZ_s] = \sum_\lambda \left(\abs{g_\lambda}^2 \, \coth{\frac{\beta \omega_\lambda}{2}} \, \cos{\omega_\lambda (t-s)} - \ii \sin{\omega_\lambda (t-s)} \right).
  \label{eq:nmqsd.combined_correlation}
\end{equation}

Consequently the finite temperature NMSSE takes the form identical to zero temperature theory.
They only differ in how the correlation function is obtained from a microscopical model or rather from a spectral density.
% TODO Too much?
It is not surprising that our combined correlation function~\ref{eq:nmqsd.combined_correlation} agrees with the result of Feynman and Vernon \cite{FeVe63_quantum_dissipative} derived in the path integral formalism for quantum Brownian motion.
But our approach is much more general since it can tackle any kind of open quantum system with linear coupling.

%%%%%%%%%%%%%%%%%%%%%%%%%%%%%%%%%%%%%%%%%%%%%%%%%%%%%%%%%%%%%%%%%%%%%%%%%%%%%%%
\subsection{unitary noise}
\label{sub:nmqsd.temperature.unitary}
% * method
% * would be better for hierarchies
% * still negative energies
% * problematic integral
% * classical noise --> alpha purely real

As shown in the last section we can treat classical thermal noise on the same footing as quantum noise under certain circumstances just by using a modified correlation function~\ref{eq:nmqsd.combined_correlation}.
It is worth noticing how the influence of thermal fluctuations modify only the real part of $\alpha$, a feature that explicitly distinguishes noisy classical perturbations \cite{FeHi10_path_integrals}.
Therefore it is quite instructive to present a different method for treating non-zero temperature within the non-Markovian quantum state diffusion.

We start off by expanding the thermal bath state in a coherent state basis \cite{WaMi08_quantum_optics}
\begin{equation*}
  \rho(\beta) = FILL IN
\end{equation*}
which is quite reminiscent of the expansion for a pure state projector that lead to our stochastic Schrödinger equation.
The corresponding pure initial states are a product involving all environmental oscillators
\begin{equation*}
  \ket{\Psi_0(\xi)} = \exp[-\frac{\abs{\vec\xi}^2}{2}] \ket{\psi_0} \bigotimes_\lambda \ket{\xi_\lambda}
\end{equation*}
where the additional prefactor is usually absorbed by using normalized coherent states.
A simple shift for the creation and annihilation operators $\adj{A}_\lambda = \adj{a}_\lambda - \cc{\xi}_\lambda$ and $A_\lambda = a_\lambda - \xi_\lambda$ respectively maps the environmental part of the initial state above onto the vacuum.
Therefore we can apply our zero-temperature derivation to the total Hamiltonian expressed in $A$ and $\adj{A}_\lambda$.
The resulting NMSSE reads
%TODO Fix apperance
\begin{equation}
  \partial_t \psi_t(\ZZ, \xi) = \left( -\ii\Hsys + L\cc\xi_t + \adj{L}\xi_t + L\ZZ_t - \adj{L}\int_0^t \alpha(t - s) \frac{\delta}{\delta \ZZ_s} \dd s \right) \psi_t(\ZZ, \xi, \cc\xi)
  \label{eq:nmqsd.nmsse_thermal_classic}
\end{equation}
with a classical driving process $\xi_t = \sum_\lambda g_\lambda \xi_\lambda \exp[-\ii \omega_\lambda t]$ and its familiar quantum counterpart $\ZZ_t$.
The former's properties are once again fixed by its correlations
\begin{equation*}
  \E\,\xi_t = 0, \quad \E\,\xi_t \xi_s =0, \quad\mbox{and}\quad \E\,\xi_t \cc{\xi}_s = 2\sum_\lambda \bar n_\lambda \abs{g_\lambda}^2 \cos{\omega_\lambda(t-s)}.
  \label{eq:nmqsd.process_properties}
\end{equation*}
% TODO Too much zero temperature...
Recovering the reduced density matrix not only requires an average over $\ZZ$ but also over all realizations of the thermal noise process $\xi_t$.
Since all thermal occupation numbers $\bar n_\lambda$ tend to zero for $T \to 0$, we obtain the zero temperature limit simply by setting $\xi_t = 0$.
This amounts to the trivial decomposition $\rho(T = 0) = \ket{\vec 0}\bra{\vec 0}$ of the zero temperature environmental state.
% TODO What is correlation? Why no functional derivative?
% TODO DISCUSSION!

%%%%%%%%%%%%%%%%%%%%%%%%%%%%%%%%%%%%%%%%%%%%%%%%%%%%%%%%%%%%%%%%%%%%%%%%%%%%%%%
\section{Dissipative Two-level System}
\label{sec:nmqsd.two_level}
% * analytic solution
% * existence of o-operator
% !!! T = 0
%%%%%%%%%%%%%%%%%%%%%%%%%%%%%%%%%%%%%%%%%%%%%%%%%%%%%%%%%%%%%%%%%%%%%%%%%%%%%%%

% TODO Introduction
\begin{equation}
  \partial_t \psi_t = -\ii\frac{\omega}{2}\sigma_z\psi_t + c \sigma_- \ZZ_t \psi_t - c\sigma_+ \int_0^t \alpha(t - s) \frac{\delta\psi_t}{\delta \ZZ_s} \dd s
  \label{eq:nmqsd.nmsse_twolevel}
\end{equation}

%%%%%%%%%%%%%%%%%%%%%%%%%%%%%%%%%%%%%%%%%%%%%%%%%%%%%%%%%%%%%%%%%%%%%%%%%%%%%%%
\subsection{O-Operator Method}
\label{sub:nmqsd.two_level.o}
%
% TODO Check for complex c, if all terms are correct

As elaborated in \autoref{sub:nmqsd.lin_nmsse.convolutionless} we can simplify the NMSSE~\ref{eq:nmqsd.nmsse_twolevel} by replacing the functional derivative with an operator $O(t, s, \ZZ)$.
We try to solve the consistency condition~\ref{eq:nmqsd.consistency_condition} by a noise-independent ansatz
\begin{equation}
  O(t, s) = c f(t, s) \sigma_-,
  \label{eq:nmqsd.o_ansatz}
\end{equation}
hence all non-Markovian feedback from the environment is now encoded in the function $f(t, s)$ to be determined.
Plugging this ansatz into the evolution equation for $O$ yields
\begin{equation}
  \partial_t c f(t, s) \sigma_- = \left[-\ii \frac{\omega}{2} \sigma_z - c^2 F(t) \sigma_+\sigma_-, c f(t, s) \sigma_-\right]
  \label{eq:nmqsd.eom_for_o}
\end{equation}
with a shorthand notation $F(t) := \int_0^t \alpha(t-s) f(t, s) \dd s$.
From its definition~\ref{eq:nmqsd.o_bar} we see that $F$ is also the prefactor for the integrated operator $\bar O(t) = c F(t) \sigma_-$.
% TODO Sounds pretty much copied!
Since the operator algebra in \autoref{eq:nmqsd.eom_for_o} closes, our ansatz solves the equation of motion for $O$ provided $f$ evolves according to
% TODO Better way to write (s < t); indentation looks strange
\begin{equation*}
  \partial_t f(t, s) = \left(\ii \omega + c^2 F(t)\right) \, f(t, s) \quad (s \le t).
\end{equation*}
Appropriate initial conditions follow trivially from \autoref{eq:nmqsd.o_initial}; they read $f(s, s) = 1$.
In the special case of an exponential bath correlation function $\alpha(t) = g \exp[-\gamma\abs{t} - \ii\Omega t]$ we can also derive a differential equation that is closed in $F$
\begin{equation*}
  \partial_t F(t) = g + (\ii (\omega - \Omega) - \gamma) F(t) + c^2 F(t)^2.
\end{equation*}
Such correlation functions play a major role in the subsequent work.

As shown in \autoref{sub:nmqsd.lin_nmsse.master}, a convolutionless NMSEE with noise independent $O$-operator can be transformed into a master equation without any approximation.
For the model under consideration it turns out to closely resemble a Lindblad-type equation except for time-dependent coefficients
\begin{equation*}
  \partial_t \rho_t = -\ii \frac{\omega}{2} [\sigma_z, \rho_t] + c^2 [\sigma_-, \rho_t \sigma_+] + c^2 F(t) [\sigma_- \rho_t, \sigma_+].
  \label{eq:nmqsd.two_level_master}
\end{equation*}


%%%%%%%%%%%%%%%%%%%%%%%%%%%%%%%%%%%%%%%%%%%%%%%%%%%%%%%%%%%%%%%%%%%%%%%%%%%%%%%
\subsection{Noise-Expansion Method}
\label{sub:nmqsd.expansion}

In this section we propose a different method for solving \autoref{eq:nmqsd.nmsse_twolevel}:
% TODO Add reference
It is based on the expansion discussed in \autoref{???}, which allows us to express the quantum trajectories $\psitZ$ in a functional Taylor series with respect to the noise process.
Due to the particular coupling structure of the model we can neglect all terms higher than linear order in $\ZZ_t$.
As further elaborated in \autoref{sec:tla.general}, our NMSSE~\ref{eq:nmqsd.nmsse_twolevel} reduces to a $\Complex$-valued integro-differential equation
\begin{equation}
  \dot\psi^+(t) = -\ii \frac{\omega}{2} \psi^+(t) - c^2 \int_0^t \alpha(t - s) \exp[\ii \frac{\omega}{2} (t - s)] \psi^+(s) \dd s,
  \label{eq:nmqsd.dotpsi_plus}
\end{equation}
which is nevertheless quite involved---even from a numerical point of view.
% TODO Bad grammar
The situation is noticeably simpler for an exponential correlation function for which we calculate an analytic solution in the appendix as well.

But even without an explicit solution for $\psi^+(t)$ we may still discover some illuminating consequences concerning the $O$-operator from the last section.
With $\psi^-(t) = \psi^-(0) \, \exp(\ii \omega t / 2)$ the full quantum trajectory reads
\begin{equation}
  \psi_t(\ZZ) = \left(\begin{array}{c}
    \psi^+(t) \\ \psi^-(t)
    \end{array}\right)
  + c \int_0^t \left(\begin{array}{c}
    0 \\ \exp[\ii \frac{\omega}{2} (t-s)] \psi^+(s)
    \end{array}\right)
  \ZZ_s \dd s.
  \label{eq:nmqsd.solution}
\end{equation}
% TODO Good idea?
This allows us to calculate the functional derivative with respect to the driving process explicitly;
for $0 \le s \le t$ we find\footnote{For $s = t$ there is no additional prefactor $\frac{1}{2}$ from integrating a $\delta$-function localized at the upper integral boundary as explained in the footnote on page~\pageref{fn:tla.boundaries}.}
\begin{equation*}
  \frac{\delta \psi_t(\ZZ)}{\delta \ZZ_s} = c \left(\begin{array}{c}
    0 \\ \exp[\ii \frac{\omega}{2}(t - s)] \psi^+(s),
    \end{array}\right)
\end{equation*}
which agrees with out ansatz~\ref{eq:nmqsd.o_ansatz} in case we choose
\begin{equation}
  f(t, s) = \frac{\psi^+(s)}{\psi^+(t)} \, \exp[-\ii \frac{\omega}{2}(s - t)].
  \label{eq:nmqsd.o_ansatz_psi}
\end{equation}
A similar structure for the $O$-operator has been obtained by Strunz \cite{St01_habil} within a Heisenberg-operator method.
% TODO Discussion, different results, Pictures!
