\chapter{Introduction}
\label{chap:introduction}
% * Classical/Quantum closed time evolution
% * necessity for open systems

% Quantum mechanics, hydrogen, larger system
% hydrogen in external field, classical environment
% open system viewpoint, simplest kind of environment: measurement
%  * exchange of energy, momentum, angular momentum, phase, charge, ...
%  * Schrödinger equation: closed system; appromation
%  * bath degrees of freedom uninteresting!
%  * best cut: system = all degrees of freedom relevant to describe experiment
%  * examples

The explanation of the discrete Hydrogen-spectrum in terms of Matrix-mechanics by Heisenberg and Wave-mechanics by Schrödinger is often perceived as the hour of birth for modern quantum theory.
Since then, the desire to understand systems with an increasing number of degrees of freedom and ever-growing complexity has been a main driving force behind the development of new theoretical ideas.
However, the overwhelming majority of quantum mechanical insights relies on more or less severe approximations, as even apparently simple models like the Helium atom have not been solved analytically.
One simplification that underlies all physical investigation is the distinction between relevant and irrelevant degrees of freedom for a certain situation:
Even the simple Hydrogen model, made up from an electron-proton pair and the mutual Coulomb attraction, ignores the surrounding electrons and protons in the same container, to give an example.
While under many circumstances the relevant degrees of freedom behave like an isolated unit, there are examples where environmental effects significantly shape the dynamical behavior of the system under consideration---these are referred to as open quantum systems.
Clearly, the cut distinguishing between relevant system and irrelevant environment is not uniquely defined, but generally, the requirement to account for a given experimental setting eliminates ambiguities to the greatest possible extent.

% Open Classical System
% * possibly only true closed system: universe
% * classical open systems
% * Brownian motion
% * colission based models
% * well understood, friction and fluctuations, but not conceptual that much harder
%  * still well defined system state
% * QM ENTAAAAANGLEMENT!!!!

Of course, the previous discussion applies not only to quantum mechanics:
A closed classical system is described by phase space coordinates $(p,q)$ and canonical equations of motion in the Hamiltonian framework.
To account for the influence of an environment, the latter are simply extended by fluctuation- and friction-forces leading to an irreversible time evolution governed by the stochastic Langevin equation \cite{GaCr85_handbook}.
A different description arises by averaging trajectories corresponding to different noise-realizations, the resulting probability density on phase space satisfies the Fokker-Planck equation.
However, both points of view are completely equivalent; there is no conceptual problem in assigning a pure state, namely a phase space point $(p(t),q(t))$, to a given trajectory at any time $t$---this simplifies matters dramatically compared to an open quantum system.

%%%%%%%%%%%%%%%%%%%%%%%%%%%%%%%%%%%%%%%%%%%%%%%%%%%%%%%%%%%%%%%%%%%%%%%%%%%%%%%
\section{Open Quantum Systems}
\label{sec:intro.open_quantum}
% * feynman-vernon
% * projection operator
% * markovian stochastic schrödinger equations
% * markov vs nonmarkov
% * Measure of non-Markovianity
% * Why SSE? --> ρ always positive -->
% * quapi
% * Connection to linear response theory Weiss Ch. 6
%%%%%%%%%%%%%%%%%%%%%%%%%%%%%%%%%%%%%%%%%%%%%%%%%%%%%%%%%%%%%%%%%%%%%%%%%%%%%%%

% general open system
%  * product Hilbert space (or one isomorphic)
%  * reduced density matrix; all answers
%  * although complete state pure; entanglement causes mixed reduced state

In order to determine the dynamics of an open quantum system from first principles, we proceed as follows:
First we embed the system into an environment such that both combined can be regarded as a closed system for all practical matters.
Mathematically, the kinematic structure is given by the product $\HHsys\otimes\HHenv$ or any Hilbert space isomorphic to it.
For such an isolated system, the time evolution is governed by the usual postulates of quantum mechanics, that is the Schrödinger or the von Neumann equation for pure or mixed initial state, respectively.
Since the environment's purpose is merely to restore a unitary time evolution, all relevant information about the system is encoded in the \emph{reduced state} $\rho(t) = \Tr_\mathrm{env} \rho_\mathrm{tot}(t)$ obtained by tracing over the environmental degrees of freedom.
Formally, this amounts to assigning the expectation value
\begin{equation}
  \qmean{A} := \Tr(\rho_\mathrm{tot}(t) \, A\otimes I_\mathrm{env}) = \Tr_\mathrm{sys}(\rho(t) \, A)
  \label{eq:intro.reduced_state}
\end{equation}
to each observable $A$ on $\HHsys$.
The second part, namely that $\qmean{\cdot}$ defines a genuine trace-class operator $\rho_t$ on $\HHsys$, follows under some additional regularity assumptions \cite{BrRo03_operator_algebras}.
Although the total system-bath state is pure at any given time, the reduced state is not necessarily a pure-state projector.
On the contrary, quantum entanglement caused by the interaction requires assigning a true mixture to the system---this makes the treatment of open quantum system so much harder compared to its classical counterpart.\\



% Two approaches: embedding into larger/axiomatic, why we choose later
% master equations
%  * trace out degrees of freedom (microscopical)
%     * only approximately, phenomonological models (redfield)
%        * valid only for some initial conditions
%        * possibly unphyical behavior due to purely perturbational treatment (Breuer or Cohen-Tanduij)

Although the reduced state's time evolution is uniquely defined by the von Neumann equation for $\rho_\mathrm{tot}(t)$ and \autoref{eq:intro.reduced_state}, a dynamical equation purely in terms of $\rho(t)$, a \emph{Master equation}, is more desirable for all practical purposes.
In general, the derivation of a closed equation in $\rho(t)$ from the unitary system-bath evolution relies on quite severe approximations.
A typical example is the \emph{Redfield}-equation \cite{BrPe2002_open_quantum}, which is based on the following physical assumptions:
First, one starts with a initial product state $\rho_\mathrm{tot}(0) = \rho(0) \otimes \rho_\mathrm{env}$.
By the weak-coupling or \emph{Born}-approximation, this product form with a fixed environmental state $\rho_\mathrm{env}$ is preserved for all times.
Furthermore, we assume that the dynamics of the environment proceed on a much smaller timescale compared to the system and any \quotes{memory}-effects are negligible small---this is the \emph{Markov}-approximation
However, because it is derived from a truncated perturbation expansion, the Redfield-master equation does not necessarily preserve all properties of a genuine density matrix.

%  * Linblad: (axiomatic)
%     * Born Approx: weak coupling, incluence of system on env. small
%     * Markov: memory
%       => semi-group; timescales
%     * quantum dynamics semi-group additionally completely positive
%        * general form: Linblad (at least for finite system)
% Markov approx well founded in quantum optics (high transition frequencies, weak coupling, short correlation time)
%     * operator algebras

A more robust approach based on similar physical assumptions, but more axiomatic in spirit has been elaborated by Linblad \cite{Li76_generators_qdsg}:
It is formulated purely in terms of propagators $\Lambda_t$ on the system's Hilbert space without making reference to a certain environment.
The Born-Markov approximation is rephrased to the condition that the family of maps $(\Lambda_t)_{t\ge 0}$ constitute a quantum dynamical semi-group \cite{AlLe87_qds}, namely $\Lambda_{s+t} = \Lambda_s\Lambda_t$.
Provided the $\Lambda_t$ are completely-positive as well, the corresponding Master equation for the reduced state necessarily takes the Linblad form
\begin{equation}
  \partial_t \rho_t = - \frac{\ii}{\hbar} [H, \rho_t] + \frac{1}{2}\sum_n \left( [L_n\,\rho_t, \adj{L}_n] + [L_n, \rho_t\,\adj{L}_n] \right).
  \label{eq:intro.linblad}
\end{equation}
Here $H = \adj{H}$ is a self-adjoint operator and $L_n$ are Linbladians describing various irreversible channels \cite{WiMi10_measurement}.

% Nowadays: Non-Markovian effects
%  * examples:
%     * solid state physics: tunneling qubit?
%     * quantum optics: decay into structued environment; cavity gedöhns, photonic band gap materials
%     * chemical- and biological systems:
%  * general problems: no general master equation
%  * Najma zwanzig projection operator --> systematic investigation
%     * approximation in practical evaluation: product initial state; although general case possible by ???
%     * unwieldy numerics due to memory effecs,

% TODO Add applications...
no universally valid and tractable non-Markovian generalization of the Lindblad Master equation is known today.
The projection formalism of Nakajima-Zwanzig provides a closed equation in the reduced state $\rho_t$ by separating relevant and irrelevant degrees of freedom in the von Neumann equation \cite{BrPe2002_open_quantum}.
However, it includes non-Markovian effects by a memory-integral such that the evolution of $\rho(t)$ depends on all reduced states at times prior to $t$.
This renders analytical or numerical solutions without further approximations virtually impossible.\\

% Alternative: Feynman Vernon
%  * direct calculation of path integrals
%  * problematic due to oscillatory effecs
%  * equivalent, but more better suited: Tanimura hierarchy

A completely different approach to the dynamics of open quantum systems is the influence-functional formalism \cite{FeVe63_quantum_dissipative,FeHi10_path_integrals}:
Besides its strength in analytical calculations


%%%%%%%%%%%%%%%%%%%%%%%%%%%%%%%%%%%%%%%%%%%%%%%%%%%%%%%%%%%%%%%%%%%%%%%%%%%%%%%%
\section{Unravellings}
\label{sec:intro.unravellings}

% Classical Mechanics: equivalance between Fokker-Planck
%  * evolution of probability density function of finite dim. cont. Markov process
% and Langevin equation: Stochastic DE, "process realizations"
% => Connection between partial differential equations and stochastic processes: Feynman Kac Formula
%  * lies at the Heart of the Path integral formulation



% Quantum mechanics: Markovian regime
%  * Stochastic Schrödinger equation for pure states
%  * such that average coincides with reduced density operator
% diffusive equation (linear and nonlinear)
% jump unravellings
% favored due to smaller size, parallelization, always positive density operator\cite{GaWi02_real_nmsse}
% Monte-Carlo: Take E(...) and evaluate for "random" trajectories; N-->\infty should work fine by law of large numbers
% favored, since they contain more information; we have a pure product state for open system!!; still have environmental degrees of freedom
%


% NMSSE
%  * based on microscopical model, not unravelling
%  * without any approximations
%  * encodes all information in bcf (similar to Dissipation-Fluctuation Theorem)
%  * good for studying transition to Markov; but also by assuming (possibly unphyical) bcf numerical investigation!!!
%  * can provide master equation
% Also fermionic environment



% Alternative Non-Markovian Description:
%  * Jump equations (Jack)
%  * Pseudo-Modes (used for spectra):
%     * for bcf used in this work, possible to replace Non-Markovian env. by damped oscillators coupled to Markovian bath
%     => usual trajectoriy treatment



%%%%%%%%%%%%%%%%%%%%%%%%%%%%%%%%%%%%%%%%%%%%%%%%%%%%%%%%%%%%%%%%%%%%%%%%%%%%%%%
\section{Established Results and Subject of this Work}
\label{sec:into.results}
% * Collission models
% * Tanimura Hierarchies
% * NMQSD (also for fermionic)
% * results
%%%%%%%%%%%%%%%%%%%%%%%%%%%%%%%%%%%%%%%%%%%%%%%%%%%%%%%%%%%%%%%%%%%%%%%%%%%%%%%

% We want to solve NMSSE!!!
%
% * Analytical solutions
% TODO ZOFE and functional expansion of O operator
%  * spectra and transfer in ...
% All based on a uncontrolled assumption of O operator

% Here we try to do it without; closely related to Tanimura
% Hierarchy linear and nonlinear


%%%%%%%%%%%%%%%%%%%%%%%%%%%%%%%%%%%%%%%%%%%%%%%%%%%%%%%%%%%%%%%%%%%%%%%%%%%%%%%
\section{Outline}
\label{sec:into.outline}
%%%%%%%%%%%%%%%%%%%%%%%%%%%%%%%%%%%%%%%%%%%%%%%%%%%%%%%%%%%%%%%%%%%%%%%%%%%%%%%

