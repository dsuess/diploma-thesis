%%%%%%%%%%%%%%%%%%%%%%%%%%%%%%%%%%%%%%%%%%%%%%%%%%%%%%%%%%%%%%%%%%%%%%%%%%%%
% Title Page
\thispagestyle{empty}
\vspace{-5em}
\begin{flushleft}
  \includegraphics[angle=0,width=60mm]{TU_Logo_SW.pdf}
  \par
\end{flushleft}
\vspace{-3em}
\begin{center}\rule{\textwidth}{0.1ex}\par\end{center}
\vspace{-4em}
\begin{center}\rule{\textwidth}{0.1ex}\par\end{center}

\vfill
\begin{center}\textbf{\Huge Hierarchy of Quantum Trajectories}\par\end{center}
\begin{center}\textbf{\Huge applied to}\par\end{center}
\begin{center}\textbf{\Huge Photosynthetic Complexes}\par\end{center}
\vfill
\begin{center}
{\large Diplomarbeit}\\
{\large zur Erlangung des wissenschaftlichen Grades}\\
{\large Diplom-Physiker}\par
\end{center}

\begin{center}vorgelegt von\par\end{center}
\begin{center}{\large Daniel Süß}\\geboren am 18.10.1987 in
Dresden\par
\end{center}
\vspace{13mm}
\begin{center}
{\large Institut für Theoretische Physik}\\
{\large der Technischen Universitaet Dresden}\\
{\large 2013}\par
\end{center}

%%%%%%%%%%%%%%%%%%%%%%%%%%%%%%%%%%%%%%%%%%%%%%%%%%%%%%%%%%%%%%%%%%%%%%%%%%%%%
% Referee Pages
\newpage
\thispagestyle{empty}
\ \vfill
\begin{tabular}{l}
Eingereicht am 23.~Oktober 2013
\end{tabular}

\vspace{1cm}

\begin{tabular}{ll}
1.~Gutachter: & Herr Prof. Dr. Walter Strunz\\
2.~Gutachter: & Herr PD Dr. Frank Großmann\\
\end{tabular}

%%%%%%%%%%%%%%%%%%%%%%%%%%%%%%%%%%%%%%%%%%%%%%%%%%%%%%%%%%%%%%%%%%%%%%%%%%%%%
% Abstract
\newpage
\thispagestyle{empty}
\selectlanguage{ngerman}
\section*{\centering{Zusammenfassung}}

Stochastische Schrödingergleichungen haben sich als effiziente Werkzeuge zur Beschreibung der Dynamik Markov'scher offener Quantensysteme etabliert.
Im Gegensatz zur numerischen Integration der Mastergleichung erlauben sie die parallele Propagation unabhängiger Realisierungen.
Eine entsprechende Erweiterung auf nicht-Markov'sche Systeme ist Gegenstand der vorliegenden Diplomarbeit.
Aufbauend auf der nicht-Markov'schen stochastischen Schrödingergleichung, die im ersten Abschnitt der Arbeit wiederholt wird, beschäftigt sich der Hauptteil mit der Herleitung einer äquivalenten Hierarchie von stochastischen Differentialgleichungen.
Die anschließende Anwendung auf Energietransfer in Lichtsammelkomplexen behandelt die Fragestellung, inwiefern quantenmechanische Effekte die Wirkungsweise von Pigment-Proteinen beeinflusst.
Außerdem bietet die vorgestellte Methode eine effektive Möglichkeit zur Berechnung von Absorptionsspektren, die wichtige Informationen über den Aufbau von chemischen Komplexen liefern.
\bigskip

\selectlanguage{english}
\section*{\centering{Abstract}}
Stochastic Schrödinger equations have been established as an efficient tool to describe the dynamics of Markovian open quantum systems.
Contrary to a numerical integration of the master equation, they admit parallel propagation of independent trajectories.
An appropriate generalization to non-Markovian systems is the subject of the present diploma thesis.
Based on the non-Markovian stochastic Schrödinger equation, which is recapitulated in the first section, an equivalent hierarchy of stochastic differential equations is derived.
The following application to energy transfer in light-harvesting systems is concerned with the question, to what extent quantum mechanical effects influence the operation of pigment-proteins.
Furthermore, the newly-devised method provides an effective approach to calculate absorption spectra, which contain crucial information on the structure of chemical complexes.

%%%%%%%%%%%%%%%%%%%%%%%%%%%%%%%%%%%%%%%%%%%%%%%%%%%%%%%%%%%%%%%%%%%%%%%%%%%%%
\tableofcontents
%%%%%%%%%%%%%%%%%%%%%%%%%%%%%%%%%%%%%%%%%%%%%%%%%%%%%%%%%%%%%%%%%%%%%%%%%%%%%
