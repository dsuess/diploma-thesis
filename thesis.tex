\documentclass[a4paper,11pt,BCOR=8mm,twoside,headsepline]{scrbook}
\author{Daniel Süß}
\title{Stochastic Dynamics of Open Quantum Systems}

% Basic input and font encoding
%\usepackage{scrhack}
\usepackage[utf8]{inputenc}
\usepackage[T1]{fontenc}
\usepackage[ngerman, english]{babel}

% math stuff
\usepackage{amsfonts, amsmath, amssymb, amsthm}

% pictures n stuff
\usepackage{graphicx}
\DeclareGraphicsExtensions{.pdf}
\usepackage{subcaption}
% labels for subfigures are capital
\renewcommand{\thesubfigure}{\Alph{subfigure}}

\usepackage{cite}
\usepackage{enumerate}

% tables
\usepackage{tabu}
\usepackage{multicol}


\usepackage{pgf}
\usepackage{tikz}
\usetikzlibrary{matrix}
\usetikzlibrary{calc}
\usetikzlibrary{arrows}
%\usetikzlibrary{external}
%\tikzexternalize[prefix=tikz/]

\usepackage{standalone}

% package to provide if-statements in new commands
\usepackage{ifthen}

% Correct spacing of abbreviations
\usepackage{xspace}

% enables referencing and generation of hyperlinks
\usepackage{hyperref}
\usepackage[all]{hypcap}

% some custom spacing
\usepackage{setspace}
\onehalfspacing
\usepackage[margin=0.8cm, font=small, labelfont=bf]{caption}

% better looking font setting
%\usepackage{microtype}

%\newcommand{\includetikz}[1]{%
%    \tikzsetnextfilename{#1}%
%    \input{img/#1.tikz}%
%}

%%%%%%%%%%%%%%%%%%%%%%%%%%%%%%%%%%%%%%%%%%%%%%%%%%%%%%%%%%%%%%%%%%%%%%%%%%%%%%%
% Custom commands

% Text commands/symbols
\newcommand{\idef}[1]{\emph{#1}}       % (i)nline (def)inition
\newcommand{\quotes}[1]{``#1''}
\newcommand{\nth}[1]{\textsuperscript{#1}\,}
\renewcommand{\th}{\nth{th}}
\newcommand{\cm}[1]{{\color{red}#1}}   % (c)orrection (m)ark
\newcommand{\NMSSE}{\textsc{NMSSE}\xspace}
\newcommand{\HEOM}{\textsc{HEOM}\xspace}
\newcommand{\ucm}{\mathrm{cm}}

% Math Macros
\newcommand{\cc}[1]{{#1}^*}
\newcommand{\abs}[1]{\vert #1 \vert}
\newcommand{\norm}[1]{\left\Vert #1 \right\Vert}
\newcommand{\op}[1]{#1}
\newcommand{\adj}[1]{\op{#1}^\dagger}
\newcommand{\dual}[1]{{#1}^*}

\newcommand{\E}[1][\empty]{%
  \ifthenelse{\equal{#1}{\empty}}
    {\mathbb{E}}
    {\mathbb{E}\left( #1 \right)}
}
\newcommand{\qmean}[1]{\langle #1 \rangle}

\renewcommand{\exp}[1][\empty]{%
  \ifthenelse{\equal{#1}{\empty}}
    {\mathrm{exp}}
    {\mathrm{e}^{#1}}
}
\newcommand{\Texp}[1][\empty]{%
  \ifthenelse{\equal{#1}{\empty}}
    {\operatorname{T}_+\mathrm{exp}}
    {\operatorname{T}_+\mathrm{e}^{#1}}
}
\newcommand{\varf}[1][\empty]{%
  \ifthenelse{\equal{#1}{\empty}}
    {\mathbb{\mathcal{C}}}
    {\mathcal{C}\left( #1 \right)}
}
\newcommand{\intl}[3]{\int_{#1}^{#2}\mathrm{d}#3\,}

\let\rvec\vec
\renewcommand{\vec}[1]{\boldsymbol{#1}}
\newcommand{\bra}[1]{\langle #1 |}
\newcommand{\ket}[1]{| #1 \rangle}
%\newcommand{\braket}[2]{\left\langle #1 \middle| #2 \right\rangle}
\newcommand{\braket}[2]{\langle #1 | #2 \rangle}
\renewcommand{\sp}[2]{\left\langle #1 , #2 \right\rangle}
\newcommand{\dualp}[2]{\left( #1 , #2 \right)}

\newcommand{\twovec}[2]{%
  \begin{pmatrix}
    #1 \\ #2
  \end{pmatrix}
}

% Math Symbols
\newcommand{\ii}{\mathrm{i}}
\newcommand{\dd}{\, \mathrm{d}}
\newcommand{\intdd}{\int\mathrm{d}}
%\newcommand{\Tr}[1]{\operatorname{Tr}_\mathrm{#1}}
\newcommand{\Tr}{\operatorname{Tr}}
\newcommand{\Res}{\operatorname{Res}}
\newcommand{\mudz}{\, \mu(\mathrm{d} z)}

\newcommand{\ZZ}{\cc{Z}}               % Our noise process
\newcommand{\tildeZZ}{\cc{\tilde Z}}
\newcommand{\adjZZ}{\mathcal{D}}         % and it's adjoint

\newcommand{\kk}{\vec{k}}
\newcommand{\zz}{\vec{z}}
\newcommand{\ww}{\vec{w}}
\newcommand{\qq}{\vec{q}}
\newcommand{\QQ}{\vec{Q}}

\newcommand{\psit}[1][\empty]{%
  \ifthenelse{\equal{#1}{\empty}}
    {\psi_t}
    {\psi_t^{(#1)}}
}
\newcommand{\psitz}[1][\empty]{%
  \ifthenelse{\equal{#1}{\empty}}
    {\psi_t(\cc\zz)}
    {\psi_t^{(#1)(\cc\zz)}}
}
\newcommand{\psitZ}[1][\empty]{%
  \ifthenelse{\equal{#1}{\empty}}
    {\psi_t(\ZZ)}
    {\psi_t^{(#1)}(\ZZ)}
}
\newcommand{\psitphi}{\psi_t(\cc{\vec\phi}_t(\cc\zz))}
\newcommand{\psio}[1]{\psi^{(#1)}}

\newcommand{\phitla}{\phi_{t, \lambda}}
\newcommand{\ccphitla}{\cc\phi_{t, \lambda}}

\newcommand{\ee}{\vec{e}}
\newcommand{\unit}{\op{\mathrm{I}}}    % Idendity operator/matrix
\newcommand{\PM}{\mathbb{P}}           % Probability measure
\newcommand{\ft}[1]{\Hat{#1}}
\newcommand{\ift}[1]{\check{#1}}
\newcommand{\CHF}{\mathcal{F}}         % Characteristic functional
\renewcommand{\propto}{\sim}

% Operators
\newcommand{\Htot}{H_\mathrm{tot}}
\newcommand{\Hsys}{H}
\newcommand{\Henv}{H_\mathrm{env}}
\newcommand{\Hint}{H_\mathrm{int}}
\newcommand{\opH}[1]{H_\mathrm{#1}}
\newcommand{\opT}[1]{T_\mathrm{#1}}
\newcommand{\opV}[1]{V_\mathrm{#1}}
\newcommand{\opU}[1]{U_\mathrm{#1}}

% Math Sets
\newcommand{\Complex}{\mathbb{C}}
\newcommand{\Reals}{\mathbb{R}}
\newcommand{\Integers}{\mathbb{Z}}
\newcommand{\SchwartzS}{\mathcal{S}}
\newcommand{\SchwartzSC}{\SchwartzS_\Complex}
\newcommand{\HH}{\mathcal{H}}
\newcommand{\HHenv}{\HH_\mathrm{env}}
\newcommand{\HHsys}{\HH_\mathrm{sys}}

% Environments
\newtheorem{thm}{Theorem}
\newtheorem{lem}[thm]{Lemma}
\theoremstyle{remark}
\newtheorem*{rem}{Remark}

% Make sure autoref print (.) for equations
\let\oldtheequation\theequation
\makeatletter
  \def\tagform@#1{\maketag@@@{\ignorespaces#1\unskip\@@italiccorr}}
  \renewcommand{\theequation}{(\oldtheequation)}
\makeatother

% Autoref commands
% \Autoref is for the beginning of the sentence
\let\orgautoref\autoref
\providecommand{\Autoref}
        {\def\equationautorefname{Equation}%
         \def\figureautorefname{Figure}%
         \def\subfigureautorefname{Figure}%
         \def\chapterautorefname{Chapter}
         \def\sectionautorefname{Section}%
         \def\subsectionautorefname{Section}%
         \def\subsubsectionautorefname{Section}%
         \def\Itemautorefname{Item}%
         \def\tableautorefname{Table}%
         \orgautoref}

% \Autorefs is plural for the beginning of the sentence
\providecommand{\Autorefs}
        {\def\equationautorefname{Equations}%
         \def\figureautorefname{Figures}%
         \def\subfigureautorefname{Figures}%
         \def\chapterautorefname{Chapters}
         \def\sectionautorefname{Sections}%
         \def\subsectionautorefname{Sections}%
         \def\subsubsectionautorefname{Sections}%
         \def\Itemautorefname{Items}%
         \def\tableautorefname{Tables}%
         \orgautoref}

% \autoref is used inside a sentence
% (this is a renew of the standard)
\renewcommand{\autoref}
        {\def\equationautorefname{Eq.}%
         \def\figureautorefname{Fig.}%
         \def\subfigureautorefname{Fig.}%
         \def\chapterautorefname{Chap.}
         \def\sectionautorefname{Sect.}%
         \def\subsectionautorefname{Sect.}%
         \def\subsubsectionautorefname{Sect.}%
         \def\Itemautorefname{item}%
         \def\tableautorefname{Table}%
         \orgautoref}

% \autorefs is plural for inside a sentence
\providecommand{\autorefs}
        {\def\equationautorefname{Eqs.}%
         \def\figureautorefname{Figs.}%
         \def\subfigureautorefname{Figs.}%
         \def\chapterautorefname{Chaps.}
         \def\sectionautorefname{Sects.}%
         \def\subsectionautorefname{Sects.}%
         \def\subsubsectionautorefname{Sects.}%
         \def\Itemautorefname{items}%
         \def\tableautorefname{Tables}%
         \orgautoref}


% centered table header
 \newcommand{\theader}[1]{\multicolumn{1}{c}{#1}}
\clubpenalty10000
\widowpenalty10000
\displaywidowpenalty=10000
%%%%%%%%%%%%%%%%%%%%%%%%%%%%%%%%%%%%%%%%%%%%%%%%%%%%%%%%%%%%%%%%%%%%%%%%%%%%%%%
%%%%%%%%%%%%%%%%%%%%%%%%%%%%%%%%%%%%%%%%%%%%%%%%%%%%%%%%%%%%%%%%%%%%%%%%%%%%%%%
%\includeonly{introduction}
%\includeonly{nmqsd,appendix}
%\includeonly{numerics}
%\includeonly{application}
%\includeonly{conclusions}

\begin{document}
\frontmatter
%%%%%%%%%%%%%%%%%%%%%%%%%%%%%%%%%%%%%%%%%%%%%%%%%%%%%%%%%%%%%%%%%%%%%%%%%%%%
% Title Page
\thispagestyle{empty}
\vspace{-5em}
\begin{flushleft}
  \includegraphics[angle=0,width=60mm]{TU_Logo_SW.pdf}
  \par
\end{flushleft}
\vspace{-3em}
\begin{center}\rule{\textwidth}{0.1ex}\par\end{center}
\vspace{-4em}
\begin{center}\rule{\textwidth}{0.1ex}\par\end{center}

\vfill
\begin{center}\textbf{\Huge Stochastical Dynamics of}\par\end{center}
\begin{center}\textbf{\Huge Open Quantum Systems}\par\end{center}
\vfill
\begin{center}
{\large Diplomarbeit}\\
{\large zur Erlangung des wissenschaftlichen Grades}\\
{\large Diplom-Physiker}\par
\end{center}

\begin{center}vorgelegt von\par\end{center}
\begin{center}{\large Daniel Süß}\\geboren am 18.10.1987 in
Dresden\par
\end{center}
\vspace{13mm}
\begin{center}
{\large Institut für Theoretische Physik}\\
{\large der Technischen Universitaet Dresden}\\
{\large 2013}\par
\end{center}

%%%%%%%%%%%%%%%%%%%%%%%%%%%%%%%%%%%%%%%%%%%%%%%%%%%%%%%%%%%%%%%%%%%%%%%%%%%%%
% Referee Pages
\newpage
\thispagestyle{empty}
\ \vfill
\begin{tabular}{l}
%Eingereicht am ???\\
\end{tabular}

\vspace{1cm}

\begin{tabular}{ll}
1.~Gutachter: & Herr Prof. Dr. Walter T. Strunz\\
%2.~Gutachter: & Herr Prof. Dr. ???\\
\end{tabular}

%%%%%%%%%%%%%%%%%%%%%%%%%%%%%%%%%%%%%%%%%%%%%%%%%%%%%%%%%%%%%%%%%%%%%%%%%%%%%
% Abstract
\newpage
\thispagestyle{empty}
\section*{\centering{Zusammenfassung}}
\section*{\centering{Abstract}}
%%%%%%%%%%%%%%%%%%%%%%%%%%%%%%%%%%%%%%%%%%%%%%%%%%%%%%%%%%%%%%%%%%%%%%%%%%%%%
\tableofcontents
%%%%%%%%%%%%%%%%%%%%%%%%%%%%%%%%%%%%%%%%%%%%%%%%%%%%%%%%%%%%%%%%%%%%%%%%%%%%%

\mainmatter
\newpage

\chapter{Introduction}
\label{chap:introduction}
% * Classical/Quantum closed time evolution
% * necessity for open systems

% Quantum mechanics, hydrogen, larger system
% hydrogen in external field, classical environment
% open system viewpoint, simplest kind of environment: measurement
%  * exchange of energy, momentum, angular momentum, phase, charge, ...
%  * Schrödinger equation: closed system; appromation
%  * bath degrees of freedom uninteresting!
%  * best cut: system = all degrees of freedom relevant to describe experiment
%  * examples

The explanation of the discrete Hydrogen-spectrum in terms of Matrix-mechanics by Heisenberg and Wave-mechanics by Schrödinger is often perceived as the hour of birth for modern quantum theory.
Since then, the desire to understand systems with an increasing number of degrees of freedom and ever-growing complexity has been a main driving force behind the development of new theoretical ideas.
However, the overwhelming majority of quantum mechanical insights relies on more or less severe approximations, as even apparently simple models like the Helium atom have not been solved analytically.
One simplification that underlies all physical investigation is the distinction between relevant and irrelevant degrees of freedom for a certain situation:
Even the simple Hydrogen model, made up from an electron-proton pair and the mutual Coulomb attraction, ignores the surrounding electrons and protons in the same container, to give an example.
While under many circumstances the relevant degrees of freedom behave like an isolated unit, there are examples where environmental effects significantly shape the dynamical behavior of the system under consideration---these are referred to as open quantum systems.
Clearly, the cut distinguishing between relevant system and irrelevant environment is not uniquely defined, but generally, the requirement to account for a given experimental setting eliminates ambiguities to the greatest possible extent.

% Open Classical System
% * possibly only true closed system: universe
% * classical open systems
% * Brownian motion
% * colission based models
% * well understood, friction and fluctuations, but not conceptual that much harder
%  * still well defined system state
% * QM ENTAAAAANGLEMENT!!!!

Of course, the previous discussion applies not only to quantum mechanics:
A closed classical system is described by phase space coordinates $(p,q)$ and canonical equations of motion in the Hamiltonian framework.
To account for the influence of an environment, the latter are simply extended by fluctuation- and friction-forces leading to an irreversible time evolution governed by the stochastic Langevin equation \cite{GaCr85_handbook}.
A different description arises by averaging trajectories corresponding to different noise-realizations, the resulting probability density on phase space satisfies the Fokker-Planck equation.
However, both points of view are completely equivalent; there is no conceptual problem in assigning a pure state, namely a phase space point $(p(t),q(t))$, to a given trajectory at any time $t$---this simplifies matters dramatically compared to an open quantum system.

%%%%%%%%%%%%%%%%%%%%%%%%%%%%%%%%%%%%%%%%%%%%%%%%%%%%%%%%%%%%%%%%%%%%%%%%%%%%%%%
\section{Open Quantum Systems}
\label{sec:intro.open_quantum}
% * feynman-vernon
% * projection operator
% * markovian stochastic schrödinger equations
% * markov vs nonmarkov
% * Measure of non-Markovianity
% * Why SSE? --> ρ always positive -->
% * quapi
% * Connection to linear response theory Weiss Ch. 6
%%%%%%%%%%%%%%%%%%%%%%%%%%%%%%%%%%%%%%%%%%%%%%%%%%%%%%%%%%%%%%%%%%%%%%%%%%%%%%%

% general open system
%  * product Hilbert space (or one isomorphic)
%  * reduced density matrix; all answers
%  * although complete state pure; entanglement causes mixed reduced state

In order to determine the dynamics of an open quantum system from first principles, we proceed as follows:
First we embed the system into an environment such that both combined can be regarded as a closed system for all practical matters.
Mathematically, the kinematic structure is given by the product $\HHsys\otimes\HHenv$ or any Hilbert space isomorphic to it.
For such an isolated system, the time evolution is governed by the usual postulates of quantum mechanics, that is the Schrödinger or the von Neumann equation for pure or mixed initial state, respectively.
Since the environment's purpose is merely to restore a unitary time evolution, all relevant information about the system is encoded in the \emph{reduced state} $\rho(t) = \Tr_\mathrm{env} \rho_\mathrm{tot}(t)$ obtained by tracing over the environmental degrees of freedom.
Formally, this amounts to assigning the expectation value
\begin{equation}
  \qmean{A} := \Tr(\rho_\mathrm{tot}(t) \, A\otimes I_\mathrm{env}) = \Tr_\mathrm{sys}(\rho(t) \, A)
  \label{eq:intro.reduced_state}
\end{equation}
to each observable $A$ on $\HHsys$.
The second part, namely that $\qmean{\cdot}$ defines a genuine trace-class operator $\rho_t$ on $\HHsys$, follows under some additional regularity assumptions \cite{BrRo03_operator_algebras}.
Although the total system-bath state is pure at any given time, the reduced state is not necessarily a pure-state projector.
On the contrary, quantum entanglement caused by the interaction requires assigning a true mixture to the system---this makes the treatment of open quantum system so much harder compared to its classical counterpart.\\



% Two approaches: embedding into larger/axiomatic, why we choose later
% master equations
%  * trace out degrees of freedom (microscopical)
%     * only approximately, phenomonological models (redfield)
%        * valid only for some initial conditions
%        * possibly unphyical behavior due to purely perturbational treatment (Breuer or Cohen-Tanduij)

Although the reduced state's time evolution is uniquely defined by the von Neumann equation for $\rho_\mathrm{tot}(t)$ and \autoref{eq:intro.reduced_state}, a dynamical equation purely in terms of $\rho(t)$, a \emph{Master equation}, is more desirable for all practical purposes.
In general, the derivation of a closed equation in $\rho(t)$ from the unitary system-bath evolution relies on quite severe approximations.
A typical example is the \emph{Redfield}-equation \cite{BrPe2002_open_quantum}, which is based on the following physical assumptions:
First, one starts with a initial product state $\rho_\mathrm{tot}(0) = \rho(0) \otimes \rho_\mathrm{env}$.
By the weak-coupling or \emph{Born}-approximation, this product form with a fixed environmental state $\rho_\mathrm{env}$ is preserved for all times.
Furthermore, we assume that the dynamics of the environment proceed on a much smaller timescale compared to the system and any \quotes{memory}-effects are negligible small---this is the \emph{Markov}-approximation
However, because it is derived from a truncated perturbation expansion, the Redfield-master equation does not necessarily preserve all properties of a genuine density matrix.

%  * Linblad: (axiomatic)
%     * Born Approx: weak coupling, incluence of system on env. small
%     * Markov: memory
%       => semi-group; timescales
%     * quantum dynamics semi-group additionally completely positive
%        * general form: Linblad (at least for finite system)
% Markov approx well founded in quantum optics (high transition frequencies, weak coupling, short correlation time)
%     * operator algebras

A more robust approach based on similar physical assumptions, but more axiomatic in spirit has been elaborated by Linblad \cite{Li76_generators_qdsg}:
It is formulated purely in terms of propagators $\Lambda_t$ on the system's Hilbert space without making reference to a certain environment.
The Born-Markov approximation is rephrased to the condition that the family of maps $(\Lambda_t)_{t\ge 0}$ constitute a quantum dynamical semi-group \cite{AlLe87_qds}, namely $\Lambda_{s+t} = \Lambda_s\Lambda_t$.
Provided the $\Lambda_t$ are completely-positive as well, the corresponding Master equation for the reduced state necessarily takes the Linblad form
\begin{equation}
  \partial_t \rho_t = - \frac{\ii}{\hbar} [H, \rho_t] + \frac{1}{2}\sum_n \left( [L_n\,\rho_t, \adj{L}_n] + [L_n, \rho_t\,\adj{L}_n] \right).
  \label{eq:intro.linblad}
\end{equation}
Here $H = \adj{H}$ is a self-adjoint operator and $L_n$ are Linbladians describing various irreversible channels \cite{WiMi10_measurement}.

% Nowadays: Non-Markovian effects
%  * examples:
%     * solid state physics: tunneling qubit?
%     * quantum optics: decay into structued environment; cavity gedöhns, photonic band gap materials
%     * chemical- and biological systems:
%  * general problems: no general master equation
%  * Najma zwanzig projection operator --> systematic investigation
%     * approximation in practical evaluation: product initial state; although general case possible by ???
%     * unwieldy numerics due to memory effecs,

% TODO Add applications...
no universally valid and tractable non-Markovian generalization of the Lindblad Master equation is known today.
The projection formalism of Nakajima-Zwanzig provides a closed equation in the reduced state $\rho_t$ by separating relevant and irrelevant degrees of freedom in the von Neumann equation \cite{BrPe2002_open_quantum}.
However, it includes non-Markovian effects by a memory-integral such that the evolution of $\rho(t)$ depends on all reduced states at times prior to $t$.
This renders analytical or numerical solutions without further approximations virtually impossible.\\

% Alternative: Feynman Vernon
%  * direct calculation of path integrals
%  * problematic due to oscillatory effecs
%  * equivalent, but more better suited: Tanimura hierarchy

A completely different approach to the dynamics of open quantum systems is the influence-functional formalism \cite{FeVe63_quantum_dissipative,FeHi10_path_integrals}:
Besides its strength in analytical calculations


%%%%%%%%%%%%%%%%%%%%%%%%%%%%%%%%%%%%%%%%%%%%%%%%%%%%%%%%%%%%%%%%%%%%%%%%%%%%%%%%
\section{Unravellings}
\label{sec:intro.unravellings}

% Classical Mechanics: equivalance between Fokker-Planck
%  * evolution of probability density function of finite dim. cont. Markov process
% and Langevin equation: Stochastic DE, "process realizations"
% => Connection between partial differential equations and stochastic processes: Feynman Kac Formula
%  * lies at the Heart of the Path integral formulation



% Quantum mechanics: Markovian regime
%  * Stochastic Schrödinger equation for pure states
%  * such that average coincides with reduced density operator
% diffusive equation (linear and nonlinear)
% jump unravellings
% favored due to smaller size, parallelization, always positive density operator\cite{GaWi02_real_nmsse}
% Monte-Carlo: Take E(...) and evaluate for "random" trajectories; N-->\infty should work fine by law of large numbers
% favored, since they contain more information; we have a pure product state for open system!!; still have environmental degrees of freedom
%


% NMSSE
%  * based on microscopical model, not unravelling
%  * without any approximations
%  * encodes all information in bcf (similar to Dissipation-Fluctuation Theorem)
%  * good for studying transition to Markov; but also by assuming (possibly unphyical) bcf numerical investigation!!!
%  * can provide master equation
% Also fermionic environment



% Alternative Non-Markovian Description:
%  * Jump equations (Jack)
%  * Pseudo-Modes (used for spectra):
%     * for bcf used in this work, possible to replace Non-Markovian env. by damped oscillators coupled to Markovian bath
%     => usual trajectoriy treatment



%%%%%%%%%%%%%%%%%%%%%%%%%%%%%%%%%%%%%%%%%%%%%%%%%%%%%%%%%%%%%%%%%%%%%%%%%%%%%%%
\section{Established Results and Subject of this Work}
\label{sec:into.results}
% * Collission models
% * Tanimura Hierarchies
% * NMQSD (also for fermionic)
% * results
%%%%%%%%%%%%%%%%%%%%%%%%%%%%%%%%%%%%%%%%%%%%%%%%%%%%%%%%%%%%%%%%%%%%%%%%%%%%%%%

% We want to solve NMSSE!!!
%
% * Analytical solutions
% TODO ZOFE and functional expansion of O operator
%  * spectra and transfer in ...
% All based on a uncontrolled assumption of O operator

% Here we try to do it without; closely related to Tanimura
% Hierarchy linear and nonlinear


%%%%%%%%%%%%%%%%%%%%%%%%%%%%%%%%%%%%%%%%%%%%%%%%%%%%%%%%%%%%%%%%%%%%%%%%%%%%%%%
\section{Outline}
\label{sec:into.outline}
%%%%%%%%%%%%%%%%%%%%%%%%%%%%%%%%%%%%%%%%%%%%%%%%%%%%%%%%%%%%%%%%%%%%%%%%%%%%%%%


\chapter{Non-Markovian Quantum State Diffusion}
\label{chap:nmqsd}
% * lin/nonlin Markovian SSE
% * little History
% * alternatives standard (projection?), pseudomodes
% * relevant/irre
%
% FIXME Citations
%
% TODO CHECK TENSES THOROUGHLY
% TODO CHECK THAT THERE IS \cc\zz EVERYWHERE
% TODO ALL IMAGES WITH SUBFIG-CAPTION!!!

The description of open quantum systems in terms of diffusive stochastic differential equations has a long tradition \cite{}.
At first, seen merely as a tool to unravel a given master equation of Lindblad type, it was realized later that they posses a strong microscopical foundation in terms of continuous measurements or memoryless quantum environments \cite{}.\\



% ✔ same applies to nmqsd unravelling: \cite{St96_lin_nmqsd}; microscopical theory: ...
%     => later is described in first section
% ✔ on its basis we derive a linear NMSSE, completely equivalent to micro. model, but numerical inferior
% * key point in hierachy later

Its non-Markovian generalization, the non-Markovian quantum state diffusion (NMQSD), took a quite similar path, which we roughly follow in this chapter:
Although first discovered as an unravelling for the Feynman-Vernon influence functional in terms of stochastic propagators \cite{St96_lin_nmqsd}, the corresponding non-Markovian stochastic Schrödinger equation (NMSSE) was derived based upon the standard open-system-model, which we \cm{recall} in \autoref{sec:nmqsd.model}.
Following the lines of Diósi, Strunz and Gisin \cite{DiSt97_nmsse,DiGiSt98_nmqsd,StDiGi99_nmq_traj} we derive both a linear and numerically superior non-linear version of the NMSSE in \autoref{sec:nmqsd.lin_nmsse} and~\ref{sec:nmqsd.nonlin_nmsse} respectively.

% * interpreation no so clear as markovian case
% * finite temperature, since derivation depends on vacuum initial bath state
%
% * study exemplary system, propose direct solution for T=0

\Autoref{sec:nmqsd.interpretation} is concerned with the question if our NMSSE has a physical interpretation or is just merely a computational tool.
%FIXME
\cm{Anschließend} we drop the requirement of zero initial temperature used in the \cm{vorherig} sections.
This chapter is \cm{abgeschlossen} by the treatment of an analytically soluble two-level system at zero temperature.\\

%FIXME Reference ok?
Most of the material covered can be found in reference \cite{St01_habil}, which we follow loosely.


%%%%%%%%%%%%%%%%%%%%%%%%%%%%%%%%%%%%%%%%%%%%%%%%%%%%%%%%%%%%%%%%%%%%%%%%%%%%%%%
\section{The Microscopical Model}
\label{sec:nmqsd.model}
% ✔ standard model (why oscillators, why linear coupling?)
% ✔ reservoir/environment
% ✔ initial states
%
% TODO Physical examples for such a model
% TODO Why harmonic oscillators for bosonic bath
% TODO Picture
%
%%%%%%%%%%%%%%%%%%%%%%%%%%%%%%%%%%%%%%%%%%%%%%%%%%%%%%%%%%%%%%%%%%%%%%%%%%%%%%%

It is the foremost goal of this work to obtain a dynamical equation for an open quantum system.
Nevertheless we introduce a full model of system and its environment first, the bosonic and non-relativistic standard model of an open quantum system extensively studied for example in the book of Weiss \cite{We99_dissipative_systems}.
There are three reasons for such a microscopical approach:
On one hand this serves the purpose to better understand the physical origin of macroscopical properties used to characterize the bath later on.
But more important, starting with a closed quantum system is the only strategy allowing us to derive the NMSSE from first principles, namely the Schrödinger equation.
%FIXME
A last argument in favor of the microscopic approach, PRODUCT STATES, ENTANGLEMENT, etc. ---we will not dwell on this any further.\\

As a starting point we consider an environment consisting of a finite number $N$ of uncoupled harmonic oscillators\footnote{%
  We use \quotes{environment}, \quotes{reservoir} and \quotes{bath} interchangeably, altough the later two suggest a large size compared to the system.
}.
A Generalization to an infinite number can be carried out formally along the same lines, replacing sums by infinite series or even integrals; a different approach within our framework is presented later.
The dynamics of both system and environment are then described by a unitary time evolution with the Hamiltonian
\begin{equation}
  \Htot = \Hsys \otimes \unit  +  \unit \otimes \Henv  +  \Hint,
  \label{eq:nmqsd.Htot}
\end{equation}
where $\Hsys$ and $\Henv$ are the free Hamiltonians of the system and the bath respectively.\footnote{%
  For some models like the damped harmonic oscillator \cite{CaLe83_diss_system} an additional renormalization term arises from the interaction.
  Nevertheless such a contribution is best attributed to $\Hsys$ since it only acts on the system's Hilbert space.
}
The latter is a sum over independent harmonic oscillators $\Henv = \sum_\lambda \omega_\lambda \adj{a}_\lambda a_\lambda$ expressed in bosonic ladder operators $a_\lambda$ and $\adj{a}_\lambda$ of the $\lambda$\th mode with frequency $\omega_\lambda$.
Treating a finite number of independent reservoirs poses no further difficulties and therefore is not elaborated in this section.

For the interaction between environment and system we confine ourselves to the case of linear coupling
\begin{equation}
  \Hint = \sum_\lambda \cc{g}_\lambda \, L \otimes \adj{a}_\lambda + g_\lambda \, \adj{L} \otimes a_\lambda.
  \label{eq:nmqsd.Hint}
\end{equation}
Here $L$ denotes the coupling operator in the system's Hilbert space and $g_\lambda \in \Complex$ the coupling strength of the $\lambda$\th mode.
% FIXME More details, taylor expansion?
Since in typical examples the coupling of an individual bath mode scales inversely with the environment size \cite{We99_dissipative_systems}, the linear coupling in~\ref{eq:nmqsd.Hint} seems reasonable for macroscopic large environments.
%FIXME stop repetition in "environment"
But our framework also incorporates small environments---even to the extreme of a single harmonic oscillator---with strong coupling as well.
For such cases the linearity needs to be imposed as another assumption of the model.

\begin{figure}
  \centering
  \documentclass{standalone}
\usepackage{amsmath}
\usepackage{tikz}
\usetikzlibrary{matrix}
\usetikzlibrary{calc}
\begin{document}
\begin{tikzpicture}[scale=.8]
  \node[align=center] at (-2.4, .7) (sys) {\footnotesize $H_\mathrm{sys}$; $|\psi_0\rangle$};
  \draw[dashed] (-2.4,.7) ellipse (1.5 and .7);
  \node[align=center,] at (2, -.8) (env) {\footnotesize $H_\mathrm{env}=\sum_\lambda a_\lambda^\dagger a_\lambda$;  $|\boldsymbol{0}\rangle$};

  \draw[thick] (0, 0) ellipse (5 and 2);

  \draw[<->, line width=2, dashed] (sys) -- (env);

  \node[align=right] at (2, .3) {\footnotesize $H_\mathrm{int} = \sum_\lambda g_\lambda^* L \otimes a_\lambda^\dagger + \mathrm{c.c.}$};

\end{tikzpicture}
\end{document}

  \caption{%
    Standard model of an open system immersed into a bosonic bath at zero temperature---this amounts to an initial product state $\ket{\psi_0}\otimes\ket{\vec 0}$.
    The environmental oscillators with frequencies $\omega_\lambda$ are described in terms of ladder operators $a_\lambda$ and $\adj{a}_\lambda$.
    A coupling operator $L$ mediates the influence of the environment.
  }
  \label{fig:nmqsd.open_system}
\end{figure}

Beside the Hamiltonian another important influence on the system's subdynamics is the initial state, specifically the initial entanglement between system and bath.
Throughout this work we only consider product initial conditions, where the bath is in the vacuum state with respect to all $a_\lambda$
\begin{equation}
  \ket{\Psi_0} = \ket{\psi_0} \bigotimes\limits_\lambda \ket{0_\lambda}.
  \label{eq:nmqsd.initial_conditions}
\end{equation}
Such a choice is not as restrictive as it seems on first glance: In \autoref{sec:nmqsd.temperature} we show how a thermal bath state can be mapped to~\ref{eq:nmqsd.initial_conditions}.
% FIXME cite Richard
However whether the NMSSE is applicable to initially entangled states is a question of current research.\\



To absorb the free dynamics of the environment in time dependent creation and annihilation operators, we switch to the interaction picture with respect to $\Henv$.
Since the bath operators only obtain an additional phase $\exp[\pm \ii \omega_\lambda t]$, the transformed Hamiltonian from \autoref{eq:nmqsd.Htot} reads\footnote{%
  We refrain from introducing another label to distinguish between time-evolution pictures---in what follows we always work in the interaction picture.
}
\begin{equation}
  \Htot(t) = \Hsys \otimes \unit  +  \sum_\lambda \left( \cc{g}_\lambda \exp[\ii \omega_\lambda t] \, L \otimes \adj{a}_\lambda + g_\lambda \exp[-\ii \omega t] \, \adj{L} \otimes a_\lambda \right).
  \label{eq:nmqsd.Htot}
\end{equation}
Our choice of unentangled initial conditions with a vacuum bath state ensures that the reduced density operator remains unaffected under the change of time-evolution picture.

It is instructive to rewrite the last equation using the operator valued force
\begin{equation}
  B(t)=\sum_\lambda g_\lambda a_\lambda \exp[-\ii\omega_\lambda t].
  \label{eq:nmqsd.force_operator}
\end{equation}
The total Hamiltonian then reads $\Htot(t) = \Hsys \otimes \unit  +  L \otimes \adj{B(t)}  +  \adj{L} \otimes B(t)$.
Already from this equation it can be seen, that the complete action of the environment on the system is encoded in the operators $B(t)$.
An important---and within our model the only---characteristic of them is the correlation function $\alpha(t-s) = \big\langle  (B(t) + \adj{B(t)})(B(s) + \adj{B(s)}) \big\rangle_\rho$ for an arbitrary bath state $\rho$.
For a thermal state at temperature $T$, the correlation function can be calculated analytically \cite{FeHi10_path_integrals}
\begin{equation}
  \alpha_T(t - s) = \sum_\lambda  \abs{g_\lambda}^2  \left( \operatorname{coth} \frac{\omega_\lambda}{2T} \, \cos \omega_\lambda (t-s)  -  \ii \sin \omega_\lambda(t-s) \right).
  \label{eq:nmqsd.thermal_correlation_function}
\end{equation}
Introducing the spectral density $J(\omega) = \sum_\lambda \abs{g_\lambda}^2 \delta(\omega - \omega_\lambda)$ and taking the limit $T \to 0$, the above equation can be rephrased as
\begin{equation}
  \alpha(t - s) = \qmean{B(t)\adj{B(s)}}_0 = \int_0^\infty J(\omega) \exp[-\ii\omega (t-s)] \dd \omega.
  \label{eq:nmqsd.correlation_function}
\end{equation}
In other words, the correlation function is simply given as one-sided Fourier transform of the spectral density.
% TODO Check this!
Of course this connection between response function and power spectrum---and its general form for $T\neq0$---is well known as fluctuation-dissipation relation.
Since a genuine physical spectral density is real, we require admissible correlation function to be hermitian $\alpha(-t) = \cc{\alpha(t)}$.\\

% TODO Types of correlation function, incommensurate frequencies, generalization --> contiuum-approximation, exp. decay, Markov, inclussion of negative frequencies,


%%%%%%%%%%%%%%%%%%%%%%%%%%%%%%%%%%%%%%%%%%%%%%%%%%%%%%%%%%%%%%%%%%%%%%%%%%%%%%%
\section{Linear NMSSE}
\label{sec:nmqsd.lin_nmsse}
% ✔ Bargman States, hilbert space valued functions
% ✔ derivation
% * problems, non-locality in noise
% ✔ reduced density operator
% * relative state --> interpretation; but also connection with H_s valued functions
% ✔ zero temperature, importance for calculations
% * quantum trajectory Carmichael
%
% TODO relation to unravelling
% TODO Make clear that interaction picture does not matter for purely system observables
%
%%%%%%%%%%%%%%%%%%%%%%%%%%%%%%%%%%%%%%%%%%%%%%%%%%%%%%%%%%%%%%%%%%%%%%%%%%%%%%%

The linear non-Markovian stochastic Schrödinger equation derived in this section is an equivalent reformulation of the interaction-picture Schrödinger equation
\begin{equation}
  \partial_t \ket{\Psi_t} = -\ii \Htot(t) \ket{\Psi_t}, \qquad \ket{\Psi_0} = \ket{\psi_0} \otimes \ket{0},
  \label{eq:nmqsd.schroedinger_ia}
\end{equation}
corresponding to the model of the last section:
Expressing the bath degrees of freedom in the Bargmann Hilbert space of anti-holomorphic functions\cite{Ba61_coherent_states} provides a representation that is well suited for a Monte-Carlo treatment.
To this end we introduce the unnormalized coherent state $\ket{z_\lambda} = \exp(z_\lambda \adj{a}_\lambda)\ket{0_\lambda}$ for each mode with resolution of the identity for the environment
\begin{equation}
  \unit = \int \frac{\exp[-\abs{\zz}^2]}{\pi^N} \, \ket{\zz}\bra{\zz} \dd^{2N}z,
  \label{eq:nmqsd.identity}
\end{equation}
Here we employ the shorthand notation $\ket{\zz} = \bigotimes_\lambda \ket{z_\lambda}$ and the \quotes{volume} integration measure for $N$ complex numbers $\dd^{2N}z = \prod_\lambda \dd\Re z_\lambda \dd\Im z_\lambda$.
Throughout this work the finite bath is often replaced by a continuum of oscillators; therefore we simply write $\mudz = \pi^{-N} \exp(-\abs{\vec z}^2) \dd^{2N}z$ to drop an explicit reference to $N$.
% TODO Rigorous existence?

\Autoref{eq:nmqsd.identity} allows us to express the full state in a time-independent environment basis
\begin{equation*}
  \ket{\Psi_t} = \int \ket{\psi_t(\cc\zz)} \otimes \ket{\zz} \mudz.
\end{equation*}
For the following derivation it is crucial to notice that the Bargmann transform $\zz \mapsto \psi_t(\cc\zz)$ is an anti-holomorphic function with values in the system's Hilbert space $\HHsys$.
Naturally it is equivalent to any other representation of the full state $\ket{\Psi_t}$.
As the coherent states are not orthogonal, but rather satisfy $\braket{\vec w}{\vec z} = \exp(\sum_\lambda \cc w_\lambda z_\lambda)$, the reduced density operator, obtained by tracing over the bath degrees of freedom, reads
%TODO Does this need a prove?
\begin{equation}
  \rho(t) = \Tr_\mathrm{env} \ket{\Psi_t}\bra{\Psi_t}
          = \int \ket{\psi_t(\cc\zz)}\bra{\psi_t(\cc\zz)} \mudz.
  \label{eq:nmqsd.reduced_matrix}
\end{equation}

% Only Schrödinger equation point of view!
% FIXME Citations of Strunz papers
After fixing the kinematic structure, the next step is to rewrite the dynamical equation:
The representation of the ladder operators follow from the usual rules $\bra\zz \adj{a}_\lambda = \cc z_\lambda \bra\zz$ and $\bra\zz a_\lambda = \partial_{\cc z_\lambda} \bra\zz$.
These expressions applied to \autoref{eq:nmqsd.schroedinger_ia} give us the system-bath Schrödinger equation in the transformed space
\begin{equation}
  \partial_t \psi_t(\cc\zz) = -\ii\Hsys\psi_t(\cc\zz)  -  \ii L \sum_\lambda \cc g_\lambda \exp[-\ii\omega_\lambda t] \cc z_\lambda \, \psi_t(\cc\zz)  -  \ii \adj{L} \sum_\lambda g_\lambda \exp[\ii\omega_\lambda t] \, \frac{\partial \psi_t}{\partial z_\lambda}(\cc\zz).
  \label{eq:nmqsd.hamiltonian_microsopic}
\end{equation}
Introducing an effective driving process like in \autoref{eq:nmqsd.force_operator}
\begin{equation}
  \ZZ_t(\cc\zz) = - \ii \sum_\lambda \cc g_\lambda \exp[\ii \omega_\lambda t] \cc z_\lambda
  \label{eq:nmqsd.stochastic_process}
\end{equation}
allows us to combine the effect of the first bath-interaction term into a single multiplication operator---or process for reasons explained in the next paragraph.
% FIXME Proof of functional chain rule?
A similar conversion works for the second term as well with the help of the functional chain rule $\frac{\partial}{\partial \cc z_\lambda} = \int \frac{\partial \ZZ_s}{\partial\cc z_\lambda} \frac{\delta}{\delta \ZZ_s} \dd s$.
Combined our new equation of motion ---the non-Markovian stochastic Schrödinger equation---reads
\begin{equation}
  \partial_t \psi_t = -\ii\Hsys\psi_t  +  L\ZZ_t\psi_t  -  \adj{L} \int_0^t \alpha(t-s) \frac{\delta \psi_t}{\delta \ZZ_s} \dd s.
  \label{eq:nmqsd.nmsse}
\end{equation}
As we shown in \autoref{sub:nmqsd.interpretation.unitary_view}, the integral boundaries arise due to the initial conditions~\ref{eq:nmqsd.initial_conditions}; a less formal argumentation goes as follows:
By construction of our processes~\ref{eq:nmqsd.stochastic_process} an initial state $\ket{\psi_0}\otimes\ket{0}$ translates to an initial $\psi_0(\ZZ)$ that is completely independent of any noise.
Then causality implies that $\psi_t$ can only depend on $\ZZ_s$ for $0 \le s \le t$.\\



Up to this point we have merely rewritten the original Schrödinger equation~\ref{eq:nmqsd.schroedinger_ia} to an equivalent form:
The original system-bath product Hilbert space $\HHsys \otimes \HHenv$ is replaced by a Hilbert space of $\HHsys$-valued functions.
A different attitude is quite fruitful, especially with a numerical solution of the NMSSE in mind:
\Autoref{eq:nmqsd.reduced_matrix} can be rewritten as $\rho_t = \E[\ket{\psi_t}\bra{\psi_t}]$, where $\E$ denotes the average over $\mudz = \pi^{-N} \exp(-\abs{\vec z}^2) \dd^{2N}z$.
Put differently the reduced density matrix $\rho_t$ arises by averaging over the stochastic pure state projectors $\ket{\psi_t(\cc\zz)}\bra{\psi_t(\cc\zz)}$ with Gaussian weight $\mudz$.
Hence we regard \autoref{eq:nmqsd.nmsse} as a stochastic differential equation for individual realisations $\psi_t(\cc\zz)$.
We refer to the later either as system state relative to $\ket{\zz}$ or, in the spirit of the stochastic Schrödinger equations emerging from continuous measurement theory \cite{Ca93_quantum_optics}, as quantum trajectory.

In this approach the driving force $\ZZ_t$ is implemented as classical stochastic process defined by the concrete version~\ref{eq:nmqsd.stochastic_process} and the underlying probability measure $\mu$.
It is a complex Gaussian process uniquely characterized by its expectation value and covariances
\begin{equation}
  %FIXME Z_t compared to \ZZ_s looks strange! The subscript is moved.
  \E\,Z_t = 0, \quad \E\,Z_t Z_s =0, \quad\mbox{and}\quad \E\,Z_t \ZZ_s = \alpha(t-s),
  \label{eq:nmqsd.process_properties}
\end{equation}
where $\alpha$ is the zero-temperature correlation function~\ref{eq:nmqsd.correlation_function} for $J(\omega) = \sum_\lambda \abs{g_\lambda}^2 \delta(\omega - \omega_\lambda)$.
By virtue of the initial conditions, $\psi_t$ depends on $\zz$ only through the driving process; thus we can drop the coherent state labels and simply write $\psi_t(\ZZ)$ denoting the trajectory corresponding to a realisation $\ZZ(\cc\zz)$.
It is this alternative point of view that makes the NMSSE-approach so powerful:
The entire influence of the environment is encoded in a complex function $\alpha$, which acts both as correlation function for the driving noise $\ZZ$ and as memory kernel for the damping term.
A generalization to an arbitrary number of bath-oscillators is now straightforward: simply replacing the correlation function allows an unified description of arbitrary harmonic environments.

Except in the limit $\alpha(t) \propto \delta(t)$, elaborated in the next paragraph, the driving process $\ZZ_t$ is correlated for different times.
This non-Markovian behavior, which makes a complete understanding of the dynamics highly desirable for application but also considerably harder, shows up in the equation of motion~\ref{eq:nmqsd.nmsse} as well.
The damping term contains the functional derivative over the whole timespan and therefore takes the complete history of $\psi_t(\ZZ)$ into account.
In its own right the derivative is just as problematic:
% FIXME Quite long sentence!
Since its computation requires not only the single realisation $\ZZ$, but in some sense all adjacent ones as well, it seems questionable to regard the NMSSE~\ref{eq:nmqsd.nmsse} as a genuine stochastic differential equation \cite{GaWi02_real_nmsse}.
% FIXME DOOMED???
Even from the purely pragmatic point of view both kinds of non-local behavior complicate a direct numerical simulation of the NMSSE, if not making it completely impracticable.
Nevertheless there are two quite distinct solutions as shown in \autoref{sub:nmqsd.lin_nmsse.convolutionless} and \autoref{chap:num}.


%%%%%%%%%%%%%%%%%%%%%%%%%%%%%%%%%%%%%%%%%%%%%%%%%%%%%%%%%%%%%%%%%%%%%%%%%%%%%%%
\subsection{Markov Limit}
\label{sub:nmqsd.markov}
% * markov limit
% * problem with negative energy oscillators
% * relation to lindblad/markovian sse
% * Ito vs Stratonovich
%
% FIXME Polish introduction
% FIXME Is dropping γ wise?

The best understood open systems are Markovian.
Based upon two physical assumptions, namely
% FIXME Good style?, What does memoryless actually mean?
\begin{description}
  \item[weak coupling] of the system to the reservoir and
  \item[memoryless environment,] that is the time evolution is completely time-local,
\end{description}
it is possible to derive a general form of a master equation governing the reduced dynamics \cite{Li76_generators_qdsg}.
Of course, the NMSSE is much more general.
It is only in the standard Markovian limit $\alpha(t) = \gamma\delta(t)$ we can expect to obtain an equation of motion that describes a reduced time evolution without memory.
A rescaling of the coupling operator $L$ allows us to set $\gamma = 1$ without loss of generality.

The vacuum initial conditions $\frac{\delta \psi_0}{\delta \ZZ_s} = 0$ with $s \in \Reals$ imply for an arbitrary bath correlation function
\begin{equation}
  \frac{\delta \psi_t}{\delta \ZZ_t} = \frac{1}{2} \, L \psi_t \qquad (t > 0)
  \label{eq:nmqsd.deriv_psit}
\end{equation}
as we show now.
% FIXME Really? Only due to interaction picture.
It is clear from its derivation that the NMSSE describes a unitary, time-dependent evolution.
Therefore it can be solved formally using the Dyson series
\begin{equation}
  \psi_t(\ZZ) = \sum_{n=0}^\infty (-\ii)^n \intl{0}{t}{t_1} \intl{0}{t_1}{t_2} \dots \intl{0}{t_{n-1}}{t_n}  \Htot(t_1) \dots \Htot(t_n) \, \psi_0,
  \label{eq:nmqsd.dyson}
\end{equation}
where $\Htot(t)$ is the reformulation of~\ref{eq:nmqsd.Htot} given by
\begin{equation*}
  -\ii \Htot(t) = -\ii \Hsys + L \ZZ_t - \adj{L} \intl{-\infty}{\infty}{s} \alpha(t-s) \frac{\delta}{\delta \ZZ_s}.
\end{equation*}
Throughout this work we often use the shorthand notation $\adjZZ_t = \int\mathrm{d}s \, \alpha(t-s) \frac{\delta}{\delta \ZZ_s}$ for the last term.

% FIXME More details?, Use Commutator notation instead?
Applying the functional derivative $\frac{\delta}{\delta \ZZ_s}$ to $\Htot(t)$ gives a single contribution $\ii \delta(t - s) L$, since both $\Hsys$ and $\adjZZ_t$ are independent of the noise.
% FIXME
% FIXME Talk about this with W
% FIXME
This allows us to calculated $\frac{\delta \psi_t}{\delta \ZZ_t}$ order by order in \autoref{eq:nmqsd.dyson}---the derivative of $\psi_0$ vanishes as imposed by the initial conditions.
We obtain for the term with a $n$-fold time-integral neglecting a constant phase
\begin{equation*}
  \intl{0}{t}{t_1} \dots \intl{0}{t_{n-1}}{t_n} \Big( \delta(t_1 - t) L \Htot(t_2) \dots \Htot(t_n) + \dots + \delta(t_n - t) \Htot(t_1) \dots L \Big).
\end{equation*}
We notice that the $i$\th summand contributes only if $t_i = t$:
For $i = 1$ this is exactly the integral boundary while for $i=2,\dots,n$ the integral boundary reaches $t$ only for $t_1 = t$.
As the latter condition has vanishing weight under the $t_1$ integral we eventually find \autoref{eq:nmqsd.deriv_psit}.

Let us return to the Markov limit of our NMSSE\@.
By virtue of the singular correlation function $\alpha = \delta$ the time-nonlocal damping operator reduces to a time-local form $\adjZZ_t = \frac{\delta}{\delta \ZZ_t}$ as it is expected from a memoryless environment.
Combined with \autoref{eq:nmqsd.deriv_psit} this leads to simple stochastic differential equation
\begin{equation*}
  \partial_t \psi_t(\ZZ) = -\ii \Hsys \psi_t(\ZZ) + L\ZZ_t\psi_t(\ZZ) - \frac{1}{2}\adj{L}L\psi_t(\ZZ),
\end{equation*}
driven by a complex White Noise $Z_t$ with $\E{Z_t \ZZ_s} = \delta(t-s)$.
In a formally exact fashion the equation above should be written as
\begin{equation}
  \dd\psi_t = (-\ii\Hsys\psi_t - \frac{1}{2} \adj{L}L \psi_t) \dd t + L\psi_t \dd \cc\xi_t
  \label{eq:nmqsd.ito}
\end{equation}
with a standard complex Brownian motion $\xi_t$.
It is well known that such stochastic differential equations are problematic as $\xi_t$ is not differentiable with respect to time.
To define the solution $\psi_t$ uniquely we need to specify an appropriate interpretation of the stochastic differential equation \cite[p.~36]{Ok03_sde}:
We imagine the Brownian motion as a limit of stochastic processes $\xi^{(n)}_t \to \xi_t$, such that $\xi^{(n)}_t$ are continuously differentiable with respect to time.
Replacing the Brownian motion in \autoref{eq:nmqsd.ito} by $\xi^{(n)}_t$ transforms it into a deterministic differential equation.
The limit of corresponding solutions $\psi^{(n)}_t$ coincides with $\psi_t$ only if we understand \autoref{eq:nmqsd.ito} it the Stratonovich sense.
However, in our case the It\=o- and Stratonovich form agree since $\E Z_t Z_s = 0$ \cite{GaCr85_handbook}.\\

% TODO Uncomment if cool
%The Belavkin or simply stochastic Schrödinger equation~\ref{eq:nmqsd.ito} is a well known result in continuous measurement theory and quantum optics, where it appears as an unravelling of the Linblad master equation \cite{BaGr09_trajectories,???}.
%Nevertheless our main ingredient in its derivation, namely the singular bath correlation function $\alpha = \delta$, shows how unphysical the Markov assumption is:
%As $\alpha$ is given by the Fourier transform of the spectral density $J$, this amounts to a system coupled evenly to Oscillators of arbitrary frequency.
%Besides
%% TODO Complete!
%% negative frequencies, no problem here; timescales

%%%%%%%%%%%%%%%%%%%%%%%%%%%%%%%%%%%%%%%%%%%%%%%%%%%%%%%%%%%%%%%%%%%%%%%%%%%%%%%
\subsection{Convolutionless Formulation}
\label{sub:nmqsd.lin_nmsse.convolutionless}
% * on the existence
% * why it solves problems, true stochastic equation
% * dynamics
% * application

% FIXME
As a cure for the non-locality issues, Diósi, Gisin, and Strunz \cite{DiGiSt98_nmqsd} proposed the powerful $O$-Operator substitution:
It is based on the additional assumption, that one may replace the functional derivative by a system operator $O$, which only depends on the realisation of $\ZZ$ itself
\begin{equation}
  \frac{\delta \psi_t(\ZZ)}{\delta \ZZ_s} = O(t, s, \ZZ) \psi_t(\ZZ).
  \label{eq:nmqsd.o_substition}
\end{equation}
Besides getting rid of the derivative, this substitution enables us to derive a convolutionless form of our NMSSE~\ref{eq:nmqsd.nmsse}
\begin{equation}
  \partial_t \psi_t = -\ii\Hsys\psi_t(\ZZ)  +  L\ZZ_t\psi_t(\ZZ)  -  \adj{L} \bar O(t, \ZZ) \psi_t(\ZZ)
  \label{eq:nmqsd.nmsse_o}
\end{equation}
with the time-local operator
\begin{equation}
  \bar O(t, \ZZ) := \int_0^t \alpha(t - s) O(t, s, \ZZ) \dd s.
  \label{eq:nmqsd.o_bar}
\end{equation}
Conclusively \autoref{eq:nmqsd.nmsse_o} turns into a genuine stochastic differential equation for the trajectory $\psi_t(\ZZ)$, but in the much smaller Hilbert space of the system.
This makes it exceptionally well suited for dealing with infinite sized environments numerically, provided the $\bar O$-operator is known.
Depending on the validity of the $O$-substitution the corresponding convolutionless NMSSE~\ref{eq:nmqsd.nmsse_o} might be as accurate as the original microscopic equation of motion~\ref{eq:nmqsd.schroedinger_ia}.

For a few simple systems---for example the Jaynes-Cummings model presented in \autoref{sec:nmqsd.twolevel} or its higher dimensional generalizations \cite{JiZhYo12_exact_nmqsd}---an exact analytic expression for $O$ is known.
In these rare cases one proceeds as follows \cite{DiGiSt98_nmqsd}:
From the consistency condition
\begin{equation}
  \partial_t \frac{\delta \psi_t(\ZZ)}{\delta \ZZ_s} = \frac{\delta}{\delta \ZZ_s} \partial_t \psi_t(\ZZ)
  \label{eq:nmqsd.consistency_condition}
\end{equation}
and the initial condition familiar from \autoref{sub:nmqsd.markov}
% FIXME This does not aggree with Markov!!!
\begin{equation}
  O(s, s, \ZZ) = L
  \label{eq:nmqsd.o_initial}
\end{equation}
we derive an equation of motion for $O(t, s, \ZZ)$.
It still contains the functional derivative, but is converted to a system of coupled, deterministic equations using a power series ansatz
\begin{equation}
  O(t, s, \ZZ) = \sum_{n=0}^\infty \int_0^t \dots \int_0^t O_n(t, s, \nu_1, \dots, \nu_n) \dd \nu_1 \dots \nu_n.
  \autoref{eq:nmqsd.o_series}
\end{equation}
Nevertheless most treatments rely on approximation schemes, for example a perturbation expansion for small coupling parameter or almost-Markovian environments \cite{YuDiGiSt99_pertubation}.
% FIXME Citation
Also a closely related hierarchy of $O$-operators provides an efficient numerical algorithm similar in concept to the main result of this work \cite{}.

%%%%%%%%%%%%%%%%%%%%%%%%%%%%%%%%%%%%%%%%%%%%%%%%%%%%%%%%%%%%%%%%%%%%%%%%%%%%%%%
\subsection{Equivalent Master Equations}
\label{sub:nmqsd.lin_nmsse.master}
%TODO Change title
%TODO More references to prior work?
% * existence of master equation
% * relation to lindblad

In the previous section we have introduced a convolutionless formulation primarily to simplify the treatment of the NMSSE\@.
But the $O$-operator substitution is also essential clarify the connection to the master equations commonly used in the theory of open quantum systems.
The latter are formulated in terms of reduced density operators, which we recover from the trajectories by averaging over the pure states projectors $P_t = \ket{\psi_t(\ZZ)}\bra{\psi_t(\ZZ)}$.
For certain systems this can be done analytically in order to derive a equivalent master equation.

As a simple example we focus on models with a $\ZZ$ independent $\bar O$-operator such as the two-level system presented in \autoref{sec:nmqsd.twolevel}.
We follow the lines of Yu et al.~\cite{YuDiGiSt99_pertubation,YuDiGi00_master}, who also treat the general case using the functional expansion~\ref{eq:nmqsd.o_series}.
The pure states projectors' equations of motion,
\begin{equation}
  \partial_t P_t = -\ii [\Hsys, P_t] + \ZZ_t L P_t - \adj{L}\bar O(t)P_t + Z_t P_t \adj{L} - P_t \adj{\bar O(t)} L,
  \label{eq:nmqsd.pt_eom}
\end{equation}
yield a closed evolution equation for $\rho_t$ after averaging over the bath degrees of freedom only if we can restate the terms containing $\ZZ_t$ in a noise-independent manner.
This can be done with the help of Novikov's formula \cite{No65_functionals}
\begin{equation}
  \E[Z_t P_t] = \E[\intdd s \alpha(t - s) \frac{\delta}{\delta \ZZ_s} P_t].
  \label{eq:nmqsd.novikov}
\end{equation}
% \FIXME
A formal proof is provided in \autoref{sec:???}, but the main idea is simple:
Under a Gaussian integral $\intdd^2 z \, \exp(-\abs{z}^2) \dots$ the multiplication by $z$ can be rewritten as a derivation $\partial_{\cc z}$.
Partial integration yields a result similar to \autoref{eq:nmqsd.novikov}.

The right hand side of Novikov's formula is simplified further using the $O$-operator substitution.
Since $\ket{\psi_t}$ is analytical in $\ZZ$ and accordingly $\bra{\psi_t}$ analytical in $Z_t$, the derivative is further simplified to
\begin{equation*}
  \frac{\delta}{\delta \ZZ_s} \bigg( \ket{\psi_t(\ZZ)}\bra{\psi_t(\ZZ)} \bigg) = \left( \frac{\delta}{\delta \ZZ_s} \ket{\psi_t(\ZZ)} \right)\bra{\psi_t(\zz)} = O(t, s) \ket{\psi_t(\ZZ)}\bra{\psi_t(\ZZ)}.
\end{equation*}
Averaging over the equations of motion for the pure state projectors~\ref{eq:nmqsd.pt_eom} finally gives the master equation for the reduced density matrix $\rho_t$
\begin{equation}
  \partial_t \rho_t = -\ii [\Hsys, \rho_t]  +  [L, \rho_t \adj{\bar O(t)}]  +  [\bar O(t) \rho_t, \adj{L}].
  \label{eq:nmqsd.master}
\end{equation}
%FIXME Reference
This expression closely resembles the well known Lindblad master equation~\ref{eq:intro.} for Markovian open quantum systems, but involves time-dependent Linbladians.
%FIXME Correct form of O(s,s,Z)
As elaborated in \autoref{sub:nmqsd.markov} the $\bar O$-operator reduces to $\bar O(t) = \frac{\gamma}{2} L$ in the Markovian limit---thus our NMSSE reproduces the correct limit.


%%%%%%%%%%%%%%%%%%%%%%%%%%%%%%%%%%%%%%%%%%%%%%%%%%%%%%%%%%%%%%%%%%%%%%%%%%%%%%%
\section{Nonlinear NMSSE}
\label{sec:nmqsd.nonlin_nmsse}
% * non-uniquesness of unravelling ==> used here to change weights
% * Motivation (Monte Carlo!, normalized states)
% * derivation
% * discussion
%
% TODO Ask W if non-analyticity of Φ needs to be mentioned explicitely
%%%%%%%%%%%%%%%%%%%%%%%%%%%%%%%%%%%%%%%%%%%%%%%%%%%%%%%%%%%%%%%%%%%%%%%%%%%%%%%

From a fundamental point of view the linear non-Markovian stochastic Schrödinger equation~\ref{eq:nmqsd.nmsse} is fascinating in its own right.
It provides a unified description of arbitrary structured environments and admits a figurative interpretation presented in \autoref{sec:nmqsd.interpretation}.
But there is a major drawback when it comes to practical application in terms of Monte-Carlo simulations:
Recall that the reduced density matrix $\rho_t$ is obtained by averaging over individual quantum trajectories $\psi_t(\ZZ)$.
The fineness of such a scheme is drastically reduced if there are few highly peaked contributions \cite{DuSh11_monte_carlo}.
% FIXME Is there other evidence than numerical? Entanglement?
As demonstrated exemplary for the spin-boson model in \autoref{sub:num.spin_boson.sample_size}, the NMSSE displays such a behavior:
The norm of most trajectories goes to zero as $t \to \infty$ due to ever growing entanglement.
To recover a unitary time evolution for both, system and bath, that is $\E[\braket{\psi_t}{\psi_t}] = \braket{\Psi_t}{\Psi_t} = 1$, few trajectories with significant contribution have to be taken into consideration.
This requires an insurmountable sample size for certain parameter regimes as further elaborated in the referred section.\\

We have just described the problem of importance sampling in statistics; for its solution let us return to the microscopical model from \autoref{sec:nmqsd.model}.
The key observation is that the average yielding the reduced density operator~\ref{eq:nmqsd.reduced_matrix} is not unique:
A change in the integration measure $\mudz$ can be compensated using a Girsanov transformation on the quantum trajectories $\psi_t(\ZZ)$.
We use this fact to rewrite the density operator as an average over normalized states
% FIXME Same type of integral, looks strange
\begin{align}
  \label{eq:nmqsd.rho_husimi}
  \rho_t &= \int \frac{\mathrm{d}^{2N} z}{\pi^N} \, \exp[-\abs{\zz}^2] \braket{\psitz}{\psitz} \, \frac{\ket{\psitz}\bra{\psitz}}{\braket{\psitz}{\psitz}} \\
         &= \int Q_t(\zz, \cc\zz) \, \frac{\ket{\psitz}\bra{\psitz}}{\braket{\psitz}{\psitz}} \mathrm{d}^{2N} z, \nonumber
\end{align}
now with a time dependent density function under the integral
\begin{equation}
  Q_t(\zz, \cc\zz) = \frac{\exp[-\abs{\zz}^2]}{\pi^N}\, \braket{\psitz}{\psitz}
                   = \frac{\exp[-\abs{\zz}^2]}{\pi^N}\, \bra{\zz} \Tr{sys} \big( \ket{\Psi_t}\bra{\Psi_t} \big)\ket{\zz}.
    \label{eq:nmqsd.husimi}
\end{equation}
Remarkably the latter coincides with the Husimi- or Q-function\footnote{%
  We point out that the Husimi function is usually defined in terms of normalized coherent states, hence the additional factor $\exp(-\abs{z_\lambda}^2)$ for each oscillator in our notation.
}
of the environmental oscillators\cite{Sc11_quantum_optics}.
Being non-negative and normalized to unity $\int Q(\zz, \cc\zz) \dd z = 1$ makes the Husimi-function a genuine (quasi)-probability distribution on phase space.
In this representation $\ket{z}$ resembles a wave packet localized around $z = (q + \ii \, p) / \sqrt{2}$, thusly there is a well defined correspondence between coherent state labels $z$ and the canonical variables $(q, p)$.
With \autoref{eq:nmqsd.rho_husimi} we conclude that the norm of a trajectory $\psitz$ determines the probability to find the environment in a quantum state localised in the vicinity of a point $(\vec q, \vec p)$ in phase space.
So instead of using a fixed environmental basis $\ket{\zz}$ to expand the full state $\ket{\Psi_t}$, we can incorporate the dynamics of the environment in comoving coherent state basis.

Making use of the microscopic Hamiltonian~\ref{eq:nmqsd.hamiltonian_microsopic} and the analyticity of $\psitz$ in $\cc\zz$, that is $\partial_{z_\lambda} \ket{\psitz} = 0$, we obtain an equation of motion, which closely resembles a Liouville equation
\begin{equation}
  \partial_t Q_z(\zz, \cc\zz) = - \sum_\lambda \partial_{\cc z_\lambda} \big( \ii g_\lambda \exp[-\ii \omega_\lambda t] \, \qmean{\adj L}_t \, Q_t(\zz, \cc\zz) \big) - \mathrm{c.c.}
  \label{eq:nmqsd.qdot}
\end{equation}
It is obvious that it contains the full back-reaction of the system due to the quantum average\footnote{%
  %FIXME Is this necessary?
  We do not indicate its (non-holomorphic) dependence on $\cc\zz$ explicitly because our main goal is not the solution of \autoref{eq:nmqsd.qdot}.
  Instead $Q_t$ is only used to derive a normalized versions of our NMSSE-trajectories.
}
\begin{equation*}
  \qmean{\adj L}_t = \frac{\bra{\psitz} \adj L \ket{\psitz}}{\braket{\psitz}{\psitz}}.
\end{equation*}
Exactly as its counterpart from classical mechanic, \autoref{eq:nmqsd.qdot} is solved using the method of characteristics
The corresponding drift velocities are given by
\begin{equation}
  \cc{\dot z}_\lambda(t) = \ii g_\lambda \exp[-\ii \omega_\lambda t] \qmean{\adj L}_t.
  \label{eq:nmqsd.zdot}
\end{equation}
% TODO Ask W if non-analyticity of Φ needs to be mentioned explicitely
We denote the corresponding flow by $\vec\phi_t$, or using the more common notation $\cc z_\lambda(t) = \cc \phi_{\lambda,t}(\cc z_\lambda)$ with initial conditions $\cc z_\lambda(0) = \cc\phi_{\lambda, 0}(\cc z_\lambda) = \cc z_\lambda$.
\Autoref{eq:nmqsd.zdot} admits the following interpreation:
%FIXME Read again!
If we start with a total state $\ket{\psitz}\otimes\ket{\zz}$ of system and environment at time $t$, then the most relevant contribution to the full state at $t + \Delta t$ corresponds to the coherent state $\ket{\zz + \dot\zz(t) \Delta t}$.
For this reason we should expand $\ket{Psi_t}$ with respect to $\ket{\zz(t)}$ instead of a fixed basis in order to capture the dominant proportion and avoid propagating states irrelevant for the final average.

It is the method of characteristics' essential point that the flow $\vec\phi_t$ yields a solution of \autoref{eq:nmqsd.qdot} by
\begin{equation}
  Q_t(\zz, \cc\zz) = \int Q_0(\zz_0, \cc\zz_0) \, \delta(\zz - \vec\phi_t(\zz_0)) \dd^{2N} z_0,
  \label{eq:nmqsd.q_characteristics}
\end{equation}
where $\delta(\zz - \zz') = \prod_\lambda \delta(\Re(z_\lambda - z_\lambda')) \delta(\Im(z_\lambda - z_\lambda'))$.
Our initial product state $\ket{\Psi_0} = \ket{\psi_0} \otimes \ket{\zz}$ combined with \autoref{eq:nmqsd.husimi} suggest, that the Husimi-function at $t=0$ is exactly the original Gaussian weight $Q_0(\zz, \cc\zz) = \pi^{-N} \exp[-\abs{\zz}^2]$ for a time-independent coherent state basis.
Therefore \autoref{eq:nmqsd.q_characteristics} finally reduces \autoref{eq:nmqsd.rho_husimi} to the sought-after average over normalized trajectories, but now with a time-independent probability measure
\begin{equation}
  \rho_t = \int \frac{\mathrm{d}^{2N}z}{\pi^N} \, \exp[-\abs{\zz}^2] \, \frac{\ket\psitphi \bra\psitphi}{\braket{\psitphi}{\psitphi}}
         = \E[ \frac{\ket{\tilde\psi_t}\bra{\tilde\psi_t}}{\braket{\tilde\psi_t}{\tilde\psi_t}}].
  \label{eq:nmqsd.reduced_matrix_comoving}
\end{equation}
Here we have introduced relative states $\tilde\psi_t(\cc\zz) = \psi_t(\vec\phi_t(\cc\zz))$ correspondig to the comoving coherent basis\footnote{%
  It is more common in the literature to introduce normalized trajectories $\psi'_t = \tilde\psi_t / \abs{\tilde\psi_t}$ immediately without any reference to $\tilde\psi_t$.
  In this work the interim state $\tilde\psi_t$ plays a particularly important role for the hierarchical equation of motion and is therefore designated explicitly.
}.\\



Of course a practical application of \autoref{eq:nmqsd.reduced_matrix_comoving} in a Monte-Carlo calculation crucially depends upon whether a closed equation of motion for $\tilde\psi_t$ exists.
Remarkably the latter satisfy a nonlinear version of our convolutionless NMSSE~\ref{eq:nmqsd.nmsse_o} as we show now starting from
\begin{equation}
  \partial_t (\psi_t \circ \cc{\vec\phi}_t) = \partial_t\psi_t \circ \cc{\vec\phi}_t + \sum_\lambda (\partial_{\cc z_\lambda} \psi_t \circ \cc{\vec\phi}_t) \cdot (\partial_t \ccphitla).
  \label{eq:nmqsd.psiprime_dot}
\end{equation}
The flow $\cc{\vec\phi}$ in the first term amounts to evaluating the original equation of motion at the comoving coherent state $\zz(t)$.
Using its integral form
\begin{equation}
  \ccphitla(\cc z_\lambda) = \cc z_\lambda + \ii g_\lambda \int_0^t \exp(-\ii \omega_\lambda s) \qmean{\adj L}_s \dd s
  \label{eq:nmqsd.comoving_flow}
\end{equation}
plugged into the microscopic version of the process~\ref{eq:nmqsd.stochastic_process} yields a shifted stochastic driving force
\begin{equation}
  \tildeZZ_t(\cc\zz) := \ZZ_t(\cc{\vec\phi}_t(\cc\zz)) = \ZZ_t(\cc\zz) + \int_0^t \cc{\alpha(t-s)} \qmean{\adj L}_s \dd s.
  \label{eq:stochastic_process_shiften}
\end{equation}
Since the $O$-operator substitution ensures that the equations of motion for $\psi_t$ are local with respect to $\ZZ$, the first summand on the left hand side of \autoref{eq:nmqsd.psiprime_dot} is obtained replacing $\ZZ_t$ by $\tildeZZ_t$ in the convolutionless NMSSE.

For the second summand, which is due to the intrinsic time dependence of the shifted coherent states, we utilize the functional chain rule once again
\begin{align*}
  \sum_\lambda \frac{\partial\ccphitla}{\partial t}(\cc z_\lambda) \cdot \frac{\partial\psi_t}{\partial \cc z_\lambda} (\cc{\vec\phi}_t(\cc\zz))
  &= \ii \sum_\lambda g_\lambda \exp[-\ii \omega_\lambda t] \qmean{\adj L}_t \, \frac{\partial\psi_t}{\partial \cc z_\lambda} (\cc{\vec\phi}_t(\cc\zz)) \\
  &= \qmean{\adj L}_t \, \int_0^t \alpha(t - s) \frac{\delta \psi_t}{\delta \ZZ_s} (\cc{\vec\phi}_t(\cc\zz)) \dd s \\
  &= \qmean{\adj L}_t \bar O(t, \tildeZZ) \tilde\psi_t(\tildeZZ),
\end{align*}
where the last line reflects the definition of the $\bar O$-operator in \autorefs{eq:nmqsd.o_substition} and~\ref{eq:nmqsd.o_bar}.
Both terms of~\ref{eq:nmqsd.psiprime_dot} combined yield the desired equation for $\tilde\psi_t$
\begin{equation}
  \partial_t \tilde\psi_t = -\ii\Hsys \tilde\psi_t + L\tildeZZ_t\tilde\psi_t - (\adj L - \qmean{\adj L}_t) \bar O(t, \tildeZZ) \tilde\psi_t.
  \label{eq:nmqsd.nmsse_nonlin}
\end{equation}
Although $\tilde\psi_t$ allows taking the average over normalized states, its equation of motion~\ref{eq:nmqsd.nmsse_nonlin} does not preserve normalization over time.
This can be achieved by adding further nonlinear terms:
Consider trajectories $\ket{\psi'_t(\tildeZZ)} = \ket{\tilde\psi_t(\tildeZZ)} / \abs{\tilde\psi_t(\tildeZZ)}$, then it is straightforward to derive the corresponding equation of motion \cite{DiGiSt98_nmqsd}
\begin{align}
  \partial_t\psi'_t &= -\ii\Hsys\psi'_t  +  \left(L - \qmean{L}_t\right) \tildeZZ_t\psi'_t  \nonumber \\
  &-  \left( (\adj{L} - \qmean{\adj L}_t) \bar O(t, \tildeZZ) - \qmean{(\adj{L} - \qmean{\adj L}_t) \bar O(t, \tildeZZ)} \right) \psi'_t
  \label{eq:nmqsd.nmsse_nonlin_full}
\end{align}
Once again the Markov-limit amounts to $\bar O(t, \tildeZZ) = \frac{1}{2} L$ and replacing the driving process $\ZZ_t$ by a complex White Noise.
Thus we obtain the well-known nonlinear unravelling of a Lindblad master equation \cite{BaGr09_trajectories}.\\



As mentioned in the motivation, the nonlinear equations should be given precedence over the linear version when it comes to Monte-Carlo simulation.
%They allow us to compute the density matrix as an average over realizations with contributions in the same order of magnitude with respect to a time independent probability distribution.
It does require propagating a new time-nonlocal, scalar quantity, namely $\qmean{\adj L}_s$ for $0 \le s \le t$ in the shifted noise $\tildeZZ_t$.
Nevertheless the improved convergence with respect to the number of realizations compensates by far the additional computational demands as elaborated in \autoref{sub:num.spin_boson.sample_size}.

%%%%%%%%%%%%%%%%%%%%%%%%%%%%%%%%%%%%%%%%%%%%%%%%%%%%%%%%%%%%%%%%%%%%%%%%%%%%%%%
\section{Interpretation of NMSSE}
\label{sec:nmqsd.interpretation}
%
% TODO Functional Taylor series for Jaynes-Cumming
%%%%%%%%%%%%%%%%%%%%%%%%%%%%%%%%%%%%%%%%%%%%%%%%%%%%%%%%%%%%%%%%%%%%%%%%%%%%%%%

% * in contrast to class. mechanics pure state for open systems problematic even if total state is pure
% * however Markov SSE more than convenient unravelling; but also wavefuntion collapse models and continous measuremts
%     => cause: entanglement
%     => solution of SSE interpreted as trajectories of post measurement states as measurements destroy entanglement
%     => "physical reality"
% * normalized linear NMSSE solutions: for large class of systems (O linear in ZZ) cite
%     * necessary crtiterion for measurement interpreation (ALL possible measurement schemes) for model considered (product initial state)
%     * examples considered there interpreation only possible for α ~ δ
%        => cause: compatibility of measurements violated due to memory

In contrast to classical mechanics, assigning a reduced pure state to an open quantum system fails in general.
Due to entanglement with the environment built up by interaction, the reduced state is consistently described only by a mixed state.
However, quantum trajectories obtained as solutions of a linear or nonlinear Markovian stochastic Schrödinger equation are more than simply a convenient tool for calculations:
These arise for example as trajectories of post-measurement pure states conditioned on a time-continuous measurement outcome \cite{Ca93_quantum_optics,BaGr09_trajectories}, because the interaction with a measurement apparatus destroys system-environment entanglement.

The question, whether the NMSSE investigated in this work admits a similar interpretation, has been answered negative for a large class of models only recently:
Krönke and Strunz \cite{KrSt12_trajectories} derived a consistency condition necessary for a measurement interpretation of the linear, convolutionless NMSSE with an $O$-operator that depends at most linear on the noise process.
This condition is violated by all exemplary systems they studied unless in the Markov limit $\alpha\propto\delta$.\\

% * Construct Hilbert space by E<psi_t|psi_t>
% * Interpretation of Z_s and δ/δZ_s as ladder operators, but not adjoint; Novikov; calculation rules
% * functional taylor expansion
% * Picture of interaction with time oscillators
% * make contact with classical Brownian motion picture
% * causal interpreation due to vacuum initial conditions; derivation of bounded integral domain

We now present a different, less practical and more figurative interpretation of the linear NMSSE~\ref{eq:nmqsd.nmsse}.
% FIXME Generalized?
The main goal is to establish the NMSSE as an alternative description for the unitary system-bath time evolution in terms of a \quotes{generalized environment}.
At first we state the Hilbert space used in the description:
% TODO Complete

%%%%%%%%%%%%%%%%%%%%%%%%%%%%%%%%%%%%%%%%%%%%%%%%%%%%%%%%%%%%%%%%%%%%%%%%%%%%%%%
\section{Finite Temperature Theory}
\label{sec:nmqsd.temperature}
% * why necessary
% * why approach does not work anymore
%
% TODO quantum vs. classical noise
% FIXME Include unitary noise?
%%%%%%%%%%%%%%%%%%%%%%%%%%%%%%%%%%%%%%%%%%%%%%%%%%%%%%%%%%%%%%%%%%%%%%%%%%%%%%%

Until now we were only concerned with the temperature zero theory, which is characterized by an initial product state with the environment in the vacuum state $\ket{\Psi_0} = \ket{\psi_0} \otimes \ket{\vec 0}$.
It translates into our NMSSE-framework as the demand of vanishing functional derivatives $\frac{\delta \psi_0}{\delta \ZZ_s} = 0$ for arbitrary $s$ at $t=0$.
This property ensures the bounded integral domain of the functional derivative, which on the other hand is crucial for a causal interpretation in terms of time-oscillators.
Also both $O$-operator substitution as well as hierarchical equations of motion depend on the temperature-zero assumption.
% FIXME IF unitary noise included these are two methods
In order to treat a thermal environment at arbitrary temperature, we devise a method that map to the vacuum initial conditions.

Let us consider an initial product, where the bath assumes a Gibbs state $\rho(\beta) = \frac{\exp[-\beta \Henv]}{Z}$; more precisely
\begin{equation}
  \rho_0 = \ket{\psi_0}\bra{\psi_0} \otimes \rho(\beta)
  \label{eq:nmqsd.initial_rho}
\end{equation}
with the bath partition function $Z = \Tr \exp[-\beta \Henv]$ at inverse temperature $\beta = T^{-1}$.\\
%FIXME Further examples; Complete
%Such a state occurs for example in the treatment of exitonic energy transfer in molecules---see \autoref{sec:app.fmo} for the details.
%There $\psi_0$ describes a one-exciton state caused by a laser-pulse stimulation at time $t=0$.
%Since


%%%%%%%%%%%%%%%%%%%%%%%%%%%%%%%%%%%%%%%%%%%%%%%%%%%%%%%%%%%%%%%%%%%%%%%%%%%%%%%
%\subsection{Thermo Field Method}
%\label{sub:nmqsd.temperature.thermofield}
% * method
%
% TODO Physical interpreatation of the thermal occupation number prefactors? Spontaneous and induced excitation/relaxation?
% TODO Purely real α --> classical thermal noise

% FIXME Citation
Thermo field dynamics was first introduced as a real-time approach to quantum fields at finite temperature \cite{}.
It is favored over other methods in the application to the NMSSE since it preserves the equation of motion as shown below.
Additionally it constitutes a consistent way to include negative frequency oscillators in the environment, which on the other hand are required for the bath correlation function used to derive our hierachy.
The last point is further elaborated in \autoref{sec:num.expansion}.
In the course of this section we follow the more detailed accounts of Yu and Strunz \cite{Yu04_heat_bath,St01_habil}.

The main idea is to introduce a second fictitious bath of oscillators $\mathcal{B}$, which is independent from the physical environment $\mathcal{A}$ and does not interact with the system.
Expressing its degrees of freedom in ladder operators $b_\lambda$ and $\adj{b}_\lambda$ gives us the new Hamiltonian in the Schrödinger picture
\begin{equation}
  \Htot = \Hsys \otimes \unit + \sum_\lambda (\cc{g}_\lambda L\otimes\adj{a}_\lambda + g_\lambda \adj{L}\otimes a_\lambda) + \unit \otimes \sum_\lambda \omega_\lambda (\adj{a}_\lambda a_\lambda - \adj{b}_\lambda b_\lambda).
  \label{eq:nmqsd.Htot_thermal}
\end{equation}
Although the corresponding eigen-energies are not bounded from below due to negative frequencies of the fictitious oscillators, there are no stability problems since the latter do not interact with the physical degrees of freedom.
For the same reason the reduced dynamics obtained from \autoref{eq:nmqsd.Htot_thermal} are identical to the original microscopical model~\ref{eq:nmqsd.Htot}, provided there is no initial entanglement of $\mathcal{B}$ with the system.
Both yield equal reduced density matrices provided we choose an initial state $\tilde\rho$ for environments $\mathcal{A}$ and $\mathcal{B}$ that reproduces \autoref{eq:nmqsd.initial_rho} after tracing over the unphysical degrees of freedom, that is
\begin{equation}
  \Tr_\mathcal{B} \tilde\rho = \rho(\beta).
  \label{eq:nmqsd.rho_tilde}
\end{equation}
Here $\Tr_\mathcal{B}$ denotes the partial trace with respect to the fictitious degrees of freedom.

Remarkably a solution $\tilde\rho$ of \autoref{eq:nmqsd.rho_tilde} is given by a pure state projector on a vacuum state with respect to new annihilation operators $A$, $B$.
They are connected to the old ladder operators by a temperature dependent Bogoliubov transformation
\begin{align*}
  A_\lambda &= \sqrt{\bar n_\lambda + 1} \, a_\lambda + \sqrt{\bar n_\lambda} \, \adj{b}_\lambda \\
  B_\lambda &= \sqrt{\bar n_\lambda} \, \adj{a}_\lambda + \sqrt{\bar n_\lambda + 1} \, b_\lambda,
\end{align*}
with $\bar n_\lambda = \left( \exp(\beta \omega_\lambda) - 1 \right)^{-1}$ denoting the mean thermal occupation number of the (physical) oscillator mode $\lambda$.
% FIXME Add citation
An extensive but elementary calculation \cite{} reveals that $\ket{0_{AB}}\bra{0_{AB}}$ with $\ket{0_{AB}} = \ket{0_A} \otimes \ket{0_B}$ satisfies \autoref{eq:nmqsd.rho_tilde}.
The doubling in degrees of freedom ensures that the reduced density matrix obtained from an initial pure state $\ket{\tilde\Psi_0} = \ket{\psi_0}\otimes\ket{0_{AB}}$ in the enlarged Hilbert space coincides with the original state at finite temperature~\ref{eq:nmqsd.initial_rho}, but lacking unphysical bath oscillators.
Expressed in these new coordinates the total Hamiltonian~\ref{eq:nmqsd.Htot_thermal} reads
\begin{align}
  \Htot = \Hsys\otimes\unit &+ \sum_\lambda \sqrt{\bar n_\lambda + 1} \, \left(\cc g_\lambda L\otimes\adj{A}_\lambda + g_\lambda \adj{L}\otimes A_\lambda \right) \nonumber \\
        \label{eq:nmqsd.Htot_thermal_shifted}
        &+ \sum_\lambda \sqrt{\bar n_\lambda} \, \left( g_\lambda \adj{L}\otimes\adj{B}_\lambda  + \cc g_\lambda L \otimes B_\lambda \right) \\
        &+ \unit \otimes \sum_\lambda \omega_\lambda \left( \adj{A}_\lambda A_\lambda - \adj{B}_\lambda B_\lambda \right). \nonumber
\end{align}
This is identical to the zero-temperature model except for the system coupling to two separate oscillator baths instead of one;
Therefore we need two independent processes $\ZZ_t$ and $\cc{W}_t$ for a stochastic version of \autoref{eq:nmqsd.Htot_thermal_shifted} in general:
\begin{align}
  \partial_t \psi_t = -\ii\Hsys\psi_t &+ L\ZZ_t\psi_t - \adj{L}\int_0^t \alpha_1(t-s) \frac{\delta \psi_t}{\delta \ZZ_s} \dd s \nonumber \\
  &+ \adj{L} \cc{W}_t \psi_t - L\int_0^t \alpha_2(t-s) \frac{\delta\psi_t}{\delta \cc{W}_s} \dd s.
  \label{eq:nmqsd.nmsse_thermal_2processes}
\end{align}
All effects of the original thermal initial state are now encoded in the correlation functions
\begin{equation}
  \alpha_1(t) = \sum_\lambda (\bar n_\lambda + 1) \abs{g_\lambda}^2 \exp[-\ii\omega_\lambda t] \quad \mbox{and} \quad
  \alpha_2(t) = \sum_\lambda \bar n_\lambda \abs{g_\lambda}^2 \exp[\ii\omega_\lambda t]
  \label{eq:nmqsd.cov_2processes}
\end{equation}
for $\ZZ_t$ and $\cc{W}_t$ respectively.
Both are Gaussian, thus independence of $\ZZ$ and $\cc W$ is equivalent to vanishing mutual covariance $\E[\ZZ_t W_s] = \E[Z_t W_s] = 0$.

As we doubled the bath degrees of freedom merely to cope with a thermal initial state, it is quite natural that the zero-temperature result~\ref{eq:nmqsd.nmsee} with a single driving process is recovered in the limit $T \to 0$:
With vanishing occupation numbers $n_\lambda \to 0$ both $\alpha_2$ and $\cc W_t$ go to zero, while $\alpha_1$ reproduces the original bath correlation function~\ref{eq:nmqsd.correlation_function}.

The thermo-field approach is especially simple for a self-adjoint coupling operator $L = \adj{L}$.
Indeed, we can combine both processes in \autoref{eq:nmqsd.nmsse_thermal_2processes} into a single one, which we denote by $\tildeZZ_t$.
Since $\ZZ$ and $\cc W$ are mutual independent by assumption and $2\bar n_\lambda + 1 = \coth{\frac{\beta\omega_\lambda}{2}}$, we find for the crucial correlation function
\begin{equation}
  \E[\tilde Z_t \tildeZZ_s] = \sum_\lambda \left(\abs{g_\lambda}^2 \, \coth{\frac{\beta \omega_\lambda}{2}} \, \cos{\omega_\lambda (t-s)} - \ii \sin{\omega_\lambda (t-s)} \right).
  \label{eq:nmqsd.combined_correlation}
\end{equation}
Consequently the finite temperature NMSSE with self-adjoint coupling operators is identical to the $T=0$ result except for a modified correlation function.
It is not surprising that the combined correlation function~\ref{eq:nmqsd.combined_correlation} agrees with the result of Feynman and Vernon already encountered in \autoref{eq:nmqsd.thermal_correlation_function}, which is usually derived by means of path integration \cite{FeVe63_quantum_dissipative}.
But our approach presented here is much more general since it can tackle any kind of open quantum system with linear coupling.

%%%%%%%%%%%%%%%%%%%%%%%%%%%%%%%%%%%%%%%%%%%%%%%%%%%%%%%%%%%%%%%%%%%%%%%%%%%%%%%%
%\subsection{unitary noise}
%\label{sub:nmqsd.temperature.unitary}
%% * method
%% * would be better for hierarchies
%% * still negative energies
%% * problematic integral
%% * classical noise --> alpha purely real

%As shown in the last section we can treat classical thermal noise on the same footing as quantum noise under certain circumstances just by using a modified correlation function~\ref{eq:nmqsd.combined_correlation}.
%It is worth noticing how the influence of thermal fluctuations modify only the real part of $\alpha$, a feature that explicitly distinguishes noisy classical perturbations \cite{FeHi10_path_integrals}.
%Therefore it is quite instructive to present a different method for treating non-zero temperature within the non-Markovian quantum state diffusion.

%We start off by expanding the thermal bath state in a coherent state basis \cite{WaMi08_quantum_optics}
%\begin{equation*}
  %\rho(\beta) = FILL IN
%\end{equation*}
%which is quite reminiscent of the expansion for a pure state projector that lead to our stochastic Schrödinger equation.
%The corresponding pure initial states are a product involving all environmental oscillators
%\begin{equation*}
  %\ket{\Psi_0(\xi)} = \exp[-\frac{\abs{\vec\xi}^2}{2}] \ket{\psi_0} \bigotimes_\lambda \ket{\xi_\lambda}
%\end{equation*}
%where the additional prefactor is usually absorbed by using normalized coherent states.
%A simple shift for the creation and annihilation operators $\adj{A}_\lambda = \adj{a}_\lambda - \cc{\xi}_\lambda$ and $A_\lambda = a_\lambda - \xi_\lambda$ respectively maps the environmental part of the initial state above onto the vacuum.
%Therefore we can apply our zero-temperature derivation to the total Hamiltonian expressed in $A$ and $\adj{A}_\lambda$.
%The resulting NMSSE reads
%%TODO Fix apperance
%\begin{equation}
  %\partial_t \psi_t(\ZZ, \xi) = \left( -\ii\Hsys + L\cc\xi_t + \adj{L}\xi_t + L\ZZ_t - \adj{L}\int_0^t \alpha(t - s) \frac{\delta}{\delta \ZZ_s} \dd s \right) \psi_t(\ZZ, \xi, \cc\xi)
  %\label{eq:nmqsd.nmsse_thermal_classic}
%\end{equation}
%with a classical driving process $\xi_t = \sum_\lambda g_\lambda \xi_\lambda \exp[-\ii \omega_\lambda t]$ and its familiar quantum counterpart $\ZZ_t$.
%The former's properties are once again fixed by its correlations
%\begin{equation*}
  %\E\,\xi_t = 0, \quad \E\,\xi_t \xi_s =0, \quad\mbox{and}\quad \E\,\xi_t \cc{\xi}_s = 2\sum_\lambda \bar n_\lambda \abs{g_\lambda}^2 \cos{\omega_\lambda(t-s)}.
  %\label{eq:nmqsd.process_properties}
%\end{equation*}
%% TODO Too much zero temperature...
%%Recovering the reduced density matrix not only requires an average over $\ZZ$ but also over all realizations of the thermal noise process $\xi_t$.
%Since all thermal occupation numbers $\bar n_\lambda$ tend to zero for $T \to 0$, we obtain the zero temperature limit simply by setting $\xi_t = 0$.
%This amounts to the trivial decomposition $\rho(T = 0) = \ket{\vec 0}\bra{\vec 0}$ of the zero temperature environmental state.
%% TODO What is correlation? Why no functional derivative?
%% TODO DISCUSSION!

%%%%%%%%%%%%%%%%%%%%%%%%%%%%%%%%%%%%%%%%%%%%%%%%%%%%%%%%%%%%%%%%%%%%%%%%%%%%%%%
\section{Jaynes-Cummings Model}
\label{sec:nmqsd.two_level}
%%%%%%%%%%%%%%%%%%%%%%%%%%%%%%%%%%%%%%%%%%%%%%%%%%%%%%%%%%%%%%%%%%%%%%%%%%%%%%%

% ✔ originaly introduced to model decay of two-level atom coupled to single mode of cavity; comparisson of semi-classical and quantized theory of radiation
% ✔ two level system H=σ_z; coupled to possibly structured environment; dipol approximation
% ✔ far-off resonance for other levels of atom --> effective two level system
% * here general spectral density
%
The Jaynes-Cummings model was originally introduced to study the decay of an atom coupled to a single quantized mode of the electro-magnetic field in a cavity \cite{JaCu63_radiation_theory}.
It describes a single electronic excitation of the atom with energy $\omega$ above ground state in terms of an effective two level system with $\Hsys = \frac{\omega}{2}\sigma_z$.
% FIXME Clear?
Other electronic levels can be neglected safely, if their excitation energy is much larger than $\omega$ or far off-resonance compared to the cavity mode.
% FIXME Really?
We approximate the coupling operator to the cavity-mode within dipole and rotating-wave approximation as $L = g\sigma_-$.
This approximation holds as long as the coupling strength $g$ is very small compared to the cavity transition frequency.
Only recently experiments in circuit quantum electrodynamics detected effects from co called counter-rotating interaction terms \cite{NiDeHu10_circuit_qed}.

Summarized the NMSSE for the Jaynes-Cummings model at zero temperature reads
\begin{equation}
  \partial_t \psi_t = -\ii\frac{\omega}{2}\sigma_z\psi_t + g \sigma_- \ZZ_t \psi_t - g\sigma_+ \int_0^t \alpha(t - s) \frac{\delta\psi_t}{\delta \ZZ_s} \dd s.
  \label{eq:nmqsd.nmsse_twolevel}
\end{equation}
This also covers the more general case of a possibly structured environment.


%%%%%%%%%%%%%%%%%%%%%%%%%%%%%%%%%%%%%%%%%%%%%%%%%%%%%%%%%%%%%%%%%%%%%%%%%%%%%%%
\subsection{O-Operator Method}
\label{sub:nmqsd.two_level.o}
%
% TODO Check for complex c, if all terms are correct

As elaborated in \autoref{sub:nmqsd.lin_nmsse.convolutionless} we can simplify the NMSSE~\ref{eq:nmqsd.nmsse_twolevel} by replacing the functional derivative with an operator $O(t, s, \ZZ)$.
We try to solve the consistency condition~\ref{eq:nmqsd.consistency_condition} by a noise-independent ansatz
\begin{equation}
  O(t, s) = g f(t, s) \sigma_-.
  \label{eq:nmqsd.o_ansatz}
\end{equation}
Hence all non-Markovian feedback from the environment is now encoded in the function $f(t, s)$ to be determined.
Plugging this ansatz into the evolution equation for $O$ yields
\begin{equation}
  \partial_t f(t, s) \sigma_- = \left[-\ii \frac{\omega}{2} \sigma_z - g^2 F(t) \sigma_+\sigma_-, f(t, s) \sigma_-\right]
  \label{eq:nmqsd.eom_for_o}
\end{equation}
with a shorthand notation $F(t) := \int_0^t \alpha(t-s) f(t, s) \dd s$.
From its definition~\ref{eq:nmqsd.o_bar} we see that $F$ is also the coefficient of the integrated operator $\bar O(t) = g F(t) \sigma_-$.
Since the operator algebra in \autoref{eq:nmqsd.eom_for_o} closes, our ansatz solves the equation of motion for $O$ provided $f$ evolves according to
\begin{equation*}
  \partial_t f(t, s) = \left(\ii \omega + g^2 F(t)\right) \, f(t, s), \qquad 0 \le s \le t.
\end{equation*}
Appropriate initial conditions follow trivially from \autoref{eq:nmqsd.o_initial}, namely $f(s, s) = 1$.
In the special case of an exponential bath correlation function $\alpha(t) = \exp[-\gamma\abs{t} - \ii\Omega t]$ we can also derive a differential equation that is closed in $F$, namely
\begin{equation}
  \partial_t F(t) = 1 + (\ii (\omega - \Omega) - \gamma) F(t) + g^2 F(t)^2.
  \label{eq:nmqsd.twolevel_f}
\end{equation}
Correlation functions of this form play a major role in the subsequent work.
Once $F$ is known we can determine solutions of the convolutionless NMSSE for given noise realizations or---since $O$ is independent of $\ZZ$---directly calculate the reduced density operator using a master equation similar to \autoref{eq:nmqsd.master}.

%%%%%%%%%%%%%%%%%%%%%%%%%%%%%%%%%%%%%%%%%%%%%%%%%%%%%%%%%%%%%%%%%%%%%%%%%%%%%%%
\subsection{Noise-Expansion Method}
\label{sub:nmqsd.expansion}

In this section we propose a different approach to \autoref{eq:nmqsd.nmsse_twolevel}:
It is based on the expansion discussed in \autoref{sec:nmqsd.interpretation}, which allows us to express the quantum trajectories $\psitZ$ in a functional Taylor series with respect to the noise process.
\begin{equation*}
  \psitZ = \twovec{\psi^+(t)}{\psi^-(t)} + \int_0^t \twovec{\psi^+_s(t)}{\psi^-_s(t)} \ZZ_s \dd s.
\end{equation*}
Due to the particular coupling structure of the model we can neglect all terms higher than linear order in $\ZZ_t$.
As further elaborated in \autoref{sec:tla.general} our NMSSE~\ref{eq:nmqsd.nmsse_twolevel} reduces to a $\Complex$-valued integro-differential equation
% TODO Does this have anything in common with damped, driven oscillator?
\begin{equation}
  \dot\psi^+(t) = -\ii \frac{\omega}{2} \psi^+(t) - g^2 \int_0^t \alpha(t - s) \exp[\ii \frac{\omega}{2} (t - s)] \psi^+(s) \dd s,
  \label{eq:nmqsd.dotpsi_plus}
\end{equation}
which is nevertheless quite involved---even from a numerical point of view.
Again the situation simplifies dramatically for an exponential correlation function; the details are provided in the appendix as well.

Nevertheless we may still discover some illuminating consequences concerning the $O$-operator without an explicit solution for $\psi^+(t)$.
With $\psi^-(t) = \psi^-(0) \, \exp(\ii \omega t / 2)$ the full quantum trajectory reads
\begin{equation}
  \psi_t(\ZZ) = \twovec{\psi^+(t)}{\psi^-(t)} + g \int_0^t \twovec{0}{\exp[\ii \frac{\omega}{2} (t-s)] \psi^+(s)} \ZZ_s \dd s
  \label{eq:nmqsd.solution}
\end{equation}
% FIXME Does not agree with my O
This allows us to calculate the functional derivative with respect to the driving process explicitly; for $0 \le s \le t$ we find\footnote{%
  For $s = t$ there is no additional coefficient $\frac{1}{2}$ from integrating a $\delta$-function localized at the upper integral boundary as explained in the footnote on page~\pageref{fn:tla.boundaries}.
}
\begin{equation*}
  \frac{\delta \psi_t(\ZZ)}{\delta \ZZ_s} = g \twovec{0}{\exp[\ii \frac{\omega}{2} (t-s)] \psi^+(s)}
\end{equation*}
which agrees with out ansatz~\ref{eq:nmqsd.o_ansatz} in case we choose
\begin{equation}
  f(t, s) = \frac{\psi^+(s)}{\psi^+(t)} \, \exp[\ii \frac{\omega}{2}(t - s)].
  \label{eq:nmqsd.o_ansatz_psi}
\end{equation}
A similar structure for the $O$-operator has been obtained using a Heisenberg-operator technique before \cite{St01_habil}.\\

\begin{figure}
  \centering
  \includegraphics[width=\columnwidth]{img/jaynescummings.pdf}
  % TODO F is actually much worse, since its a jump; maybe I should show that
  \caption{%
    Jaynes-Cummings model with an exponentially decaying bath correlation function $\alpha(t) = \frac{\gamma}{2} \exp[-\gamma\abs{t} - \ii\Omega t]$ in resonance ($\Omega = \omega$) and coupling strength $g = \sqrt{2}$ calculated using \autoref{eq:tla.solution}.
    % FIXME
    Insets show $\vert F \vert$, represents magnitude of $\bar O$.
    \textbf{(A)} Strongly damped $\gamma = 4\omega$,
    \textbf{(B)} Weakly damped $\gamma = 0.1\omega$; the dashed line represent the same parameter set, but slightly moved off-resonance ($\Omega = 1.01\omega$).
  }
  \label{fig:nmqsd.jaynes_cummings}
\end{figure}

% * Discuss \bar O, here non-trivial part is F(t); relevant part in numerical application \cite
% * formula for \psi^+(t) (\sigma_z); sigma_z = -1 <==> psi^+ = 0 (from density operator + Tr ρ = 1)

% FIXME Citation
Recall that the convolutionless NMSSE of the last section reduced to a nonlinear differential equation~\ref{eq:nmqsd.twolevel_f} for
\begin{equation}
  F(t) = \int_0^t \alpha(t - s) \exp[\ii \frac{\omega}{2} (t - s)] \frac{\psi^+(s)}{\psi^+(t)} \dd s,
  \label{eq:nmqsd.twolevel_f_integral}
\end{equation}
where $f(t, s)$ has already been replaced by the expression~\ref{eq:nmqsd.o_ansatz_psi}, derived using the noise expansion.
We will now argue that it is exactly the denominator $\psi^+(t)$ that may cause problems in a numerical integration of~\ref{eq:nmqsd.twolevel_f}:
Indeed, calculating the reduced density matrix from \autoref{eq:nmqsd.solution} and using the condition $\Tr \rho_t = 1$, we find the connection $\qmean{\sigma_z}_{\rho_t} = 2\abs{\psi^+(t)}^2 - 1$.

% * already problematic in this simple model: <σ>_z --> -1 for t --> ∞; but also in between
%     => denominator in eq... problematic around psi^+(t) \approx 0 if
%        * alpha not decayed rapidly enough to cancel nonzero numerator
%        * or ψ^+_s approaches zero at t very steep
%     * figure show this behavior: if damping is large enough (A) no problem due to surpression by alpha
%     * (B) weak damping, decays on timescale tau = 10 => in that timescale below t=5, 15,... f(t, s) accumulates large contributions
%

Since the Jaynes-Cummings model is expected for $\gamma \neq 0$ to relax to the electronic ground state, namely the eigenstate of $\sigma_z$ with eigenvalue $-1$, the denominator in~\ref{eq:nmqsd.twolevel_f_integral} eventually goes to zero.
Even worse are local extremal points $t_i$, where $\qmean{\sigma_z}_{\rho_t} \approx -1$, as we see in \autoref{fig:nmqsd.jaynes_cummings}.
% FIXME Relaxes?
The excitation in A decays almost exponentially due to the small memory time $\tau = \gamma^{-1}$ of the environment.
Therefore $\frac{\psi^+(s)}{\psi^+(t)} \approx 1$ in the relevant time scale $s \in [t - \tau, t]$ of \autoref{eq:nmqsd.twolevel_f_integral}so $\abs{F}$ approaches a constant value uniformly.
In contrast, the highly non-Markovian system B shows singular peaks in $\abs{F}$ for such $t$, where the expectation value of $\sigma_z$ is approximately zero;
the correlation function $\alpha$ does not decay rapidly enough to suppress contributions in the integral with $\psi^+(s) \gg \psi^+(t)$.

% * resonance phenomenon: by small shift in ω, peaks much smaller
%     * cause: its not |ψ^+|, but ψ^+, for ω≠Ω cancelation in integral due to phase
% * since \bar O is only used to propagate \psi^+ ==> there large \bar O cancled by \psi^+(t);
%     => although differences of resonant and none-resonant in F huge, no differences in <σ_z>

% FIXME REALLY?
This resonant behavior is not surprising, since deriving \autoref{eq:nmqsd.twolevel_f} once with respect to time yields a differential equation similar to a damped harmonic oscillator.
% TODO Add details why?
Also, notice how shifting the bath-frequency $\Omega$ only slightly off-resonance results in a much smoother $F$---as the dashed line in \autoref{fig:nmqsd.jaynes_cummings} B shows---while $\qmean{\sigma_z}$ remains virtually unchanged.
% FIXME Clear?
Actually these large peaks do not contribute excessively to end result, as they are canceled with the very small value of $\psi^+(t)$ in $\bar O(t) \psitZ$ (or in the corresponding master equation).
Still the numerical solution of \autoref{eq:nmqsd.twolevel_f} requires a much higher accuracy for the resonant case in order to correctly reproduce these peaks and not diverge to infinity.\\

% * not possible in real systems ==> more modes in bath, ZZ-dependent O
%     => more severe, densely spaced divergences known to appear in certain areas of parameter spaces, accumulate error
%     * exceed numerical accuracy
%     * supposed to be one reason for failure of O-operator calculation \cite private communication
% * better to propagate state to circumvent divergences ==> HiHiHiearchies!

Of course all statements made in this section only hold for the simple Jaynes-Cummings model, and it is not clear, whether the $\bar O$-operator behaves similarly for more realistic systems.
% FIXME Is irregular the best word, also mention that its much worse
Notwithstanding a similar irregular behavior of $\max\limits_{m,n} \abs{\bar O_{mn}(t)}$, where $\bar O_{mn}(t)$ are the matrix elements of $\bar O(t)$ in the system-basis used, has been found in numerical ZOFE-calculations\footnote{%
  %FIXME Add citation
  ZOFE stands for zero order functional expansion---a numerical method based on a functional expansion of $\bar O(t, \ZZ)$ in terms of the noise process \cite{}.
}
% FIXME Add citation "private conversation"
for certain parameter regimes.
% FIXME Really?
As for a larger number of bath modes resonances are more likely to occur, the situation is much more delicate for realistic systems.
We come to the conclusion that it can be advantageous to devise a numerical scheme in terms of the more regular $\bar O(t, \ZZ) \psitZ$ instead, as sharp contributions during the propagation might be smoothed away---this is exactly the approach used for our stochastic hierarchical equations of motion presented in the next chapter.

\chapter{Numerical treatment}
\label{chap:num}
% * numerical treatments necessary
% * main goal: calculate reduced density operator
% * other things: absorption spectra, single trajectories


%%%%%%%%%%%%%%%%%%%%%%%%%%%%%%%%%%%%%%%%%%%%%%%%%%%%%%%%%%%%%%%%%%%%%%%%%%%%%%%
\section{Hierarchical Equations of Motion}
\label{sec:num.heom}
% * Tanimura HEOMs
%%%%%%%%%%%%%%%%%%%%%%%%%%%%%%%%%%%%%%%%%%%%%%%%%%%%%%%%%%%%%%%%%%%%%%%%%%%%%%%


%%%%%%%%%%%%%%%%%%%%%%%%%%%%%%%%%%%%%%%%%%%%%%%%%%%%%%%%%%%%%%%%%%%%%%%%%%%%%%%
\section{Stochastic Hierarchical Equations of Motion}
\label{sec:num.sheom}
% * main idea
%%%%%%%%%%%%%%%%%%%%%%%%%%%%%%%%%%%%%%%%%%%%%%%%%%%%%%%%%%%%%%%%%%%%%%%%%%%%%%%

%%%%%%%%%%%%%%%%%%%%%%%%%%%%%%%%%%%%%%%%%%%%%%%%%%%%%%%%%%%%%%%%%%%%%%%%%%%%%%%
\subsection{Time derivation}
%TODO change title
\label{sub:num.sheom.time_deriv}

%%%%%%%%%%%%%%%%%%%%%%%%%%%%%%%%%%%%%%%%%%%%%%%%%%%%%%%%%%%%%%%%%%%%%%%%%%%%%%%
\subsection{Linear Hierarchy}
\label{sub:num.sheom.lin}
% * derivation
% * terminator

%%%%%%%%%%%%%%%%%%%%%%%%%%%%%%%%%%%%%%%%%%%%%%%%%%%%%%%%%%%%%%%%%%%%%%%%%%%%%%%
\subsection{Nonlinear Hierarchy}
\label{sub:num.sheom.nonlin}
% * derivation


%%%%%%%%%%%%%%%%%%%%%%%%%%%%%%%%%%%%%%%%%%%%%%%%%%%%%%%%%%%%%%%%%%%%%%%%%%%%%%%
\section{Correlation Function Expansion}
\label{sec:num.expansion}
% * pade spectrum decomposition
% * also mention other (Matsubara e.g.)
%%%%%%%%%%%%%%%%%%%%%%%%%%%%%%%%%%%%%%%%%%%%%%%%%%%%%%%%%%%%%%%%%%%%%%%%%%%%%%%


%%%%%%%%%%%%%%%%%%%%%%%%%%%%%%%%%%%%%%%%%%%%%%%%%%%%%%%%%%%%%%%%%%%%%%%%%%%%%%%
\section{Spin-Boson Model}
\label{sec:num.spin_boson}
% * short intro
% * Depth-dependence
% * dependence on number of expansion terms
% * single trajectories
%%%%%%%%%%%%%%%%%%%%%%%%%%%%%%%%%%%%%%%%%%%%%%%%%%%%%%%%%%%%%%%%%%%%%%%%%%%%%%%


%%%%%%%%%%%%%%%%%%%%%%%%%%%%%%%%%%%%%%%%%%%%%%%%%%%%%%%%%%%%%%%%%%%%%%%%%%%%%%%
\section{FMO-Complex}
\label{sec:num.fmo}
% * model
%%%%%%%%%%%%%%%%%%%%%%%%%%%%%%%%%%%%%%%%%%%%%%%%%%%%%%%%%%%%%%%%%%%%%%%%%%%%%%%


%%%%%%%%%%%%%%%%%%%%%%%%%%%%%%%%%%%%%%%%%%%%%%%%%%%%%%%%%%%%%%%%%%%%%%%%%%%%%%%
\subsection{Absorption Spectra}
\label{sub:num.fmo.absorption}
% * derivation of formula
% * why NMSSE so cool for it

%%%%%%%%%%%%%%%%%%%%%%%%%%%%%%%%%%%%%%%%%%%%%%%%%%%%%%%%%%%%%%%%%%%%%%%%%%%%%%%
\subsection{Transfer Dynamics}
\label{sub:num.fmo.dynamics}


\chapter{Application}
\label{chap:app}
% * why?
% * current results/techniques
%%%%%%%%%%%%%%%%%%%%%%%%%%%%%%%%%%%%%%%%%%%%%%%%%%%%%%%%%%%%%%%%%%%%%%%%%%%%%%%%

% * Growing interest, application, ...
% * to show its working apply our method to energy transfer and absorption spectra of molecular aggregat
% * DEF. Molecular aggregat: assemblies of monomers (molecules, atoms, ...), where monomers largely keep individual properties
%     * interaction leads to collevcitve phenomena
%     * for sake of clarity: monomer = molecule in our notation
% * Non-markovian effects
% * demonstrate applicability to systems of recent interest


%%%%%%%%%%%%%%%%%%%%%%%%%%%%%%%%%%%%%%%%%%%%%%%%%%%%%%%%%%%%%%%%%%%%%%%%%%%%%%%%
\section{Basic Model}
\label{sec:app.model}
% * assumptions
% * Frenkel excitons
%
% FIXME Too many subsections?
% TODO Experimental setup, necessary for app.model.exciton
% TODO product initial state ok, vertical Frank-Condon transition
%%%%%%%%%%%%%%%%%%%%%%%%%%%%%%%%%%%%%%%%%%%%%%%%%%%%%%%%%%%%%%%%%%%%%%%%%%%%%%%%

%%%%%%%%%%%%%%%%%%%%%%%%%%%%%%%%%%%%%%%%%%%%%%%%%%%%%%%%%%%%%%%%%%%%%%%%%%%%%%%%
\subsection{The Aggregat Hamiltonian}
\label{sub:app.model.hamiltonian}

% GENERAL MOLECULE
% ---------------
% ✔ molecular Hamiltonian, Born Oppenheimer approx (large difference in mass)
%     => seperation of time scales,
% ✔ split into electronic and vibrational degrees of freedom
% ✔ in aggregat further vibrational degrees of freedom: inter- and intramolecular as well as solvent
%
% FIXME Rename V_mn since we use it below for matrix elements

In the follwing chapter we treat molecular aggregats with a size in the order of magnitude from a few up to a hundred molecules.
Let us consider the latter composed of electrons and point-like nuclei quantum mechanically described by canonical-conjugated pairs of operators $(p_j, q_j)$ and $(P_j, Q_j)$ respectively.
The corresponding Hamiltonian is given by
\begin{equation}
  \opH{mol} = \opT{el} + \opT{nuc} + \opV{el-el} + \opV{nuc-nuc} + \opV{el-nuc}
  \label{eq:app.mol_hamil}
\end{equation}
% TODO More?
with the kinetic energies $T$ and appropriate Coulomb interactions $V$.
% TODO Really?
We drop possible contributions from internal spin degrees of freedom since they induce only negligible corrections for the systems under consideration.

The vast difference in masses of electrons and nuclei allows us to separate the dynamics of both into two individual parts using the Born-Oppenheimer approximation:
As electrons move on a much faster time scale they can respond to any changes in the nuclear arrangement almost instantaneously.
This amounts to including the motion of nuclei mediated by the Coulomb potential $\opV{el-nuc}$ only adiabatically when calculating the electron dynamics from \autoref{eq:app.mol_hamil}.
% TODO Is this clear and to the point?
Therefor we can reorganize the summands in \autoref{eq:app.mol_hamil} more appropriately to
\begin{equation}
  \opH{mol} = \opH{el}(\QQ) + \opT{nuc} + \opV{nuc-nuc},
  \label{eq:app.mol_hamil_bo}
\end{equation}
where the notation $\opH{el} = \opT{el} + \opV{el-el} + \opV{el-nuc}(\QQ)$ indicates that we regard the electronic Hamiltonian to depend only parametrically on the nuclear coordinates $\QQ$.
% FIXME Notice??
For the processes under consideration only the valence electrons need to be taken into account explicitly; others are included to the nucleon-part without further notice.


% FIXME In all possible combinations?
The same reasoning applies to the complete Hamiltonian of the aggregat, which besides contributions of the form~\ref{eq:app.mol_hamil} for each individual molecule contains intermolecular interactions between electrons and nuclei in all possible combinations.
Therefore it can rephrased similarly to \autoref{eq:app.mol_hamil_bo}
\begin{equation}
  \opH{agg} = \opH{el}(\QQ) + \opT{vib} + \opV{vib-vib},
  \label{eq:app.agg_hamil}
\end{equation}
Here we use the more general notion of vibrational degrees of freedom, which not only comprises the intra- and intermolecular nuclear coordinates, but also possible environmental degrees of freedom not belonging to the aggregat.
These appear for example when studying molecular compounds immersed in a liquid solvent.

% ELECTRONIC PART
% ---------------
% * Holstein model
% ✔ no exchange interaction due to seperation of molecules, no overlap (tight binding)
% ✔   => anti-symmetrization in Hartree anatz non necessary; product basis
% ✔      =>   H_el = Σ_ma ε_ma |φ_ma><φ_ma|  +  ½ Σ ...

The Born-Oppenheimer approximation allows us to analyse the electronic separately from the vibrational part of \autoref{eq:app.agg_hamil} for a fixed coordinate vector $\QQ$.
We split up the former into contributions for each individual electron
\begin{equation*}
  \opH{el} = \sum_m H_m^\mathrm{(el)} + \frac{1}{2} \sum_{m,n} U_{mn}^\mathrm{(el-el)},
\end{equation*}
where $H_m^\mathrm{(el)}$ contains the $m$\th electron's kinetic energy as well as its coupling to the vibrational degrees of freedom and $U_{mn}$ is simply the Coulomb interaction between the $m$\th and $n$\th electron.
The \quotes{free} Hamiltonians $H_m^\mathrm{(el)}$ define distinct electronic states by
\begin{equation*}
  H_m^\mathrm{(el)}(\QQ) \varphi_{ma} (q, \QQ) = \epsilon_{ma}(\QQ) \varphi_{ma} (q, \QQ)
\end{equation*}
for each given environmental configuration $\QQ$.
The index $m$ runs over all electrons under consideration and $a$ is used to label the individual states, which we assume to be ordered by the corresponding energies.
Similar to the Hartree-Fock method we build up an expansion basis for the total electronic state by a product ansatz
\begin{equation}
  \phi_{\vec a}(\qq, \QQ) = \prod_m \varphi_{m, a_m}(q_m, \QQ),
  \label{eq:app.product_states}
\end{equation}
which in general needs to be anti-symmetrized to fulfill the Pauli exclusion principle.

If there is at most one valence electron per molecule we need to take into consideration, which is furthermore tightly bound, then the situation simplifies dramatically:
In this case the spreading of the single-electron states $\ket{\varphi_{ma}} = \ket{m, a}$ is small compared to the distance between two molecules; we can neglect the overlap $\braket{m, a}{n, b}$ for different molecules $m \neq n$.
Consequently \autoref{app.product_states} yields a complete basis for the electronic degrees of freedom.
We also have the following representation for the Hamiltonian~\ref{eq:app.agg_hamil}% FIXME Formula!
\begin{equation}
  \opH{el} = \sum_{m, a} \epsilon_{m, a} \, \ket{m, a}\bra{m, a} + \frac{1}{2}\sum_{m,n,a,b,a',b'} U_{mn}(aa', bb') \, \ket{m,a; n,b}\bra{m,a'; n,b'}
  \label{eq:app.agg_hamil_basis}
\end{equation}
with the matrix elements of the Coulomb interaction\footnote{%
  % TODO Mutual enough?
  This does not include the exchange interaction, since we assume a vanishing mutual overlap for the electrons.
}
\begin{equation*}
  U_{mn}(aa', bb') = \bra{m,a; n,b} U_{mn} \ket{m,a'; n,b'}.
\end{equation*}
% TODO Is this correct?
Note that all terms in \autoref{eq_app.agg_hamil_basis} still depend on vibrational coordinates.
For example the matrix elements $U_{mn}(aa'; bb')$ is influenced by the distance between the $m$\th and $n$\th molecule, while the electronic eigenenergies $\epsilon_{m, a}$ primarily depend on the positions of other electrons belonging to the same molecule.

%%%%%%%%%%%%%%%%%%%%%%%%%%%%%%%%%%%%%%%%%%%%%%%%%%%%%%%%%%%%%%%%%%%%%%%%%%%%%%%%
\subsection{The Exciton Model}
\label{sub:app.model.exciton}
% ✔ beside electronic ground state only first excited singlet state φ_m^g, φ_m^e for each molecule
% ✔   => effective 2 level system
%     * ok if only one S_1 state is initially excited and all first-level energies are same order of magnitude
% ✔ different contributions to interaction term; Heitler-London approximation (p.370)
% ✔   => Interaction term gives only "hopping" contributions (resonant excitation energy transfer)
%     => if we start with single excitation, we remain in the single-excitation Hilbert space
% ✔   => basis vectors |π_n> = |φ_n^e> Π_i≠n |φ_i^g> => single exciton state
%     * need ground state |0> = Π_i |φ_i^g> as well due to dissipation
%     * multi-exciton states for nonlinear stuff
%     * interaction matrix elements can be calculated from center-of-mass coordinate of molecule and Coulomb interaction
%        => more details (dipole approximation, etc.) in spectrum-section
%
% TODO Dipol-Dipol interactoin approximation (p.372)

% TODO Good? Position of footnote?
% FIXME I am too long!!!
In order to describe the experimental setting described in the introduction we do not need to consider the complete electronic Hamiltonian~\ref{eq:app.agg_hamil_basis}:
if only a single valence electron is initially in the lowest excited state $S_1$ above its ground state $S_0$\footnote{%
  Note that $S_0$ describes the lowest energy state of the valence electron with all other electrons of the molecule fixed, not to be confused with the atomic ground states.
}
and if the various transition energies are in the same order of magnitude, then it is sufficient to take only the $S_0$ state $\ket{m, 0}$ as well as the first excited stated $\ket{m, 1}$ for each molecule into account.
% FIXME Is charged induced transition a word?
Under these circumstances the matrix elements $U_{mn}(aa', bb')$ can be classified with respect to a few physical processes such as electrostatic interactions or charge-induced transitions.
But most important is the resonant contribution $U_{mn}(01; 10)$ (and its inversion $U_{mn}(10; 01)$) describing a $S_0 \to S_1$ excitation for the $m$\th electron induced by a  $S_0 \to S_1$ transition at the $n$\th molecule.
In the following we neglect all but the last class of processes, which is frequently called Heitler-London approximation.

Restricting the allowed electronic states to the two lowest energy levels has a remarkable interpretation in terms of quasi-particles:
The product
\begin{equation}
  \ket{m} = \ket{m, 1} \prod_{n \neq m} \ket{n, 0}
  \label{eq:app.exciton_state}
\end{equation}
% TODO What about gorund state?
% FIXME Too much due to, therefore,...
describes an excited electron localized in the vicinity of the $m$\th molecule, which we refer to as an exciton of the electronic system.
Due to the Heitler-London approximation our adiabatic Hamiltonian~\ref{eq:app.agg_hamil_basis} conserves the number of excitons.
Therefore an initial state $\ket{m}$ (or linear combinations thereof) always remains in the one-exciton Hilbert space $\HH^{(1)}$.
% TODO WHY???
The interaction matrix elements
\begin{equation*}
  V_{mn} = V_{nm} = \bra{m, 0; n, 1} U_{mn} \ket{m, 1; n, 0}
\end{equation*}
allow us to express the restriction of $\opH{el}$ to $\HH^{(1)}$ as
\begin{equation*}
  \opH{el}^{(1)}(\QQ) = \sum_m \epsilon_m(\QQ) \ket{m}\bra{m} + \sum_{m,n} V_{mn}(\QQ) \ket{n}\bra{m}.
\end{equation*}
% FIXME Second sentence strange...
For the rest of this section we assume the $V_{mn}$ to be independent of vibrational degrees of freedom; a more general treatment poses no further difficulties.\\

% VIBRATIONAL PART
% ----------------
% ✔ include dynamica

% FIXME Too much "degrees of freedom"
Up to this point we have neglected the dynamical evolution of the vibrational environment, which is essential in a complete description of a molecular aggregat.
% TODO REALLY? WHY?
In common settings for the physical systems under consideration a harmonic approximation is sufficient to obtain a realistic model.
There are two reasons for this:
First of all most proteins disintegrate at temperatures much higher than room temperature; therefore thermal excitation only leads to small energy gains for each vibrational degree of freedom.
The other mechanism for driving the environment is dissipation of the electronic system.
But the latter is small compared to the vast number of vibrational degrees of freedom and energy typically spreads evenly across the environment.
We can thusly assume that all $Q_\lambda$ experience only a small displacement from their equilibrium positions, which we set to $Q_\lambda = 0$.

% TODO More?
As a consequence both $\opV{vib-vib}(\QQ)$ and $\epsilon_m(\QQ)$ can be expanded in a Taylor series neglecting all but the first non-trivial term.
To alleviate notation we further assume that each vibrational degree of freedom only couples to one specific exciton.
This leads to exactly the microscopical model presented in \autoref{sec:nmqsd.model}: a bath of harmonic oscillators linearly coupled to the electronic system
\begin{align*}
  \opH{agg} =
  \sum_m \epsilon_m(0) \ket{m}\bra{m} + \sum_{m,n} V_{mn} \ket{m}\bra{n} + \sum_{m, \lambda} \omega_{m, \lambda} \adj{A}_{m, \lambda} A_{m, \lambda} \\
  + \sum_{m, \lambda} g_{m, \lambda} \ket{m}\bra{m} \otimes \left( \adj{A}_{m, \lambda} + A_{m, \lambda} \right),
\end{align*}
where $A_{m, \lambda}/\adj{A}_{m, \lambda}$ are ladder operators corresponding to the $\lambda$\th vibrational mode coupling to the $m$\th exciton.

% TODO ε_m(0) = ε_m ... site energy; optical transition energy at the equilibrium configuration of env. phonons associated with S_0 state
% V_mn electronic coupling strength


%%%%%%%%%%%%%%%%%%%%%%%%%%%%%%%%%%%%%%%%%%%%%%%%%%%%%%%%%%%%%%%%%%%%%%%%%%%%%%%%
\section{Transfer Dynamics }
\label{sec:app.fmo}
%%%%%%%%%%%%%%%%%%%%%%%%%%%%%%%%%%%%%%%%%%%%%%%%%%%%%%%%%%%%%%%%%%%%%%%%%%%%%%%%

%% FMO Structure plots %%%%%%%%%%%%%%%%%%%%%%%%%%%%%%%%%%%%%%%%%%%%%%%%%%%%%%%%%
\begin{figure}[t]
  \centering
  \begin{subfigure}[t]{.6\textwidth}
    \centering
    \includegraphics[width=.8\textwidth]{img/fmo_monomer.png}
    % FIXME Better caption
    % FIXME Higher dpi
    \caption{%
      Spatial arrangement
    }
    \label{fig:app.monomer_full}
  \end{subfigure}
  \begin{subfigure}[t]{.3\textwidth}
    % FIXME Colors
    \centering
    \definecolor{qqttcc}{rgb}{0,0.2,0.8}
    \definecolor{ffqqzz}{rgb}{1,0,0.6}
    \definecolor{ffqqqq}{rgb}{1,0,0}
    \definecolor{qqcccc}{rgb}{0,0.8,0.8}
    \definecolor{ffzzqq}{rgb}{0.6,0.6,0.6}
    \definecolor{qqffqq}{rgb}{0,1,0}
    \begin{tikzpicture}[line cap=round,line join=round,>=triangle 45, xscale=.5, yscale=.014]
      \draw[->,color=black] (0,-55.09) -- (0,409.79);
      \foreach \y in {0, 100, 200, 300}
      \draw[shift={(0,\y)},color=black] (2pt,0pt) -- (-2pt,0pt) node[left] {\footnotesize $\y$};
      \draw[color=black] (-2.30, 180) node[rotate=90] {$\epsilon_m$ [$cm^{-1}$]};
      \clip(-0.38,-55.09) rectangle (5.9,409.79);
      \draw [line width=2pt, color=red] (0.17,180)-- (1.17,180);
      \draw [line width=2pt,color=qqffqq] (1.39,300)-- (2.39,300);
      \draw [line width=2pt,color=ffzzqq] (3.63,400)-- (4.63,400);
      \draw [line width=2pt,color=qqcccc] (2.89,250)-- (3.89,250);
      \draw [line width=2pt,color=orange] (3.85,210)-- (4.85,210);
      \draw [line width=2pt,color=ffqqzz] (2.57,80)-- (3.57,80);
      \draw [line width=2pt,color=qqttcc] (1.81,0)-- (2.81,0);
      \draw [line width=1pt] (0.74,189.05) -- (1.8,284.18);
      \draw [line width=1pt] (4.13,384.05) -- (3.43,259.34);
      \draw [line width=1pt] (4.13,384.05) -- (4.32,221.85);
      \draw [line width=1pt] (3.4,233.65) -- (3.04,87.89);
      \draw [line width=1pt] (3.03,66.68) -- (2.39,9.1);
      \draw [line width=1pt] (4.32,193.38) -- (3.21,87.14);
      \draw [line width=1pt] (1.95,283.23) -- (2.25,9.1);
      \begin{scriptsize}
        \draw[color=red] (0.68,168) node {180};
        \draw[color=qqffqq] (1.90,288.0) node {300};
        \draw[color=ffzzqq] (4.15,388.0) node {400};
        \draw[color=qqcccc] (3.4,238.0) node {250};
        \draw[color=orange] (4.4,198.0) node {210};
        \draw[color=ffqqzz] (3.05,68.0) node {80};
        \draw[color=qqttcc] (2.3,-12.0) node {0};
      \end{scriptsize}
    \end{tikzpicture}
  % FIXME LABEL AND CAPTION
    \caption{%
      Site energies
    }
    \label{fig:app.site_energies}
  \end{subfigure}
  \caption{%
    Spatial and energetic structure of the simplified FMO-monomer with seven BChls; the coloring and numbering is used throughout this section.
    \textbf{(a)} Was created using the molecular visualization system \textsc{PyMOL} based on the \textsc{PDB} entry \textsc{3eni} \cite{pymol,TrCaBl09_fmo_structure}.
    \textbf{(b)} Shows the site energies of \emph{Chlorobaculum tepidum} \cite{AdRe06_fmo}; lines indicate dominant couplings leading to the two distinct transport channels.
  }
\end{figure}
%%%%%%%%%%%%%%%%%%%%%%%%%%%%%%%%%%%%%%%%%%%%%%%%%%%%%%%%%%%%%%%%%%%%%%%%%%%%%%%%

% ✔ why FMO?
% ✔   * small size: typical model system photosyntetic exciton energy transfer
% ✔   => function: transfer electronic excitation energy from the chlorosome (light harvesting antenna) to the photosyntetic reaction center in green sulfur bacteria
% ✔   * 90s: electronic quantum coherence observed; only recently realized: key feature in nearly unit yield transport
%     * role of coherence: avoid local energetic traps; aid efficient trapping of electronic energy by the pigments facing the reaction center \cite{IsFl09_fmo}
%        => exciton superposition states (formed during fast excitation event) allow the excitation to "reversibly sample all posible paths"
%        => efficient directing the energy transfer to find the most effective sink for the excitation energy \cite{EnCaRe07_photosyn}
% ✔      => efficiency beyond classical sampling-by-hopping
% ✔   * before that: semiclass. hopping (Förster theory)

As a first exemplary application of our hierarchical equations of motion we study energy transfer in the Fenna-Matthews-Olson (FMO) complex found in low-light adapted green sulfur bacteria.
This protein complex plays a crucial role in connecting the light harvesting antenna (chlorosome) to the photosyntetic reaction center, where the absorbed solar energy is converted to a charge gradient.
% FIXME Quantumness???
The vast amount of literature on the subject is not only by virtue of its relatively small size---making it an ideal model for numerical investigation---but particularly due to the strong influence of quantum mechanical effects on the transfer, even at physiological temperature.
Only lately Engel et al.\ as well as Ishizaki-Fleming demonstrated that the FMO complex achieves its remarkable almost-unit efficiency by coherent exciton motion instead of classical hopping described by Förster theory \cite{EnCaRe07_fmo,IsFl09_fmo}. \\



% ✔ structure: 3 identical subunits, called monomers
% ✔ these consist of eight BChl molecules
% ✔   * here we focus on one monomer (as shown in \cite{RiRoSt11_fmo_trimer} reasonable approximation due to small coupling between monomers
%     * energy transfer between monomers via resonacne Coulomb interaction (weak!)
% ✔ site energies and electronic coupling depend on protein environment, different values in literature;
%     ✔ here: Chlorobaculum tepidum, from \cite{AdRe06_fmo}A
%     ✔ data was obtained how???
% ✔ no detailed info about specral density for FMO complex
%     ✔ assume independent, but equivalent for each BChl, data from \cite{IsFl09_fmo}
%     * detailed study in \cite{RiRoSt11_fmo}

The FMO protein complex is subdivided into three identical monomers, each comprising eight bacteriochlorophyll pigments (BChls).
% FIXME +3
In contrast to the first seven BChls, the eighth was only discovered in recent years due to its rather weak coupling to the remaining BChls and instability during the isolation procedure in experiments \cite{TrCaBl09_fmo_structure,ScMuEl10_eighth}.
As the main goal here is to show the applicability and reliability of our hierarchical equations of motion we ignore BChl number eight in what follows---this simplified model has been investigated thoroughly with a vast array of methods.
For the same reason, we also restrict our attention to an individual monomer: as shown by Ritschel et al.\ such a limitation is reasonable for the short time scales under consideration as the intermonomeric interaction strength is rather weak.

In \autoref{fig:app.monomer_full} we display the spatial structure and numbering of the BChls in single a monomer.
The BChls 1 and 6 are situated in the vicinity of the light harvesting antenna and receive excitation energy captured, while BChl 3 acts as energy sink.
As both site energies $\epsilon_n$ and electronic coupling strengths $V_{mn}$ depend on the protein environment, different values for different species can be found in the literature.
%> FIXME Where to put electronic coupling
Here we use the data obtained from optical spectroscopy in \emph{Chlorobaculum tepidum} \cite{AdRe06_fmo}, see \autorefs{fig:app.site_energies} and~\ref{fig:app.fmo_ishfl}.
Although the spectral density may be important for the details of the excitation transfer, no comprehensive information on this matter is available at present.
Good agreement with experiments on \emph{Prosthecochloris aesturaii} was achieved \cite{ReScEn08_fmo_spectral_density} under the semi-empirical assumption that each exciton couples to an independent environment with a Drude spectral density
\begin{equation}
  J(\omega) = \frac{2 \lambda}{\pi} \frac{\gamma\omega}{\omega^2 + \gamma^2},
  \label{eq:app.drude}
\end{equation}
with reorganization energy $\lambda = 35\,\mathrm{cm^{-1}}$ and relaxation time $\gamma^{-1} = 50\,\mathrm{fs}$.\\



%% Single Page with Transfer @77K %%%%%%%%%%%%%%%%%%%%%%%%%%%%%%%%%%%%%%%%%%%%%%
\begin{figure}[p]
  \centering
  \includegraphics{img/fmo_ishfl}
  % FIXME Reference to color encoding?
  \caption{%
    \label{fig:app.fmo_ishfl}
    Exciton transfer of the simplified FMO-monomer with seven BChls at 77\,K using our stochastic hierarchical equations up to first~(doted) and second order~(dashed) averaged over 10000 trajectories.
    For comparison the solid line shows the results of Ishizaki and Fleming~\cite{IsFl09_fmo}, which were obtained in the HEOM approach.
    Population for BChls 1--4 \textbf{(A)} and BChls 3--6 \textbf{(B)} with initial excitation on site 1 and 6 respectively.
    The inset displays a purely electronic system without coupling to the vibrational degrees of freedom.
    Details on parameters can be found in \autoref{sec:coth.fmo}.
  }

  \vspace{.3cm}
  \centering
  % FIXME Run again, timescale is not right
  \begin{subfigure}[b]{0.3\textwidth}
    \includegraphics[width=\textwidth]{img/fmo_transfer_0.png}
    \caption{%
      $t = 0.00 \, \mathrm{ps}$
    }
  \end{subfigure}
  \begin{subfigure}[b]{0.3\textwidth}
    \includegraphics[width=\textwidth]{img/fmo_transfer_1.png}
    \caption{%
      $t = 0.2 \, \mathrm{ps}$
    }
  \end{subfigure}
  \begin{subfigure}[b]{0.3\textwidth}
    \includegraphics[width=\textwidth]{img/fmo_transfer_2.png}
    \caption{%
      $t = 0.50 \, \mathrm{ps}$
    }
  \end{subfigure}

  \begin{subfigure}[b]{0.3\textwidth}
    \includegraphics[width=\textwidth]{img/fmo_transfer_3.png}
    \caption{%
      $t = 1.00 \, \mathrm{ps}$
    }
  \end{subfigure}
  \begin{subfigure}[b]{0.3\textwidth}
    \includegraphics[width=\textwidth]{img/fmo_transfer_4.png}
    \caption{%
      $t = 2.00 \, \mathrm{ps}$
    }
  \end{subfigure}

  \caption{%
    \label{fig:app.fmo_transport_pretty}
    Same as \autoref{fig:app.fmo_ishfl} B.
    The intensity of each molecule represents the population of the associated exciton state.
    For the sake of clarity we do not show the full molecular structure.
  }
\end{figure}
%%%%%%%%%%%%%%%%%%%%%%%%%%%%%%%%%%%%%%%%%%%%%%%%%%%%%%%%%%%%%%%%%%%%%%%%%%%%%%%%

\Autoref{fig:app.fmo_ishfl} shows the results of our calculations at cryogenic temperature $T=77\,\mathrm{K}$ using our stochastic hierarchy as well as the established \HEOM-results of Ishizaki and Fleming \cite{IsFl09_fmo}.
Both initial excitations move remarkably fast and directed---and up to $t=700\,\mathrm{fs}$ in a quantum-coherent, wavelike fashion---toward the energy sink at BChl 3.
However, the final population of the latter is only about half as large on the left hand side due to the relatively high site energy of BChl 2.
This prolongs the lifetime of an exciton-state on BChl 1 significantly.
It has been conjecture that this barrier is partly responsible for the high yield of the FMO-complex \cite{IsFl09_fmo}.
Indeed, the relatively small energy gap $\Delta\epsilon = 180\, \mathrm{cm^{-1}}$ between the first and third BChl cannot prevent back-transfer due to thermal excitation from the latter along this channel.
The larger gap of $\Delta\epsilon = 300\,\mathrm{cm^{-1}}$ between the second and the third suppresses depopulation much more efficiently.
This is exactly where quantum effects influence the operation significantly:
Delocalization helps to overcome the resulting subsidiary energetic minima at BChl 1 much quicker than a classical hopping-excitation could.

No such initial \quotes{energy barrier} exists for the transport starting BChl 6, therefore, the excitation-population at molecules 3 and 4 increases up to $t = 1\,\mathrm{ps}$.
For even longer times, all other sites have only vanishing probability left as shown in \autoref{fig:app.fmo_transport_pretty}.

In the inset to \autoref{fig:app.fmo_ishfl} we also show the dynamics for a purely electronic system:
As expected the population-probability shows a purely oscillatory behavior with no effective excitation transport, thus emphasizing the importance of vibrational degrees of freedom in the exciton energy transfer.

Remarkably, the results of first and second order in our stochastic hierarchy are almost indistinguishable from each other and agree very well the reference.
% FIXME Is this a good idea?
Calculations including one more order (not shown) verify, that the second order is already enough to obtain convergence in these parameter regimes.
% FIXME Really?
As we show in \autoref{sec:coth.fmo}, the dominant exponential bath mode is given by\footnote{%
  Recall our notation $\alpha(t) = g \exp[-\gamma\abs{t}]$.
}
$g \approx (2916 - \ii 3716) \, \mathrm{cm^{-2}}$ and $\gamma \approx 106\,\mathrm{cm^{-1}}$.
Therefore the proposed truncation condition $\sqrt{g} \ll k \gamma$, which for the given parameters reads $83 \ll 106 k$, might be too restricting, yet.

That we do not obtain complete agreement in \autoref{app.fmo_ishfl} A has the following reason:
The parameter $g$ of the second bath mode (a low temperature correction term) remains real.
% FIXME Due to Drude; only one Pole, not discontinous at zero!
Therefore, the complete bath correlation function, which is simply the sum of both modes, obtains a non-vanishing imaginary part at $\alpha(0)$ due to truncating the infinite Matsubara expansion.
But since $\alpha$ is related to our driving processes $\ZZ$ by $\alpha(t-s) = \E[Z_t \ZZ_s]$, we cannot treat such an unphysical bath correlation function exactly.
% FIXME Picture?
Therefore we included an additional, almost-Markovian mode with purely imaginary $g$, such that $\Im\alpha(t)$ goes smoothly to zero as $t \to 0$ while changing $\alpha$ as little as possible---the details may be found in the appendix.\\



%% MevsPhi plot %%%%%%%%%%%%%%%%%%%%%%%%%%%%%%%%%%%%%%%%%%%%%%%%%%%%%%%%%%%%%%%%
\begin{figure}[p]
  \centering
  \includegraphics[width=\columnwidth]{img/fmo_mevsphi}
  \caption{%
    Exciton transfer of the simplified FMO-monomer at 300\,K and internal convergence check of the hierarchies. Solid lines are results from \cite{IsFl09_fmo}.\\
    \textbf{(A)} Stochastic hierarchy with 10000 realizations and orders 1--3.\\
    \textbf{(B)} Same sets of parameters, but calculated in the \HEOM formalism with truncation at orders 1, 2 and 5 using \textsc{PHI} \cite{StSc12_heom}.\\
    % FIXME Dashed lines same but 50000 realizations
    \textbf{(C)} The deviation of the stochastic hierarchy with respect to its third order result. Dotted lines indicate the maximum value over time. The starred curves correspond to a larger sample size of 50000 realizations.\\
    \textbf{(D)} Same as (C), but for the \HEOM calculation. Here, the reference is the fifth order result.
  }
  \label{fig:app.fmo_mevsphi}
\end{figure}
%%%%%%%%%%%%%%%%%%%%%%%%%%%%%%%%%%%%%%%%%%%%%%%%%%%%%%%%%%%%%%%%%%%%%%%%%%%%%%%%

The exciton dynamics at physiological temperature in \autoref{fig:app.fmo_mevsphi} shows a similar qualitative behavior.
However, the transfer is less efficient and directed as decoherence leads to a much stronger smearing of the excitation over all BChls, even the ones not shown in this picture.
For the same reason, the wave-like motion only lasts up to $t \approx 400\,\mathrm{fs}$.
Notwithstanding, at $t = 1\,\mathrm{ps}$ the population of the relevant, third BChl is $0.2$---still about two-thirds of the result at cryogenic temperature---and increases further.

This time, our stochastic hierarchical equations of motion reproduce the results of Ishizaki-Fleming almost exactly even with just one order.
In contrast to the $T = 77\,\mathrm{K}$ result, there is no pronounced offset, because the imaginary part of $\alpha(0)$ is insignificantly small
% FIXME Fill in alpha
\begin{equation*}
  \alpha(t) = \dots.
\end{equation*}
Again, we postpone the calculations to the appendix.
Nevertheless, the differences between first and second order are larger than for the former calculation.
% FIXME REally?
% FIXME Reference
Since the only distinction in the dominant bath mode is a larger coupling constant $g$, this strengthens our truncation condition~\ref{eq:}.

On the right hand side of \autoref{fig:app.fmo_mevsphi} B we carry out the same calculations using the \HEOM formalism \cite{StSc12_heom}.
Clearly, convergence with the number of hierarchies is much worse:
The first order calculations displays highly undamped oscillations and the second order is necessary to get the qualitative picture right.
We have found that a truncation at fifth order is necessary to correctly reproduce the coherent oscillations between 0\,fs and 300\,fs.

To check internal convergence of each method systematically, we proceed as follows:
First we define a measure for deviation of a given reduced density matrix $\rho(t)$, obtained from a numerical calculation, with respect to some reference $\rho^{\mathrm{ref}}(t)$ as
\begin{equation}
  A[\rho(t), \rho^\mathrm{ref}(t)] = \max\limits_n \left\vert \rho_{nn}(t) - \rho^{\mathrm{ref}}_{nn}(t) \right\vert.
  \autoref{eq:app.accuracy}
\end{equation}
since we confine our discussion to the populations of the exciton states $n$ given by $\rho_{nn}$.
As we are only interested in the convergence of a given method, the reference state is calculated using the same method and selected by increasing the truncation order $D$ until $A[\rho_{D}(t), \rho{D+1}(t)] < 10^{-3}$; then $\rho^{(ref)} = \rho_D$.
This amounts to $D=3$ for our \NMSSE-hierarchy and $D=5$ for $\HEOM$.
The accuracy of lower-order calculations with respect to the chosen reference is shown in \autoref{fig:app.fmo_mevsphi} C and D for the stochastic and the density hierarchy, respectively.

We already mentioned in the general discussion above that the first order's deviation of about 2\% is just barely visible in \autoref{fig:app.fmo_mevsphi} A.
In contrast, the same accuracy is obtained in the \HEOM-formalism not until we truncate the hierarchy at third order.
However, we notice an important difference in the two plots:
While the \HEOM-calculations show the largest discrepancy at the initial wave-like motion and then drop off by about an order of magnitude, the results from the stochastic hierarchy remain more or less constant after $t = 0.2\,\mathrm{ps}$.
This as well as the tremulous time-dependence indicate, that the error of the latter is due to stochastic effects and not caused by systematic deviation.
We check this statement by repeating the same calculations using a larger sample size---a start marks the corresponding results in \autoref{fig:app.fmo_mevsphi}.
Clearly, they show less deviation thus confirming our assertion.\\



In conclusion, the discussion above shows, that the hierarchical equations based on the nonlinear \NMSSE provide a highly efficient method to calculate exciton energy transfer dynamics in the FMO-complex.
We obtain almost perfect agreement for both, 77\,K and 300\,K, with the established results of Ishizaki and Fleming in first order calculations.
The discrepancy of these to higher order calculations is less than 1\% demonstrating very rapid convergence with respect to the truncation order.
Of course, such an accuracy is completely unnecessary for most investigations of biological systems.
The theoretical model itself is usually much less reliable due to approximations and experimental errors for the parameters involved.
Also, the improved numerical efficiency, due to a reduced number of auxiliary states\footnote{%
  In the case of a single bath mode, we have eight auxiliary states for $D=1$ compared to 330 for $D=4$.
}
and propagating Hilbert space vectors instead of density matrices, is more than compensated by the large sample size required.
However, the stochastic hierarchies' advantages really come to fruition when it comes to more realistic systems.
% FIXME Can we do this
For example, data from fluorescence line-narrowing measurements on \emph{Prosthecochloris aesturaii} indicate that realistic spectral densities are far more structured, requiring as much as 25 exponential modes
This amounts to 176 auxiliary states for $D=1$ compared to a tremendous number of 41 million states required by a fourth-order calculation.

%%%%%%%%%%%%%%%%%%%%%%%%%%%%%%%%%%%%%%%%%%%%%%%%%%%%%%%%%%%%%%%%%%%%%%%%%%%%%%%%
\section{Absorption Spectra}
\label{sec:app.spectra}
% * experimental setup
% * why not so good, but we still use them -> 2D spectroscopy
%%%%%%%%%%%%%%%%%%%%%%%%%%%%%%%%%%%%%%%%%%%%%%%%%%%%%%%%%%%%%%%%%%%%%%%%%%%%%%%%

%%%%%%%%%%%%%%%%%%%%%%%%%%%%%%%%%%%%%%%%%%%%%%%%%%%%%%%%%%%%%%%%%%%%%%%%%%%%%%%%
\subsection{NMSSE for Spectra}
\label{sub:app.spectra.nmsse}
% * why nmsse so good for this?

%%%%%%%%%%%%%%%%%%%%%%%%%%%%%%%%%%%%%%%%%%%%%%%%%%%%%%%%%%%%%%%%%%%%%%%%%%%%%%%%
\subsection{Results}
\label{sub:app.spectra.results}
% * other techniques
% * cool behavior of hierarchies

\chapter{Conclusions and Outlook}
\label{cha:conclusions}

% general non-Markovian systems
%  * analytical and numerical difficult due to memory
% this work: devised powerful method based on NMSSE
%  * in terms of quantum trajectories
%  * extends the benefit of Monte Carlo methods & quantum trajectories to non-Markovian systems
%  * recover density matrix as Monte Carlo average

Starting with the goal to investigate large open quantum systems coupled to structured environments, we devised a hierarchy of quantum trajectories in this work.
Since it extends the benefits of Monte-Carlo methods from the Markovian regime to the more general case of a non-Markovian environment, the newly-devised approach is highly suited to attack problems of current research.\\

% recalled underlying theory of NMSSE in section 2
%  * functional derivative
% short discussion of established results based on Jaynes-Cummings model
%  * using expansion in noise connect O-operator to observable <σ_z>
%  * for certain parameters found resonant behavior also present in more realistic systems
%  * since peaks of O are canceled in time evolution of state --> not necessary
%  * idea to directly attack functoin deriv

With the \NMSSE constituting the foundation of this work, we recalled its derivation based on a microscopic model for the system and its environment in Chapter~2.
Subsequently, the most challenging characteristic of the \NMSSE, namely the appearance of a functional derivative with respect to the noise involving the complete history of the system, was discussed based on the Jaynes-Cummings model.
Beside the established $O$-operator substitution, we introduced a new direct approach that makes use of a functional Taylor series of the quantum trajectory with respect to the noise.
With the help of the latter, we were able to explain the resonant peaks showing up in the magnitude of the $O$-operator for certain sets of parameters which are likely responsible for divergences in a numerical propagation.
The observation that these large contributions disappear in the \NMSSE due to cancellation with small components of the quantum trajectory has convinced us to attack the problem of the functional derivative directly.

% derived the NMSSE-hierarchy for an exponential bath correlation function
% linear and nonlinear version; effectivity of latter demonstrate by means of Spin-Boson
% also: effectivity of terminator and discussion of truncation
% for application: expansion in necessary form of bcf

For that matter, we derived a hierarchy of linear stochastic Schrödinger equations in Chapter 3 by absorbing the memory integral with the derivatives into auxiliary pure states.
The corresponding nonlinear hierarchy constitutes an important result of this work, since it dramatically reduces the sample size required for the Monte-Carlo average.
We demonstrated this statement with the help of the Spin-Boson model, which was also used to demonstrate the influence of the truncation order on the accuracy of the results.

% Applied new method to quantum aggregates, which can be described in terms of open quantum sytems
% First: FMO complex in light harvesting systems
%  * beside general advantages of Monte-Carlo methods compared to master equations:
%  * able to reproduce all established results with first order calculations
%  * since computational effort depends crucially on number of auxiliary states, very important for efficiency
% Spectra
%  * no averaging necessary => NMSSE comes into its own
%  * combinded with very systematic construction with increasing number of orders
%  * experimental relevant information obtained easily

The main part of this work is the investigation of exciton energy transfer in light-harvesting complexes in Chapter 4.
With quantum effects responsible for its remarkable efficiency, the \textsc{FMO}-protein complex constitutes a great example of how quantum mechanics influences the operation of living cells.
Our calculations from \autoref{sec:app.fmo} confirm the results from prior work that even at physiological temperature the coherent, wave-like motion of excitons is crucial for the operation of the \textsc{FMO}-complex.
As we demonstrated, the \NMSSE-hierarchy is capable of reproducing the established results obtained in the \HEOM-formalism.
Even more, our newly-devised approach achieves the same precision as the established density matrix hierarchy consuming only a fraction of the computational resources.
Although the Monte-Carlo evaluation requires the propagation of many trajectories, the \NMSSE-hierarchy benefits from the reduced number of auxiliary states due to the smaller truncation order necessary.
Besides, the calculation of independent realizations is easily distributed to many computers.

The second application of the pure state hierarchy presented in this work is the study of optical absorption spectra of molecular aggregates.
Since the latter is reduced to the calculation of a single pure state trajectory, no Monte-Carlo average is necessary.
Also, we have demonstrated that the low-energy part of the absorption spectrum can be obtained with less computational expense than the full spectrum.
This makes the \NMSSE highly interesting for the study of systems that are currently out of reach from established methods as the lowest energy peaks are often sufficient to assess the structure of the complex roughly.\\

% OUTLOOK
% test method for even larger systems, find boundaries of what s possible
% relativion between density operator hierarchy; maybe unravelling?
% not just averaging --> more information (entanglement)
% treat entangled initial state by adjusting initial value (complicated expansion in noise)

With the reliability and efficiency of the \NMSSE-hierarchy demonstrated in this work, future investigation will have to attack new problems.
For example, we have restricted the investigation in \autoref{sec:app.fmo} to very simple spectral densities.
The treatment of a more realistic environment will clearly benefit from the improved convergence of the \NMSSE-hierarchy with respect to the truncation order.
Therefore, the question, how details of the spectral density affect the energy transfer, constitutes a suitable subject of future work.
This problem has already been treated in the \textsc{ZOFE}-approach \cite{RiRoSt11_fmo}, however, it is not sure if the latter is still valid for the parameters under consideration.
Since we can check the accuracy of a calculation simply by comparing with higher-order results, the \NMSSE-hierarchy may shed some light into this question.

A further point, which has not been treated in this work at all, is that quantum trajectories contain more information than the reduced density operator.
Indeed, since the \NMSSE is equivalent to the microscopic Schrödinger equation introduced in \autoref{sec:nmqsd.model}, the set of all quantum trajectories contains---at least in principle---all information on the environment and the system-bath entanglement.
It is a fascinating question, if it is possible to gain knowledge about the full state from the finite set of trajectories calculated in a Monte-Carlo simulation.

Finally, future work should investigate how a large class of initial conditions can be treated in the \NMSSE-formalism.
Throughout this thesis, we have only considered a product state with the environment at thermal equilibrium.
Especially the treatment of initially entangled states with the \NMSSE remains an open problem.



\appendix
%%%%%%%%%%%%%%%%%%%%%%%%%%%%%%%%%%%%%%%%%%%%%%%%%%%%%%%%%%%%%%%%%%%%%%%%%%%%%%%
%%%%%%%%%%%%%%%%%%%%%%%%%%%%%%%%%%%%%%%%%%%%%%%%%%%%%%%%%%%%%%%%%%%%%%%%%%%%%%%
\chapter{Mathematical Preliminaries}
\label{ch:math}

We will now substantiate the claim of \autoref{sub:nmqsd.interpretation.unitary_view}, that the linear Non-Markovian Stochastic Schrödinger Equation
\begin{equation}
  \partial \psi_t = -\ii h \psi_t + L\ZZ_t \psi_t - \adj{L}\int_0^t \alpha(t-s) \frac{\delta\psi_t}{\delta\ZZ_s} \dd s
  \label{eq:math.nmsse}
\end{equation}
can also be understood as a Schrödinger equation for the closed system, consisting of the system and a generalized environment.
Our investigation will proceed as follows: First we will be concerned with the kinematic structure and provide an explicit construction of the underlying bath Hilbert space;
subsequently we will explore how the noise process $\ZZ_t$ and the functional derivative in \autoref{eq:math.nmsse} can be realized as operators on this Hilbert space.

In this section we will not attempt to present a mathematical rigorous treatment of the NMSSE, but provide the basic idea, how \autoref{eq:math.nmsse} fits into the established framework of \emph{White Noise Analysis}.
Therein we will follow the spirit of \cite{Hi80_brownian_motion,HiKuPoSt93_white_noise}.

%%%%%%%%%%%%%%%%%%%%%%%%%%%%%%%%%%%%%%%%%%%%%%%%%%%%%%%%%%%%%%%%%%%%%%%%%%%%%%%
\section{Hilbert Space of \quotes{Time Oscillators}}
\label{sec:math.hilbert_space}
%%%%%%%%%%%%%%%%%%%%%%%%%%%%%%%%%%%%%%%%%%%%%%%%%%%%%%%%%%%%%%%%%%%%%%%%%%%%%%%

%TODO besser erklären, wie verschiede versionen zustande kommen
%TODO reference
%TODO ZZ_t explizit nochmal reinschreiben?
%TODO Abkürzungen? NMSSE? BCF?
Let us first recall some basic terminology from probability theory (see e.g.~\cite{Sc05_mims}):
A $C$-valued \idef{random variable} $X$ is a measurable map from a measure space $(\Omega, \mathcal{A}, \PM)$ to the measurable space $(C, \mathcal{B})$.
Here $\Omega$, $C$ denote sets, $\mathcal{A}$, $\mathcal{B}$ are $\sigma$-algebras of $\Omega$ and $C$ respectively, and $\PM$ is a probability measure on $(\Omega, \mathcal{A})$.\footnote{In what follows, we will consider only Borel-$\sigma$-algebras and therefore not mention them any further.}
Expectation values, variances, etc., may then be expressed as integrals of $X$ with respect to $\PM$.

%TODO Satz fertig
It is important to mention, that --- versions\dots
Therefore we may always use the microscopical version of $\ZZ_t$ defined in \autoref{eq:???}, where $\Omega=\Complex^N$ and $\PM=\exp[-\abs\zz^2]\dd^Nz$.
But since this is---strictly speaking---valid only for a finite number $N$ of bath oscillators, we will take a different route:
Using only the \emph{bath correlation function}
\begin{equation}
  \alpha(t) = \int_\Reals \exp[-\ii\omega t] \, J(\omega)\dd\omega,
  \label{eq:math.def_alpha}
\end{equation}
we establish a suitable measure space, which will be used to support our new environmental Hilbert space afterwards.
Here $J(\omega)\dd\omega$ denotes the \idef{spectral density}, which is a positive, bounded measure, i.e.~$J\ge0$ and $\int J(\omega)\dd\omega < \infty$.
This restriction excludes the important Markovian limit $J=1 \iff \alpha(t)\propto\delta(t)$, but\dots
Such a unified approach has the advantage of only making reference to the bath correlation function, stressing that it is the only property of the environment relevant within our model.\\


%TODO interpretation von Ω?
%TODO Write down conventions for Fourier Transform.
As a first step, we will fix our space $\Omega$, which will be used as domain for our random variables later.
Its physical interpretation will be more clearly after we have developed our formalism.
In what follows the \idef{Schwartz space} $\SchwartzS$ of real valued, infinitely often differtiable functions on $\Reals$ with rapid decrease will play an important role---see \cite[184-188]{Ru91_functional_analysis} for its definition and properties.
The reason for its importance is found in the following theorem.
\begin{thm}[Minlos's Theorem, {{ \cite[Thm.~1.1]{HiKuPoSt93_white_noise }}}]
  \label{thm:math.minlos}
  Let $\CHF$ be a characteristic functional on $\SchwartzS$, i.e. $\CHF\colon \SchwartzS \rightarrow \Reals$ with the properties
  \begin{enumerate}[i)]
    \item $\CHF$ is continous on $\mathcal{S}$,
    \item $\CHF$ is positive definite,
    \item $\CHF(0) = 1$.
  \end{enumerate}
  Then there exists a unique probability measure $\PM$ on $\dual{\SchwartzS}$ (the dual space of $\SchwartzS$), such that for all $f \in\SchwartzS$
  \begin{equation}
    \int_{\dual{\SchwartzS}} \exp[\ii \, \dualp{\xi}{f}] \dd\mu(\xi) = \CHF(f).
    \label{eq:math.fourier_minlos} 
  \end{equation}
\end{thm}

Here $\dualp{\xi}{f}$ denotes the dual pairing of $\SchwartzS/\dual{\SchwartzS}$, which can formally be written as $\dualp{\xi}{f} = \int_\Reals \xi(t) f(t) \, \dd t$.
We also recall that a function $f$ is called \idef{positive definite}, if\dots
%TODO Insert definition POSTIVE DEFINITE
In \cite{HiKuPoSt93_white_noise} they take advantage of this theorem to construct a real valued White Noise processes on $\dual\SchwartzS$ with the choice $\CHF(f) = \exp(-\int_\Reals f(t)^2 \dd t)$.
This can be rephrased as $\CHF(f)=\exp(-\,\varf[f])$ with the \emph{variance functional} $\varf[f]=\int f(t)^2 \dd t$.


\begin{lem}[{{Fourier transform, \cite[Thm.~7.7]{Ru91_functional_analysis}}}]
  Let $\SchwartzSC=\SchwartzS+\ii\SchwartzS$ denote the complexified space of test-functions. The Fourier transform is a continuous, linear one-to-one mapping of $\SchwartzSC$ onto $\SchwartzSC$.
\end{lem}
But since we are interested in complex processes with memory, we have to generalize these results.
The reformulation at the end of the last paragraph points us into the right direction; first of all we define an appropriate covariance functional by
\begin{equation}
  \varf[f] = \int_{\Reals^2} \alpha(t-s) f(t) f(s) \dd t \dd s \quad (f \in\SchwartzS).
  \label{eq:math.def_covfunc}
\end{equation}

\begin{lem}
  \label{lem:math.covf_sp}
\end{lem}
\begin{proof}
  Since $\abs{\alpha(t)} \le \int J(\omega)\dd\omega = A < \infty$ for $t\in\Reals$, we have for $f,g\in\SchwartzSC$
  \begin{equation*}
    \abs{\varf(f,g)} \le \iint \abs{\alpha(t-s) \cc{f(t)} g(s)} \dd s \dd t
                     \le A \int\abs{f(t)}\dd t \, \int\abs{g(s)}\dd s 
                     < \infty,
  \end{equation*}
  where we used that all test-functions in $\SchwartzSC$ are also integrable.
  Continuity also follows from this inequality, since convergence in the $\SchwartzSC$-sense implies convergence in the $L^1$-sense.
  The linearity and conjugate-symmetry are trivial; therefore we only need to check that $\varf(f,f)=0$ implies $f=0$.
  For this matter we use the well-known fact, that the Fourier transform is a one-to-one, continuous and linear mapping of $\SchwartzSC$ onto $\SchwartzSC$ (see \cite[Theorem 7.7]{Ru91_functional_analysis}).
  Consequently for $f\in\SchwartzSC$ with $\varf(f,f) = 0$, there is a $\ift{f}\in\SchwartzSC$ with $\int\exp[-\ii\omega t]\ift{f}(\omega)\dd\omega = f(t)$.
  A short calculation then reveals
  \begin{align*}
    \varf(f,f) & = \iint \alpha(t-s) \cc{f(t)} f(s) \dd t \dd s \\
               & = \iint  \int\exp[-\ii \Omega (t-s)]J(\Omega)\dd\Omega  \int\exp[\ii\omega t]\cc{\ift{f}(\omega)}\dd\omega  \int\exp[-\ii\omega' s]\ift{f}(\omega') \dd\omega'  \dd s \dd t \\
               & = \iiint \left(   \int \exp[-\ii(\Omega - \omega)t] \dd t  \int \exp[\ii(\Omega - \omega)s] \dd s  \right) \cc{\ift{f}(\omega)}\ift{f}(\omega') J(\Omega) \dd\Omega \dd\omega \dd\omega' \\
               & \propto \int \abs{\ift{f}(\Omega)}^2 J(\Omega) \dd\Omega.
  \end{align*}
  Therefore $\varf(f,f)=0$ implies $\ift{f}=0$, leading to the conclusion $f=0$.
\end{proof}
\begin{rem}
  For the Markovian regime the statement holds true as well, since $\varf$ coincides with the $L^2$-scalar product in that case.\\
\end{rem}

However, we cannot apply \autoref{thm:math.minlos} directly to $\CHF(f)=\exp[-\, \varf(f,f)]$ because it does not apply to the complexified space $\SchwartzSC$.
Instead we will follow the strategy in \cite{Hi80_brownian_motion}: first we restrict $\varf$ to $\SchwartzS$, where it does not necessarily have the form \eqref{eq:math.def_covf}.



We want complex Hilbert space, Minlos only works with real. First define complex one, read of ``real'' sp, complexify again. Define $\alpha$ as FT, Define complex Pre-Hilbert space; define Complexification of $\SchwartzS$
\begin{lem}
  $\sp{\cdot}{\cdot}$ is well defined scalar product, continous on $\SchwartzS_\Complex$
\end{lem}
Complete $\SchwartzS_\Complex$; Hilbert Space $\mathcal{H}_\Complex$
\begin{lem}
  $\mathcal{H}_\Complex$ is seperable
\end{lem}

Go over to $\SchwartzS$, what is scalar product, use polarisation; show that complexified scalar product gives back old scalar product
\begin{thm} \label{thm:mp_existence}
  On $(\dual{\mathcal{S}_\Complex}, \mathcal{B})$ there exists a Gauss Measure $\mu$ such that for $f \in \SchwartzS_\Complex$
  \begin{equation}
    \label{eq:mp_fourier_gauss}
    \int_{\dual{\SchwartzS_\Complex}} \exp{\left(\ii \, \Re(\xi, f)\right)} \dd\mu(\xi) = \exp(-\frac{1}{4} \Vert f \Vert_\mathcal{H})
  \end{equation}
\end{thm}

Define $(L^2)$; scalar product; In the language of probability theory we have prop space $(\dual{\SchwartzS_\Complex}, \mathcal{B}, \mu)$. For $f,g \in \SchwartzS$ we have random variables $\cc{(\cdot, f)}$ and $(\cdot, g)$ such that $\E \ldots$. Can be continued to $\mathcal{H}$; later $Z_t := (\cdot, \delta_t)$; Formally
  \begin{equation*}
  (\xi, f) = \int \xi(t) f(t) \dd t = \int \int \xi(s) \delta(s-t) \dd s f(t) \dd t = \int (\cdot, \delta_t) f(t) \dd t 
  \end{equation*}
Therefore $(\cdot, f) = \int Z_t f(t) \dd t$; Formal scalar product; Example Brownian motion, 

  \begin{thm} \label{thm:mp_expansion}
  $(L^2)$ has the following ONB...
  \end{thm}
Symmetric Fock Space, how to recover microscopical model


%%%%%%%%%%%%%%%%%%%%%%%%%%%%%%%%%%%%%%%%%%%%%%%%%%%%%%%%%%%%%%%%%%%%%%%%%%%%%%%
\section{Noise Creation- and Anhilation Operators}
\label{sec:math.operators}
%%%%%%%%%%%%%%%%%%%%%%%%%%%%%%%%%%%%%%%%%%%%%%%%%%%%%%%%%%%%%%%%%%%%%%%%%%%%%%%

On all elements with finite expansion, $f \in \mathcal{H}$ define $\op Z_f$; calculate Adjoint, Formal notation, define $\op Z_t$




%%%%%%%%%%%%%%%%%%%%%%%%%%%%%%%%%%%%%%%%%%%%%%%%%%%%%%%%%%%%%%%%%%%%%%%%%%%%%%%
%%%%%%%%%%%%%%%%%%%%%%%%%%%%%%%%%%%%%%%%%%%%%%%%%%%%%%%%%%%%%%%%%%%%%%%%%%%%%%%
\chapter{Numerical Implementaion}
\label{cha:implement}


\bibliographystyle{alpha}
%\bibliographystyle{plain}
\bibliography{references}

\newpage
\section*{\centering{Danksagung}}\bigskip
\thispagestyle{empty}

An erster Stelle möchte ich mich natürlich ganz herzlich bei Herr Prof. Strunz für die Betreuung während der Diplomarbeit bedanken.
Sie haben es mir durch die Freiheit bei der Aufgabenstellung ermöglicht, den Verlauf dieser Arbeit größtenteils selber zu bestimmen.
Außerdem haben Sie mich in zahlreichen Diskussionen immer wieder auf den richtigen Weg zurückgebracht.
Weiterhin möchte ich bei Herr Dr. Großmann dafür bedanken, dass er sich als Zweitgutachter zur Verfügung gestellt hat.\\

Alexander Eisfeld möchte ich für die vielen Gespräche, Korrekturen und Hinweise danken, die maßgeblich zum Entstehen dieser Arbeit beigetragen haben---auch wenn mein erster Entwurf allen deinen Veröffentlichungen widersprach.
Gerhard Ritschel möchte ich für seine Hilfe bei vielen Details der Rechnungen und für seine Unterstützung bei allem was mit Temperatur zu tun hat bedanken.\\

Natürlich wäre diese Danksagung nicht vollständig ohne eine Erwähnung der ganzen TQO-Arbeitsgruppe.
Insbesondere möchte ich mich bei Franziska Peter bedanken, auch wenn ich nicht sagen darf wofür.
Lena Simon möchte ich für das Korrekturlesen in sprichwörtlich letzter Minute danken.
Bei Richard Hartmann möchte ich mich für die Diskussionen und Erkenntnisse zum Thema dieser Arbeit bedanken, auch wenn ich glaube dass $\gamma \neq 0$ gelten sollte.\\

Bei meinen Eltern möchte ich herzlichst für die tatkräftige Unterstützung während der Studienzeit bedanken.





% Alex
% Gerhard
%
% Franzi, Lena
% Richard
% Rest der Arbeitsgruppe
%

\newpage
\selectlanguage{ngerman}
\thispagestyle{empty}
\section*{\centering{Erklärung}}\bigskip

Hiermit versichere ich, dass ich die vorliegende Arbeit ohne unzulässige Hilfe Dritter und ohne Benutzung anderer als der angegebenen Hilfsmittel angefertigt habe. Die aus fremden Quellen direkt oder indirekt übernommenen Gedanken sind als solche kenntlich gemacht. Die Arbeit wurde bisher weder im Inland noch im Ausland in gleicher oder ähnlicher Form einer anderen Prüfungsbehörde vorgelegt.\\[2.3cm]
Daniel Süß\\
Dresden, \today
\end{document}
